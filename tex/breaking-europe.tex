\chapter{The Sundering of Europe}

\section{The Fall of Rome}

In the fourth century, Rome's influence in the Western world was unparalleled.
The empire had conquered much of the known world by the time of Constantine,
and even though it was split into multiple parts, it held quite a bit of sway
in Europe and in Northern Africa.
So, the fall of the Western Roman Empire in the fifth century was a big turning point in history.

In 378, the Goths were invading the Roman empire (they were running from the Huns),
and the Roman emperor \textbf{Valens} fought them at \textbf{Adrianople}.
Prior to the battle, Valens had given the \textbf{Visigoths} the status of foederati, and let them live across the Danube.
They had even helped the Romans fight the Huns.
But now, the invading army was comprised of:
the Thervings (Visigoths), led by Fritigern;
Bacurius the Iberian;
and the Greutungs, led by Alatheus and Saphrax.
Valens died at the battle (a result of being abandoned by his guards),
and the new emperor Gratian came to power as a result.
The Goths won a decisive victory, striking a resounding blow against the might of the Romans.
Adrianople is often considered the beginning of the fall of western Rome,
despite the actual battle being fought by soldiers in the Eastern Roman Empire.

Gratian's reign ended in 383 when Magnus Maximus, emperor in the West, attacked Gaul and killed him.
\textbf{Theodosius~I} became emperor.
Theodosius fought these civil wars with the West,
and upon defeating Maximus, he appointed Valentinian II as Augustus in the West.

During the reign of Theodosius, a half-Vandal Roman general named \textbf{Stilicho}
was appointed guardian of the young future emperor \textbf{Honorius}.
At this point, \textbf{Alaric} became the leader of the Visigoths,
the same people known as the Thervings at Adrianople, after the death of Fritigern.
The Visigoths had separated from the Ostrogoths, who had gone their own way.
Alaric tried to rebel a couple times, but he was stopped by Stilicho.
Stilicho made peace with Alaric, who joined the Romans and was no longer considered an enemy.

In 410, Stilicho was disgraced (a victim of conspiracy) and hunted down and killed.
Alaric was again an enemy of the empire, so he went on the offensive.
At this time, Rome was not really the capital of the Western Roman Empire
(oddly enough, that title had been given to Ravenna),
but it was still considered an eternal city, and was still a central point of the empire.
Now, Rome itself had not been touched by any enemy since 387 BC, when it had been sacked by Brennus.
First, Alaric invaded Greece, killing many and wiping out the Eleusinian Mysteries.
Alaric invaded Italy, and besieged Rome, and tried to negotiate, three times.
On the third siege, Alaric got annoyed and entered the city,
pillaging and ransacking a lot of the big important buildings,
but on the whole, not causing a particularly massive amount of damage.
The sacking prompted Augustine to write \textit{City of God}.

Meanwhile, elsewhere in the empire, Honorius's lands were breaking.
Constantine~III, last of the imperial usurpers, managed to raise an army in Britain, and march on Gaul.
He defeated Sarus (who was loyal to Honorius), and co-ruled with Honorius for a time.
The Visigoths went west from Italy, and ruled Spain from their capital at Toledo.
The Vandals settled in the south of Spain.
Other barbarian tribes settled into parts of Spain and Gaul, splitting the empire even further.

In 444, the Huns were united by \textbf{Attila} (the ``Scourge of God''),
who also included many Germanic tribes in his subjects.
Attila had taking the kingship after his brother Bleda had died.
He kept attacking the Eastern empire until 450, making money,
but always stopping at the walls of Constantinople.
He claimed the Emperor's sister Honoria as his wife,
and he invaded Gaul in 451 to get his ``dowry'', which was half of the Western Roman Empire.
The invasion was stopped by Flavius Aetius and the Visigoths under Theodoric~I,
at the Catalunian Plains and at Chalons.
Allegedly, Pope Leo~I convinced Attila not to attack Rome ---
soon after, Attila died of a nosebleed in his sleep.

By 439, the \textbf{Vandals} had crossed the Strait of Gibraltar, moved east, and taken Carthage from Rome.
They set up a new kingdom in Northern Africa, where they would reside for a century.
\textbf{Genseric} was their king, and his son the prince was named Huneric.
After going back and forth trying to attack Italy, they got their true chance when Valentinian killed Aetius,
and in 455, the Vandals managed to enter Rome itself, and plunder it for two weeks.
Now, their name is obviously synonymous with someone who defaces property.
They sailed away with lots of gold, and conquered Sicily.
From this point onwards, their ships were a menace to the Romans.

In 476, a former soldier named Odoacer had invaded Rome.
He deposed the 16 year old emperor Romulus~Augustulus, but did not kill him.
Odoacer became the first ruler of Italy after Rome, and ended the Western Roman Empire.

\section{Building a Holy Roman Empire}

While Rome's world was collapsing around it, some new dynasties were being created
in what used to be the province of Gaul.
The region was now called Francia, and it was ruled by the Franks.
