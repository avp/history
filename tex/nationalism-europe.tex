\chapter{European Nationalism}

\epigraph{%
  Patriotism is when love of your own people comes first;
  nationalism, when hate for people other than your own comes first.
}{Charles de Gaulle}

While the United States was first getting started in the 19th century,
many parts of Europe were experiencing huge changes of their own.

\section{Italian Unification}

\subsection*{Origins}

Napoleon had wreaked havoc on Italy (along with the rest of Europe).
With his fall, the old power structures of Italian city-states had fallen apart,
and nationalism was on the rise.
The process of Italian unification, called \textbf{\textit{Risorgimento}},
began after the Congress of Vienna.

In 1820, the Two Sicilies Insurrection, a mutiny by soldiers in the Kingdom of the Two Sicilies,
took place against the reigning king Ferdinand~I.
Later, during the Piedmont Insurrection,
Santore di Santarosa wanted to unite Italy under the House of Savoy.

By 1830, the Italians began to look more favorably on the idea of unifying Italy.
The Duke of Modena, Francis~IV,
was an ambitious man who wanted to unite northern Italy under his rule.
His supporters were \textit{Carbonari}, and he arrested Ciro Menotti for a planned uprising.
In 1831, however, Pope Gregory~XVI asked for Austrian help against rebellions across the peninsula.
The Austrian armies went across Italy, suppressing rebellion.

\subsection*{Revolutions and Independence Wars}

In 1848, revolution in Lombardy began in earnest.
Shortly afterwards,
revolts broke out in Sicily and Naples, putting Ruggero Settimo in power in Sicily.
Unrest in Tuscany resulted in Grand Duke Ferdinand granting the Tuscans a constitution.
Pope Pius~IX gave a constitution to the Papal States as well.

Milan was besieged by the Austrians under \textbf{Marshal Josef Radetzky}, but he failed and retreated.
Eventually, Charles Albert, King of Sardinia,
declared the \textbf{First Italian Independence War} on Austria.
Radetzky won at the Battle of Custoza and took Lombardy,
except for Venice, where the Republic of San Marco was established.

The \textbf{Second Italian Independence War} started in 1859.
The Sardinian Prime Minister, \textbf{Camillo Benso, Count of Cavour} (colloquially Camillo Cavour),
allied with Napoleon~III\@.
The last major battle of the war was the \textbf{Battle of Solferino} and San Martino.
It was the last major battle in which armies were personally commanded by monarchs.
The battle resulted in a Franco-Sardinian victory,
and leaders at the battle included General Ludwig von Benedek and Franz von Gyulai.
Jean-Henri Dunant wrote \textit{A Memory of Solferino},
and he started a discussion that would end with the creation of the Geneva Convention,
as well as the inception of the \textbf{International Red Cross}.
Sixteen days after the battle, the \textbf{Peace of Villafranca} was signed.

\subsection*{A Unified Italy}

\textbf{Giuseppe Garibaldi} was a native of Nice.
He had been part of the Piedmont uprising, after which he went to South America for a while,
returning to Italy in 1848.
His defense of Montevideo during the Uruguayan Civil War against Jan Manuel do Rosas,
combined with his following heroics in Italy,
resulted in people calling him ``The Hero of Two Worlds''.
He'd also fought in the Ragamuffin War (also called the War of Tatters) in southern Brazil,
and he almost fought in the American Civil War,
but didn't because Lincoln didn't make abolishing slavery the stated goal of the war.

After the Revolutions of 1848, Garibaldi became the hero of the unification movement.
He was quite popular in the south, but in the north,
the House of Savoy was in power in the Kingdom of Piedmont-Sardinia.

Garibaldi's \textbf{Redshirts}, the Italian Legion,
were a volunteer force that Garibaldi had formed in Uruguay,
when Garibaldi was fighting its Civil War.
Their flag was black, with a volcano in the middle.

By the middle of 1860, there were few states remaining in Italy.
Garibaldi wasn't happy with the fact that France had taken Nice, so he wanted to take it back.
Cavour convinced Garibaldi to instead direct his forces south, toward the Sicilian rebellions.
In the \textbf{Expedition of the Thousand} (Mille Expedition),
Garibaldi took a his Redshirts and landed at Marsala on the coast of Sicily.
Following battles such as \textbf{Calatafimi},
Garibaldi advanced on Palermo, taking it and inciting Ferdinando Lanza.
He proceeded to topple the Kingdom of the Two Sicilies.

Garibaldi proceeded to subdue the rest of Italy, and retired to Caprera.
In 1861, Italy became a true nation-state,
unified under \textbf{Victor Emmanuel~II}, House of Savoy, of Piedmont.
Cavour became his Chief Minister.

\textbf{Giuseppe Mazzini} was another notable radical in the unification movement.
He started the political society called \textit{La Giovine Italia} (Young Italy),
whose motto was ``God and the People''.
One of his followers was Felice Orsini, who tried to kill Napoleon~III (more on that later).
Mazzini was Chief Minister in Rome in 1849, before Garibaldi's Expedition.
After unification, Mazzini wasn't happy with the fact that Italy was still a monarchy.

\section{French Regimes and Leaders}

\subsection*{Bourbon Restoration}

After Napoleon was finally subdued, France fell back into old monarchic habits for a while.
Some revolutions would take it back out of that and into modernization, though.

\textbf{Louis~XVIII}, brother of Louis~XVI, ruled from 1814 to 1824.
He created the Charter, which kept some of the advances of the French Revolution.
The revived absolute monarchy had a parliament,
consisting of a Chamber of Deputies and a Chamber of Peers.

\subsection*{July Revolution}

In 1830, \textbf{Charles~X} of Bourbon was ruling over France.
He'd succeeded Louis~XVIII, and opinion of Charles quickly went south.
In addition,
Charles's minister \textbf{Jules~de~Polignac} had invaded Algeria and established rule there.

On March 17,
the majority in the Chamber of Deputies gave the \textbf{Address of the 221}
(a no confidence motion).
Charles got mad and proceeded to dissolve Parliament and the National Guard of Paris.
Adolphe Thiers created \textit{Le National} and an association called
``Help yourself, and Heaven will help you''.

At this point, revolution was right around the corner.
When Charles signed the July Ordinances, also called the \textbf{Ordinances of Saint-Cloud},
he suspended freedom of the press.
The revolution began on July 27.

The opening of the \textbf{July Revolution} is a time known as the Three Glorious Days,
during which opposition to Polignac led to barricades being built all over Paris.
By the end of the July Revolution,
France was placed under a constitutional monarchy, and Charles~X abdicated.

\subsection*{July Monarchy}

In Charles's place was \textbf{Louis Philippe~I}, the ``Citizen-King'' of France,
who'd been picked by the National Assembly over Henry~V.
Notably, he had defected to Austria with Charles Francois Dumouriez in 1793.
He ruled over the \textbf{July Monarchy} from 1830 until 1848.

Louis Philippe wanted a ``just milieu'' between royal power and popular pressure.
Under Louis Philippe, the \textit{haute bourgeoisie} were quite dominant.
The conservatives in the House of Deputies were led by Francois Guizot.
The center-left faction was led by \textbf{Adolphe Thiers}, who would become president in 1871.
In a notable political cartoon, Louis Philippe was shown transforming into a pear.
He sent Marshal Soult to put down the Canut Revolt in Lyon.

The king survived seven assassination attempts while in power.
In 1835, while in Austria, Louis Philippe was attacked by Giuseppe Fieschi,
who built an ``infernal machine'',
a gun that was 25 gun barrels tied together.
Fieschi just ended up killing 18 other people while the king went through relatively unscathed.

Because so many people were trying to end the monarchy, it began to rule more harshly.
By 1848, it fell apart as people revolted against the king.
Louis Philippe abdicated and the \textbf{Second Republic} was formed.

\subsection*{Republic and Empire}

By the end of 1848,
\textbf{Louis Napoleon Bonaparte} was overwhelmingly elected president of the Republic.
He had previously tried to overthrow the government multiple times.
During one attempt, he hired a ship called the \textit{Edinburgh-Castle} and sailed to Boulougne,
but he failed miserably, was arrested, imprisoned in Fort Ham, and ridiculed vigorously.

His prime minister was Odilon Barrot, who nobody objected to because he was so neutral.
Louis Napoleon sent forces under Nicolas Oudinot to the Papal State because the Pope had run away.
He also signed the Falloux Law, which restored some power for the Catholic Church and its schools.

In 1851, Louis Napoleon wasn't allowed to run for president again.
Naturally, he decided to stage a coup and make himself President for Life.
By the following year,
he had created the \textbf{Second Empire} and made himself \textbf{Napoleon~III} of France.
As noted in the other section,
Napoleon~III colluded with Cavour in Italy and signed the Treaty of Villafranca,
following heavy losses at Magenta and Solferino.

Napoleon~III reconstructed Paris with his prefect of the Seine, \textbf{Baron Haussmann}.
He picked Emile Ollivier as his prime minister,
installed Maximilian as ruler in Mexico,
and commissioned Charles Garnier to build the Paris Opera House.
He wrote \textit{The Extinction of Pauperism},
and noted that ``history appears the first time as tragedy, the second time as farce''.

In 1858, \textbf{Felice Orsini} and friends tried to kill Napoleon~III
(the event is sometimes called the Orsini Affair).
One of the accomplices posed as a Portuguese beer salesman.
They failed, but they did manage to hit his carriage,
also containing his wife Eugenie de Montijo, with three bombs.

Napoleon~III was captured at the Battle of Sedan during the Franco-Prussian War,
which we'll look at in detail later on.
After the war, the monarchy was overthrown, and the \textbf{Third Republic} began,
with Adolphe Thiers as the new president.

In 1894, Jewish artillery captain \textbf{Alfred Dreyfus} was convicted of treason
and imprisoned on Devil's Island.
Two years later,
an investigation by Georges Picquart identified Francis Esterhazy as the actual culprit.
High ranking military officials suppressed the new evidence,
convicting Dreyfus on a few more charges instead.
The incident and the cover-up began to spread,
prompting outrage, such as in Emile Zola's \textit{J'accuse}.
By 1906, Dreyfus was exonerated.

\section{Prussia Under The Iron Chancellor}

\subsection*{Otto von Bismarck}

\textbf{Otto von Bismarck} is called the ``Iron Chancellor'' for a reason.
He played a big part in unifying Germany in the 19th century,
and his influence would be felt for years to come.

\subsubsection*{Foreign Policy}

In 1862, Bismarck was appointed Minister President of Prussia by \textbf{King Wilhelm~I}.
His diplomatic strategy was called \textit{realpolitik},
and he started small wars with other countries.
In that same year, he made the ``blood and iron'' speech,
discussing how those two things would be used to improve Prussia's role in the world.
When Frederick~VII of Denmark died in 1863,
Bismarck was in the middle of a dispute over Schleswig-Holstein.
This started the Second Schleswig War.

During the \textbf{Austro-Prussian War} (Seven Weeks' War),
Austria went back on the agreement that ended the Schleswig War.
The war lasted seven weeks; the Prussians referred to it as ``blitzkrieg''.
At the \textbf{Battle of Koniggratz}, the Prussians won a decisive victory,
owing to the elder von Moltke's use of needle guns.
Bismarck stopped his troops from continuing,
and reestablished good will towards Austria with the Peace of Nikolsburg.

In 1872, Bismarck and Harry von Arnim, ambassador to France,
started to disagree because Arnim wanted to be chancellor.
Bismarck sent Fritz von Holstein to spy on Arnim, who took sensitive papers and fled.
He also used the Gastein Convention to push borders to the north.

In 1873, Bismarck created the League of the Three Emperors,
along with Tsar Alexander~II and Francis Joseph of Austria-Hungary.
He aim was to keep control of Eastern Europe
and control other groups that weren't part of their three countries,
such as Poland.

\subsubsection*{Domestic Issues}

Bismarck was also very strong in his domestic policy.
During the 1871 \textbf{Kulturkampf}, he waged a ``culture struggle'' on the Catholic Church.
He didn't want Pius~IX and friends taking control using papal infallibility.

The May Laws (Falk Laws) of 1873, enforced by Adalbert Falk,
closed many seminaries and further weakened the Church.
Many bishops and priests were jailed.
Kulturkampf was opposed by Ludwig Windthorst, leader of the Center Party.
The policy was abandoned in 1878
because the Catholics got organized and Bismarck decided that it wasn't worth it.

At one point, Bismarck was almost assassinated by Ferdinand Cohen-Blind, a student.
Also notably, Bismarck was so good with foreign policy that he predicted World War~I to the month,
saying ``the crash will come twenty years after my departure''.
He would be succeeded as Chancellor by Leo Caprivi.

\subsection*{Franco-Prussian War}

When Prussia defeated Austria in the Seven Weeks' War, France got scared.
Nap\-oleon~III didn't want Germany to tip the balance of power.
Bismarck wanted war with France,
but he needed it to be France's fault so Germany would be behind him.

In 1870, Bismarck published the \textbf{Ems Dispatch},
an edited conversation between Wilhelm~I and Count Benedetti (French ambassador to Prussia).
The idea had come about as a consequence
of Leopold of Hohenzollern being offered the Spanish throne,
and it precipitated a whole new war.

France declared war in July 1870, and each side saw the other as the aggressor.
Bismarck's Chief of Staff, Moltke the Elder, led the German army to lots of victories.
After battles such as \textbf{Sedan} and Metz, the Germans had a lot of momentum.
They captured Napoleon~III at Sedan,
and the rest of the war was mostly a siege of Paris, but it didn't work very well.

The war provided Bismarck with what he needed to unify Germany.
By the end of the war, Wilhelm~I was declared German Emperor and the new German Empire was created.
The end of the war also saw France surrender Alsace-Lorraine, and they had to pay an indemnity.

\section{Twilight of the Tsars}

In Russia, the Romanovs were reigning through the 19th century.
At the beginning of the 20th century, the Tsardom would be abolished,
so let's examine the last of the Romanovs.

\subsection*{Alexander~I}

\textbf{Alexander~I} (1801--1825), the Blessed, son of Paul~I, was around during the Napoleonic Era.
Some claim that he knew about the plot to assassinate his father and bring himself to power.

He signed the Treaty of Tilsit with Napoleon on a raft in the Niemen River,
following the Battle of Friedland in the War of the Fourth Coalition.
Alexander~I also fought at Leipzig during the War of the Sixth Coalition.
He increased Russia's territory following the Finnish War,
and he became King of Poland due to the Congress of Vienna, and Napoleon's fall.

\subsection*{Decembrist Revolt}

Alexander~I was succeeded by \textbf{Nicholas~I} (1825--1855).
Most notably, he faced the \textbf{Decembrist Revolt} in 1825.
The Decembrists aimed to put either Sergey Trubetskoy or Constantine Pavlovich in charge.
Trubetskoy ran away, and Pavlovich didn't show.
Supporters were thrown into the Neva River.
Leaders of the revolt formed the Union of Welfare, the Union of Salvation,
and the Northern and Southern Societies.
Pavel Pestel, leader of the Southern Society, was hanged at the end of the revolt.
At one point, a cavalry charge ended in failure when the horses slipped on ice.
After the revolt, the Chernigov Regiment tried to mutiny.

\subsection*{Crimean War}

Nicholas~I also was tsar when Russia entered into the \textbf{Crimean War} (1853--1856),
on the namesake peninsula on the Black Sea,
against the allied forces of Britain, France, the Ottoman Empire, and Sardinia.
Causes of the war included religion in the Holy Land,
as well as the fact that everyone wanted land in the failing Ottoman Empire.
The London Straits Convention preceded open war, which the Ottomans declared in 1853.

A notable engagement is the \textbf{Battle of Balaclava}.
The Allied forces arrived from the village of Kamara, and the Russians set up on Woronzov Heights.
The battle is most well known for the \textbf{Charge of the Light Brigade},
in which \textbf{Lord Raglan} sent cavalry under the \textbf{Earl of Cardigan}
into the Valley of Death,
which was defended by Pavel Liprandi.
As you can guess by the name of the location, the Charge didn't end well for the British troops.
Notably, Colin Campbell's 93rd Highland Regiment formed the ``Thin Red Line'' and repelled Russian attacks.
The battle was memorialized by Alfred Lord Tennyson in his poem ``The Charge of the Light Brigade''
and by Iron Maiden in ``The Trooper''.

Nonmilitary personnel in this war were also important.
\textbf{Florence Nightingale} (the lady with the lamp)
and Mary Seacole served as nurses during the war.
Journalists the photographer Roger Fenton and the journalist William Howard Russell.

Other important battles include the Russian defeat at Inkerman,
during which the Russian forces were in heavy fog and ended up going the wrong way.
Lesser battles were fought at Alma and Sinope.
At the \textbf{Siege of Sevastopol},
French and British forces realized that Sevastopol was the key to the Black Sea.
Both sides lost many troops to disease, and Lord Raglan himself died of dysentery.
Sevastopol was the last major decisive battle of the war.
The 1856 Treaty of Paris ended the Crimean War.

\subsection*{Alexander~II}

During the Crimean War,
Nicholas~I died and his son \textbf{Alexander~II}, the Liberator, succeeded him.
Alexander was tsar at the conclusion of the war and signed the Treaty of Paris.
His wife was Marie of Hesse, and his mistress was Catherine Dolgorukov.
During his reign, he signed the Dictatorship of Heart.
He also waged the Russo-Turkish War, after which he signed the Treaty of San Stefano,
attended the Congress of Berlin, and revised the treaty into the Treaty of Berlin.

The most important thing Alexander~II did was issue a ukase to emancipate the Russian serfs in 1861.
This reform led him to be called Alexander the Liberator.
He also reorganized the judicial system
and created a system of \textbf{zemstvos} for local government,
with help from Nikolay Milyutin.
He started mandatory military service for everyone, even nobles,
and he was helped in military reform by Dmitry Milyutin, Nikolay's brother.
His secret police was called the Third Section (Third Department),
and they exiled lots of people to Siberia.
Alexander was the first tsar with a beard since Peter.

There were people who didn't really like Alexander, and after an assassination attempt,
Count Loris-Melikov was appointed head of the Supreme Executive Commission
and given power to fight the rebels.
In 1881, Alexander was traveling to Mikhailovsky Manege for roll call.
On the side of the street, Nikolai Rysakov,
a member of the \textbf{People's Will} (Narodnaya Volya),
had a package in his hand.
When he threw the bomb, it failed to penetrate the bulletproof carriage (a gift from Napoleon~III).
A second bomber, Ignacy Hryniwiecki, threw his bomb at the tsar and mortally wounded him.
Later, the Church of the Savior on Blood was built where Alexander had been killed.

\subsection*{Alexander~III}

\textbf{Alexander~III}, son of Alexander~II, the Peacemaker, succeeded his father as tsar.
While he was tsar, Russia fought in no big wars.
Notably, Alexander passed the \textbf{May Laws} (1882),
which prevented Jewish people from inhabiting many rural areas,
including shtetls and the Pale of Settlement
(where they previously had been expressly allowed to stay).

\subsection*{Nicholas~II}

\textbf{Nicholas~II} reigned from 1894 until the revolutions
that forced the downfall of the tsar in 1917.
In 1891, while in Otsu, Japan, he had been the target of an assassination attempt.
When Nicholas took the throne, Russia was one of the biggest powers in the world.
His advisors included Prime Minister Peter Stolypin and Sergei Witte,
and his Minister of the Interior was Alexander Protopopov.
His government was notably anti-Semitic,
and his secret police forged documents proving that Jews would conquer the world.

Nicholas's coronation was held in Uspensky Cathedral on Khodynka Field,
and free beer and cups were served.
Rumor spread that there wouldn't be enough beer for everyone, and this being Russia,
the crowd trampled each other to get their share, suffocating and killing over a thousand people.
The incident became known as the \textbf{Khodynka Tragedy}.

Nicholas's son, \textbf{Alexei}, had hemophilia.
\textbf{Tsarina Alexandra} wanted him cured,
and the best mystic she found was \textbf{Grigori Rasputin}, the ``Mad Monk'',
who was recommended by Anna Vyrubova, whose life he had saved earlier.
People sometimes claim that Rasputin had associations with the khlysty group.
Eventually, people like Oswald Rayner decided to kill Rasputin.
They poisoned him.
Then they shot him.
When that didn't work, they dumped him into the Neva River, and he likely drowned.
Later, Rasputin's body was dug up and burned.

\subsection*{Russo-Japanese War}

Nicholas and the Russians had been moving east for a while,
and war with Japan was a natural consequence.
When the Japanese attacked Port Arthur in 1904, they preemptively declared war on Russia.
The ensuing \textbf{Russo-Japanese War} (1904--1905)
was of particular importance to this part of Nicholas's rule.

Now, the part of the Russian fleet
that wasn't stuck at Port Arthur was all the way in the Baltic Sea.
It's a nine-month journey to the east side of Russia,
and Britain wasn't letting Russia use the Suez Canal,
so Nicholas had to bring his ships around the long way to help at Port Arthur.
While the Baltic Fleet was moving, the biggest land battle occurred at Mukden.
When the fleet finally arrived,
they faced the Japanese under Togo Heihachiro at the \textbf{Battle of Tsushima Strait},
and the fleet was almost destroyed immediately.
Other notable battles include Motien Pass and Ulsan.
The Yalu River was a notable location of land battles while crossing it,
while the Battle of the Yellow Sea included a blockade of Russian forces.

The war was ended by the \textbf{Treaty of Portsmouth}.
Theodore Roosevelt helped negotiate the treaty, along with Sergei Witte and Komura Jutaro.

\subsection*{Revolution in 1905}

The \textbf{Russian Revolution of 1905}
was an empire-wide revolt that aimed to take down the tsardom,
partially instigated by the humiliation that Russia had suffered following the Russo-Japanese War.
As the revolution grew, it included a strike on the Trans-Siberian Railroad,
as well as the Potemkin Mutiny.
\textbf{Father Georgy Gapon} organized marchers on \textbf{Bloody Sunday},
calling for the Assembly of Russian Factory Workers to march on the Winter Palace.

The revolution caused Nicholas to issue the \textbf{October Manifesto},
a document that tried to create a new constitutional monarchy in Russia.
Written by Sergei Witte,
the manifesto gave veto powers to the Duma, and he allowed for more liberty in expression.
The document gave its name to the \textbf{Octobrists}, led by Alexander Guchkov,
who would end up dominating the Third and Fourth Dumas.
However, the manifesto was repealed just a year later by the Fundamental Laws.

\section{Victorian England}

\subsection*{Queen Victoria}

Queen \textbf{Victoria}, House of Hanover, Empress of India,
came to power in 1837 after William~IV\@.
Victoria had been raised by Sir John Conroy and the Duchess of Kent,
her mother, under the Kensington System.
She married Prince Albert of Saxe-Coburg Gotha in 1840, and she had 9 children,
giving her the nickname of ``grandmother of Europe.''

Victoria reigned for 63 years, making her the second longest reigning English monarch.
When she died in 1907, she was succeeded by her son \textbf{Edward~VII}.

\subsection*{Robert Peel}

\textbf{Robert Peel} was a noteworthy Conservative prime minister
around the early years of Victoria's reign.
He began the Bedchamber Crisis in 1841
when he suggested that Victoria replace some Whig advisors with Conservative ones.
In 1843, an insane Scot named Daniel M'Naghten tried to kill Peel,
but ended up killing Peel's personal secretary Edward Drummond instead.

Peel repealed the \textbf{Corn Laws} in 1842,
and reintroduced the income tax among his other financial reforms.
The repeal of the Corn Laws was opposed to typical Conservative ideology,
and the incident resulted in loss of his ministry.

\subsection*{Disraeli and Gladstone}

Two rivals in Parliament were known for shaping the country during Victoria's reign.

The Conservative \textbf{Benjamin Disraeli}
remains the only Prime Minister to have been of Jewish descent.
When he was getting started, he'd published the newspaper \textit{The Representative},
and he'd led the Young England movement with George Smythe.
Under the Earl of Derby, he'd served as Chancellor of the Exchequer three times,
including during the ``Who? Who? Ministry''.
He represented Britain at the Congress of Berlin,
and he was supported by Queen Victoria, who made him an earl.
Disraeli made Victoria Empress of India using the Royal Titles Act of 1867,
When Disraeli purchased 44\% shares in the Suez Canal Company,
he took a loan from Lionel de Rothschild and family and passed the 1867 Reform Act.

Victoria didn't like the Liberal \textbf{William Gladstone}, a ``Grand Old Man'', very much.
As Prime Minister, Gladstone introduced two bills for Irish Home Rule, but they were rejected.
During his \textbf{Midlothian Campaign} against Disraeli,
Gladstone denounced atrocities that were taking place in the Ottoman Empire,
in a pamphlet called ``Bulgarian Horrors and the Question of the East''.
His Secretary of State Edward Caldwell reformed the military (Caldwell Reforms),
and he ``invincibles'' stabbed Lord Cavendish during his ministry.
After Mahdists in Khartoum killed ``Chinese'' Gordon during the Mahdist War,
Queen Victoria held Gladstone responsible.

\section{European Imperialism}

\subsection*{Scramble for Africa}

The \textbf{Berlin Conference} (1884--1885) was the formalization of the Scramble for Africa.
Europeans got together and carved up the continent in a manner that they felt suited them.
It was organized by Otto von Bismarck, and it resulted in the General Act of the Berlin Conference.

\subsubsection*{Leopold and Congo}

\textbf{Leopold~II} of Belgium (1865--1909) was given some land in south central Africa,
and he established the \textbf{Congo Free State}.
The Casement Report went over the Rubber Atrocities that Leopold inflicted
on workers who weren't able to produce enough rubber for him, including things like hand amputation.
Leopold's personal army was called the Force Publique,
and he allegedly buried slaves alive and sold them to be eaten.

Leopold also sent \textbf{Henry Morton Stanley} into Africa to find \textbf{David Livingstone}.
When he found Livingstone,
Stanley asked, ``Dr.\ Livingstone, I presume?'', a quote that is now famous for some reason.
Stanley had earlier led the Emin Pasha Relief Expedition into central Africa;
the expedition is now notorious for its ambition and the large number of deaths on the way.

\subsubsection*{Portuguese in Africa}

Portugal took states on the coasts of Africa:
Angola,
Mozambique,
Guinea-Bissau,
Cape Verde,
and Sao Tome and Principe.
The states were called PALOP and Portugal took them
because it has lost a lot of land in South America.

\subsubsection*{All Rhodes Lead to Africa}

\textbf{Cecil Rhodes} was born in 1853, and he was sent to South Africa as a child.
He entered the diamond industry, and in 1888,
he founded the \textbf{De Beers diamond company}.
A year later,
his British South Africa Company received a charter from the crown
to exploit mineral wealth on the continent.

Rhodes became a giant in the African mining industry,
gaining political appointments and large amounts of wealth.
His exploitative business tactics often placed him in a moral gray area.
For example, with his business partner Charles Rudd,
Rhodes convinced King Lobengula of Matabeleland to sign the \textbf{Rudd Concession},
predicated on the false assumption that at most 10 white men would mine in Matabeleland.
Lobengula's attempts to back out of the deceptive treaty fell on deaf ears.

Rhodes wanted to connect Africa from north to south,
and he envisioned a ``Cape to Cairo'' railway that would facilitate this.
Unfortunately for Rhodes, Belgium, Germany, France,
and the other European powers on the continent prevented this dream from becoming reality.

\subsection*{Boer Wars}

\subsubsection*{First Boer War}

The First Boer War doesn't come up very often, but it happened in 1880, ending promptly in 1881.
The war was between Britain and the Boers (settlers in South Africa).
It was caused by the annexation of the \textbf{Transvaal Republic} in 1877 by Britain.
The major battle to know here is the Battle of Majuba Hill, a decisive Boer victory.

\subsubsection*{Second Boer War}

The \textbf{Second Boer War}
(this is the important one, mostly just called the ``Boer War'') started in 1899.
The UK (under Horatio Kitchener) fought the South African Republic and the Orange Free State.
It started after a conference between High Commissioner Alfred, Lord Milner (UK)
and Paul Kruger (president of the Transvaal)
fell apart at Bloemfontein.
It was partly caused when the \textbf{Jameson Raid}
failed to get uitlanders to rise up in the Transvaal.
Following the raid, Rhodes was forced to step down as Cape prime minister.

The first major battle was at Talana Hill,
after which the British retreated from Dundee following the death of William Penn Symons.
During the war, \textbf{Robert Baden-Powell} held Mafeking under a siege,
using artillery such as ``the Wolf'' and ``Lord Nelson'';
other battles included the Black Week and the Siege of Ladysmith.
The war marked the first use of concentration camps,
which were condemned by the Fawcett Commission under Emily Hobhouse.
Kitchener used scorched-earth tactics to combat guerrilla fighters led by Kruger and Louis Botha.

The war was ended by the \textbf{Treaty of Vereeniging},
and the Transvaal and the Orange Free State were brought under British sovereignty.
Reconstruction was handled by Milner and ``Milner's Kindergarten'',
a group of Oxford-trained civil servants.

\subsection*{Britain in India}

British company rule in India had begun in 1757 following the \textbf{Battle of Plassey},
at which Robert Clive and the British East India Company defeated the Nawab of Bengal.
Over the following century, the British took over the entire subcontinent.

In 1857,
Indian conscripts of the British army began to hear rumors
that cartridges of the Enfield rifles they used
were coated with pig and cow fat.
This, along with other factors, led the \textbf{Sepoy Mutiny} to break out in Meerut.
The East India Company set up in Delhi, where a siege led to the arrest of Bahadur Shah~II\@.
William Hodson had Bahadur Shah's children shot at the Bloody Gate.
A siege at Lucknow was relieved by forces under Henry Havelock and Colin Campbell.
The next siege at Kanpur was led by Tatya Tope,
and Havelock and Campbell moved their relief column to end the siege.
The rebels were defeated because they weren't very organized and they had no clear goals,
but the large scale of the revolt made the British government
take control of the region from the companies.
Thus ended company rule in India, leading to the start of the British Raj.
