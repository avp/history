\chapter{World War II}

\epigraph{%
  I hate war as only a soldier who has lived it can,
  only as one who has seen its brutality, its futility, its stupidity.
}{Dwight D.\ Eisenhower}

\section{Prelude: Expansionism}

\subsection*{Germany's \textit{Lebensraum}}

The chancellor of Austria from 1932 to 1934 was \textbf{Engelbert Dollfuss}.
When he was killed by Nazis, he was replaced by \textbf{Kurt von Schuschnigg}.
Both Dollfuss and Schuschnigg tried to stop Hitler from taking Austria and bringing it into Germany,
an annexation called \textbf{Anschluss}.
Hitler justified the expansionism by stating that Germany needed Lebensraum, ``living space''.
When Schuschnigg insisted on a referendum being passed,
he was replaced by \textbf{Arthur Seyss-Inquart}, who was placed into the governorship of the new state of Ostmark.
Hitler annexed Austria in 1938, and took down Wilhelm Miklas, president of Austria.

A few months after the Anschluss, Hitler decided he wanted the \textbf{Sudetenland},
a region in Northwestern Czechoslovakia.
At the time, Konrad Henlein was leading a separatist movement in the Sudetenland.
A conference was called in Munich, with Hitler,
British Prime Minister \textbf{Neville Chamberlain}, and French Prime Minister Edouard Daladier.
The Munich Pact resulted in appeasement for Hitler,
and Chamberlain went home declaring ``peace for our time''.
This policy was appeasement was condemned by ``Cato'' in the book \textit{Guilty Men}.

In May 1939, German and Italian foreign ministers Joachim von Ribbentropp and Galeazzo Ciano signed an alliance.
The original name for it was the Pact of Blood,
but Mussolini decided that it would be better received if it was instead called the \textbf{Pact of Steel}.
The pact solidified the main alliance that would make up the \textbf{Axis Powers}.

\subsection*{Japanese Aggression}

To trace the roots of the war in the Pacific, let's go back a bit to Japan's invasion of Manchuria back in 1931.
In September 1931, the Kwantung Army of the Empire of Japan, led by Kanji Ishiwara, invaded Manchuria,
in response to the Mukden Incident (a staged bombing to provide pretext for invasion).
This prompted the creation of the \textbf{Manchukuo} puppet state under \textbf{Pu Yi}, the last Chinese emperor.

Roosevelt's Secretary of War \textbf{Henry Stimson} issued the Stimson Doctrine,
outlining how the US would deal with Manchukuo.
Manchukuo was also home to \textbf{Unit 731}, a covert biological warfare division of the Japanese Army,
which conducted some of the most heinous war crimes of the war,
experimenting on humans, poisoning water, and killing thousands of people.
Unit 731 and its affiliates were known as the Epidemic Prevention and Water Purification Department.

Japan joined the Anti-Comintern Pact in 1936, and in 1940, it joined Germany and Italy to form the Axis Powers.
At this point, the emperor \textbf{Hirohito} was talking about the revival of \textit{hakko ichiu},
the God-given right of Japan's emperor to unite and rule the world.
The USSR fought Japan at the Battle of Lake Khasan (1938) and the \textbf{Battle of Khalkhin Gol} in 1939.
Notably, Georgy Zhukov fought at Khalkhin Gol, assisting Grigory Shtern in defending Baintsagan Hill.
We'll see more of Zhukov in the Eastern Front in Europe.
After the battles, the Soviet-Japanese Neutrality Pact was signed in 1941.

During Japanese occupation of Nanking, they perpetrated the \textbf{Rape of Nanking}.
Between December 1937 and January 1938 (during the Second Sino-Japanese War),
hundreds of thousands of civilians were killed by Japanese occupying forces.
During the massacre, the Nazi John Rabe helped form a safety zone where he sheltered civilians.
Toshaki Mukai won a sword contest in which the objective was to kill a hundred people.
The event was so horrific that many Japanese textbooks still omit it.

\textbf{Hideki Tojo} was the Prime Minister of Japan, and he believed that war with the US would be inevitable.
He convinced Hirohito to allow an attack on America, the UK, and the Netherlands.
Others, such as Admiral \textbf{Yamamoto Isoroku}, disagreed.
Yamamoto claimed that Japan wouldn't last in a prolonged war against the Americans.


\section{Blitzkrieg}

\subsection*{Invading Poland}

On August 23, 1939, Soviet minister \textbf{Vyacheslav Molotov} and Joachim von Ribbentropp signed the
\textbf{Nazi-Soviet Pact}, also called the Molotov-Ribbentropp Pact.
The treaty was formally a non-aggression agreement between Hitler and Stalin,
but it also set up government and partition of territory in Eastern Europe,
notably splitting up Poland preemptively.

On August 31, the Nazis set up a false flag operation,
posing as Poles and staging an attack on a German radio station.
The Germans then used this Gleiwitz Incident, part of the Operation Himmler propaganda campaign,
as a reason to invade Poland.
The Soviet Union invaded shortly thereafter, having come to an agreement with Japan as well.
The Polish, facing a war on two fronts, retreated to Romania, and Poland was occupied.
The Germans then annexed western Poland and the Free City of Danzig.

The invasion of Poland prompted France, Great Britain, and the rest of the Commonwealth to declare war on Germany.
Hitler tried to make peace with the UK and France, saying that they shouldn't interfere with Poland,
but Chamberlain responded that
``Past experience has shown that no reliance can be placed upon the promises of the present German Government.''
As a result, Hitler ordered an attack on France, which would commence in the spring of 1940.

\subsection*{Winter War}

In November 1939, the USSR invaded Finland because the Finns didn't give the Russians the land they wanted.
The USSR probably wanted to conquer Finland in its entirety,
seeing how it had three times as many soldiers, thirty times as many planes, and 100 times more tanks than Finland.
Even with such staggeringly uneven odds, the Finns were able to hold off the Soviets for months.
This was largely due to mediocre leadership in the Red Army,
following Stalin killing or imprisoning all the good generals.

The war started when the village of Manila was shelled, probably by the Russians.
The Finns fought using skis and Molotov cocktails, which proved effective against Soviet tanks.
A Finnish marksman named Simo H\"ayh\"a, or the ``White Death'', killed at least 505 men,
the highest number of sniper kills in any major war.
The Russians' advance to Oulo was stopped by the Finns at the battle of Suomussalmi.

Under the leadership of Semyon Timoshenko and Kirill Meretskov,
the Russians were eventually able to push through the Mannerheim Line,
a Finnish defense system constructed between Taipale and the Gulf of Finland.
They entered the Isthmus of Karelia, and the war was ended by the Moscow Peace Treaty.

In June 1940, the USSR annexed the Baltic states of Estonia, Latvia, and Lithuania,
as well as Bessarabia, Northern Bukovina, and Hertza.

\subsection*{Interlude: Britain and France}

\textbf{Winston Churchill} had been First Lord of the Admiralty in World War~I,
and when this new World War broke out, he was reappointed to the post, serving in Chamberlain's War Cabinet.
During the ``Phony War'' (while the USSR was fighting Finland), Churchill was quite prominent.
It quickly became apparent that nobody had any faith in Chamberlain's handling of the war,
so he resigned and recommended Churchill to George~VI as the new prime minister.

The French had set up a defensive position in eastern France that they called the \textbf{Maginot Line}.
The line contained a full underground rail network and air conditioned chambers and it was very heavily fortified.

\subsection*{Germany Moves West}

\subsubsection*{Invasion of France}

The Germans decided that the Maginot Line wasn't worth attacking, because it was quite difficult to breach.
So, they decided to go around it.
In a strategy known as the \textbf{Manstein Plan},
the Germans attacked France via Belgium, the Netherlands, and Luxembourg.
The German attack through the \textbf{Ardennes} proved far more effective than Allied leaders had thought it would be.
The French had prepared their Dyle Plan without anticipating this offensive route.
A large number of Allied forces were quickly encircled and beaten.
In June, Italy attacked France, declaring war on the Allies.
By June 14, Paris was occupied, and France quickly surrendered to the German and Italians.
During the Battle of France, Churchill gave his ``We shall fight on the beaches speech'':
\begin{quoting}
  Even though large tracts of Europe and many old and famous States have fallen
  or may fall into the grip of the Gestapo and all the odious apparatus of Nazi rule,
  \textbf{%
    we shall not flag or fail.
    We shall go on to the end.
    We shall fight in France, we shall fight on the seas and oceans,
    we shall fight with growing confidence and growing strength in the air,
    we shall defend our island, whatever the cost may be.
    We shall fight on the beaches,
    we shall fight on the landing grounds,
    we shall fight in the fields and in the streets,
    we shall fight in the hills; we shall never surrender,
  }
  and if, which I do not for a moment believe,
  this island or a large part of it were subjugated and starving,
  then our Empire beyond the seas, armed and guarded by the British Fleet, would carry on the struggle,
  until, in God's good time, the New World, with all its power and might,
  steps forth to the rescue and the liberation of the old.
\end{quoting}

France was occupied by Axis forces, and the authoritarian \textbf{Vichy France} was established,
ruled by \textbf{Philippe Petain} and \textbf{Pierre Laval}.
The Vichy 80 were a notable group of parliamentarians who voted against the establishment of the new state.
Vichy France supplanted the Third Republic, governing under the motto of ``Work, Family, Fatherland''.
The Vichy regime colluded with the Nazis, although it did officially remain neutral.
In fact, during the Vel d'Hiv roundup, many Jews were sent to the \textbf{Drancy Internment Camp},
just northeast of Paris.

The French Resistance also formed under Charles de Gaulle, who had led the Free French.
De Gaulle gave the \textbf{Appeal of June 18}, a radio broadcast which originated the Resistance.

\subsubsection*{Battle of Britain}

After offering peace to Britain and getting rejected again, Hitler launched an air campaign against the UK\@.
Churchill commented on the coming battle in his ``finest hour speech'':
\begin{quoting}
  What General Weygand has called the Battle of France is over;
  the Battle of Britain is about to begin.
  Upon this battle depends the survival of Christian civilization.
  Upon it depends our own British life, and the long continuity of our institutions and our Empire.
  The whole fury and might of the enemy must very soon be turned on us.
  Hitler knows that he will have to break us in this island or lose the war.
  If we can stand up to him,
  all Europe may be freed and the life of the world may move forward into broad, sunlit uplands.
  But if we fail, then the whole world, including the United States,
  including all that we have known and cared for, will sink into the abyss of a new dark age made more sinister,
  and perhaps more protracted, by the lights of perverted science.
  Let us therefore brace ourselves to our duties, and so bear ourselves,
  that if the British Empire and its Commonwealth last for a thousand years, men will still say:
  \textbf{This was their finest hour.}
\end{quoting}

Hitler's Luftwaffe began battling the Royal Air Force for superiority in the skies over the island.
The British were greatly aided by the new radar systems that they had developed,
and the fact that Goering was so focused on bombing London gave the British a strategic edge.
The Germans ramped up their attack on Adlertag, or ``Eagle Day''.
Churchill ended up firing the commander who was in charge of the battle, Hugh Dowding.
The German end goal was to launch \textbf{Operation Sea Lion},
which involved taking the English Channel and launching a full scale invasion of Britain.
However, they weren't successful enough during the Battle of Britain to carry it out,
and Operation Sea Lion was indefinitely postponed.
Churchill had this to say in the aftermath of the Battle of Britain:
\begin{quoting}
  The gratitude of every home in our Island, in our Empire, and indeed throughout the world,
  except in the abodes of the guilty, goes out to the British airmen who,
  undaunted by odds, unwearied in their constant challenge and mortal danger,
  are turning the tide of the World War by their prowess and by their devotion.
  \textbf{Never in the field of human conflict was so much owed by so many to so few.}
\end{quoting}

The loss of the Battle of Britain prompted Germany to begin a large scale bombing offensive against British cities
known as \textbf{The Blitz}.
London would be attacked 71 times during the 37 week period.

\subsubsection*{Naval Warfare}

The pride of the German navy going into World War~II was the battleship \textbf{Bismarck}.
Along with her sister ship \textit{Tirpitz}, \textit{Bismarck} was the biggest ship ever built by Germany.
Commanded by Ernst Lindemann, \textit{Bismarck} only conducted one offensive operation.
The ship was intercepted by the Royal Navy, and they fought at the \textbf{Battle of the Denmark Strait}.

\textit{Bismarck} sunk the battlecruiser HMS \textit{Hood},
prompting the Royal Navy to chase the Germans relentlessly.
Eventually, she would be hit with torpedoes from HMS \textit{Ark Royal}
and sunk with help from other members of the British fleet.

\subsection*{Operation Barbarossa}

At this point, Hitler thought that Britain was still holding out hope
that the Allies would be aided by the US and the Soviet Union.
So, Hitler tried to get the Soviets to join the Axis Powers.
When the Soviets asked for some territorial concessions, Hitler decided to prepare for war.

On June 22, 1941, the Axis, led primarily by the Germans, invaded the USSR in \textbf{Operation Barbarossa},
named after Frederick Barbarossa (recall the emperor during the Third Crusade).
The offensive was in direct opposition to the agreement
that the countries had agreed to in the Molotov-Ribbentropp Pact earlier.
The Axis's main goal was to create a line of demarcation from Arkhangelsk to Astrakhan,
crush communism, and take the Soviet Union.

The Axis proceeded to launch the \textbf{Siege of Leningrad}.
The Nazis set up the siege in September 1941, but the battle would not end until January 1944.
The city was low on supplies, but a supply route called the Road of Life over the frozen Lake Ladoga
prevented supplies from running out.
Troops stationed in and around the city were commanded by Carl Mannerheim,
but he elected not to have his men directly fight in the siege.
Immediately following the battle, Finnish forces took back Karelia from the Soviets.

By October, the Axis seemed to be doing pretty well for itself.
The Kiev offensive was extremely successful, allowing the Axis to advance into the Crimea and eastern Ukraine.
The German pincer offensive known as Operation Typhoon was about to go into effect, pinning Moscow,
and allowing the Germans to take the city.
The only sieges still in effect were at Leningrad and at Sevastopol.

As the offensive against Moscow was put into action, the harsh Russian weather began.
The Germans were forced to stop just outside the city simply because they were really tired.
Even though the Nazis had taken quite a bit of territory,
their goals remained unmet, and the momentum of their \textit{blitzkrieg} had run out.

The Russians, notoriously slow at mobilizing troops, finally put together enough men to match the Axis's numbers.
By December, they began a huge counteroffensive to get the Nazis off their land.

\section{American Policy}

\subsection*{``Neutrality''}

Going into World War~II, the US was under the Neutrality Acts of 1936, supporting complete isolationism:
Americans understandably didn't want anything to do with the growing unrest in Europe.
However, the Neutrality Acts were woefully insufficient,
because they didn't really distinguish between aggressors and victims,
opting instead to treat both as ``belligerents'' in a war America wanted no part in.

When war broke out in Europe, FDR requested that Congress switch to a policy of \textbf{cash and carry}.
The policy allowed for the selling of materiel to belligerents,
provided that the buyers transported the goods themselves and paid in cash.

By March 11, 1941, the US enacted a new \textbf{Lend-Lease} policy for selling arms to the Allies.
Aid was free to the Free French, Great Britain, China, and the USSR\@.
In exchange, the US was given some leases on bases in Allied territory.
Roosevelt justified the policy by likening it to a garden hose:
\begin{quoting}
  Well, let me give you an illustration:
  Suppose my neighbor's home catches fire, and I have a length of garden hose four or five hundred feet away.
  If he can take my garden hose and connect it up with his hydrant, I may help him to put out his fire.
  Now, what do I do?
  I don't say to him before that operation,
  ``Neighbor, my garden hose cost me \$15; you have to pay me \$15 for it.''
  What is the transaction that goes on?
  I don't want \$15, I want my garden hose back after the fire is over.
  All right.
  If it goes through the fire all right, intact, without any damage to it,
  he gives it back to me and thanks me very much for the use of it.
  But suppose it gets smashed up, holes in it, during the fire; we don't have to have too much formality about it,
  but I say to him, ``I was glad to lend you that hose; I see I can't use it any more, it's all smashed up.''
  He says, ``How many feet of it were there?''
  I tell him, ``There were 150 feet of it.''
  He says, ``All right, I will replace it.''
  Now, if I get a nice garden hose back, I am in pretty good shape.
\end{quoting}
Eventually, the bill was passed when Everett Dirksen introduced it while a bunch of Congressmen were at a luncheon.
The office created to administer the Lend-Lease Act was headed by Edward Stettinius,
and it ended up improving on a previous Destroyers for Bases act.

In August 1941, Churchill met Roosevelt in Placienta Bay, Newfoundland,
aboard USS \textit{Augusta} and HMS \textit{Prince of Wales}.
They signed the \textbf{American Charter}, defining Allied goals for the war.
It stated that there would be no ``territorial aggrandizement''
and emphasized ``the right of all peoples to choose the form of government under which they will live''.

\subsection*{Pearl Harbor}

By 1939, the US had stopped trade with Japan and placed more economic pressure on Japan.
Japan's attacks on China stalemated in 1940.
The \textbf{McCollum memo} was a memo that outlined a course of action against Japan in the Pacific,
suggesting that the US provoke Japan into an ``overt act of war''.
Roosevelt put troops in the Philippines, saying that the US would react if anyone in that area was attacked by Japan.
This put a damper on Japan's aim to take a defensive perimeter around their country.
They didn't want Americans interfering with Japan's plan to take the Dutch East Indies.
As a result, Yamamoto made the decision to directly attack the US on home soil.

On December 2, 1941, the Japanese navy received a message to ``climb Mount Niitaka'',
giving information on how and when to attack the Americans (the Japanese referred to it as Operation Z).
Japan then sent the ``Fourteen Part Message'' to the US, breaking off negotiations with the Americans.
On December 7, Japanese aircraft attacked the naval base at \textbf{Pearl Harbor} on Oahu, Hawaii.
The first person to see the fighters coming was Lieutenant Kermit Taylor, who was manning a radar at Fort Shafter.
The attacks targeted Wheeler Field, and destroyed battleships
\textit{Arizona},
\textit{Oklahoma},
\textit{West Virginia},
and \textit{California}.
The Japanese fleets used the code words ``Tora, Tora, Tora'' to signal a successful surprise attack.

The next day, Roosevelt went in front of a Joint Session of Congress to ask for a declaration of war.
He gave his famous Infamy Speech:
\begin{quoting}
  Yesterday, December 7, 1941 --- \textbf{a date which will live in infamy} ---
  the United States of America was suddenly and deliberately attacked by naval and air forces of the Empire of Japan.
\end{quoting}
The only congressperson to vote against a declaration of war was Jeanette Rankin.
As a result of alliances, Britain, China, and Australia also formally declared war on Japan,
and Germany and the other Axis powers declared war on the US\@.

\section{European Theater}

\subsection*{War in Africa}
% Operation Torch

Fighting in North Africa started when Italy declared war.
The British army crossed into Libya, taking the Italian Fort Capuzzo.
They then launched Operation Compass, which crushed the Italians in Africa,
and Germany sent \textbf{Erwin Rommel}, the Desert Fox, field marshal of the Afrika Korps, to Africa.

The British fought Rommel all around Libya and Egypt, fighting at places such as Tobruk,
until eventually they met at the \textbf{Second Battle of El Alamein}.
Just before the battle, General Claude Auchinleck (the Auk), had been replaced by \textbf{Bernard Montgomery}.
Auchinleck had defended Ruweisat Ridge at the First Battle of El Alamein.
Montgomery planned Operation Lightfoot, a means to cut accessible corridors through the Axis minefields.
British tanks attacked Axis gasoline supplies at Tel el Aqqaqir.
Rommel was forced to get past land mines and barbed wire that went to the Qattara Depression.
The decisive defeat of the Germans at El Alamein led Churchill to say:
\begin{quoting}
  The fight between the British and the Germans was intense and fierce in the extreme.
  It was a deadly grapple.
  The Germans have been outmatched and outfought with the very kind of weapons
  with which they had beaten down so many small peoples, and also large unprepared peoples.
  They have been beaten by the very technical apparatus
  on which they counted to gain them the domination of the world.
  Especially is this true of the air and of the tanks and of the artillery,
  which has come back into its own on the battlefield.
  The Germans have received back again that measure of fire and steel which they have so often meted out to others.
  \textbf{%
    Now this is not the end.
    It is not even the beginning of the end.
    But it is, perhaps, the end of the beginning.
  }
\end{quoting}

The US entered the war during \textbf{Operation Torch} in late November, 1942.
\textbf{Dwight Eisenhower} commanded the troops in North Africa,
and \textbf{George Patton} came in from Casablanca.
They were defeated by Rommel at the \textbf{Battle of the Kasserine Pass} early in 1943,
but by May, the Allies broke the Mareth Line and shattered the Axis defense.

\subsection*{Attack on Italy}

In January 1943, Roosevelt and Churchill met at the \textbf{Casablanca Conference}.
It was agreed that the Allied forces in the south would turn their attention to Sicily,
which Churchill called the ``soft underbelly'' of Europe.
The Allies also decided to begin nonstop bombing of Germany from here on out (Operation Pointblank),
and that they would accept no less than the ``unconditional surrender'' of the Axis powers.
Roosevelt explained:
\begin{quoting}
  We mean no harm to the common people of the Axis nations,
  but we do mean to impose punishment and retribution upon their guilty, barbaric leaders.
\end{quoting}

In July 1943, \textbf{Operation Husky} was launched,
following a disinformation campaign against the Italians known as Operation Mincemeat.
The amphibious invasion landed between Licata and Scoglitti,
and the Allies began pushing forward into the continent.

As soon as the Allies took Sicily, Italian public sentiment immediately turned against Mussolini.
Victor Emmanuel~III called Mussolini into his office and told him he was fired.
The king replaced him with Marshal \textbf{Pietro Badoglio}.
Germany came into Italy, taking Rome and forcing Badoglio and the king to run away.
By October 1943, Italy declared war on Nazi Germany from Malta.

\subsection*{Bombing Campaign}

Starting in 1942, RAF Bomber Command were helped by the US Air Force in raiding Germany.
The general strategy was one of ``Europe First'';
the US would help take out Hitler and Germany before focusing on Japan.
The meeting in Casablanca then issued the Casablanca directive:
\begin{quoting}
Your primary object will be the progressive destruction and dislocation
of the German military, industrial and economic system
and the undermining of the morale of the German people
to a point where their capacity for armed resistance is fatally weakened.
\end{quoting}

\subsection*{Invasion of France}

\subsubsection*{Operation Overlord}

The Nazis knew that the Allies were going to try to get into Europe via France.
Rommel was sent to reinforce the Atlantic Wall, the huge system of defenses that the Axis had put in place.
The Germans also set up \textit{Rommelspargel} (Rommel's asparagus);
they placed 15-foot tall logs in the ground to damage gliders and paratroopers.

The Allies set up a plan called \textbf{Operation Overlord} to take the continent through Normandy.
To mislead the Germans as to their true intentions, the Allies created Operation Fortitude,
a military deception campaign divided into North and South sub-plans.
They made up phantom armies that would invade from Norway and Pas de Calais,
thus distracting the Axis with fake attacks that would never actually happen.
Another military deception in preparation for the invasion was called Operation Bodyguard.

Eisenhower's plan to invade at Normandy required special consideration because of the condition of the beachhead.
Artificial ports called Mulberry Harbors and special tanks called Hobart's Follies were developed.
On \textbf{D-Day}, June 6, 1944, an airborne assault heralded the arrival of a 5,000 vessel amphibious landing.
The Americans landed at Utah and Omaha Beaches,
the British at Sword and Gold Beaches,
and the Canadians at Juno Beach.
On August 15, the Allies launched \textbf{Operation Dragoon} on southern France.
By the end of August, more than 3 million Allied troops would be in France.

\subsubsection*{Operation Market Garden}

The next major Allied operation, on September 17, was \textbf{Operation Market Garden}.
Bernard Montgomery aimed to capture a number of bridges in the Netherlands.
His first attempt at a plan was Operation Comet, but that ended up scrapped because of bad weather.

To take the bridges, Montgomery decided to drop in paratroopers from the 82nd and 101st Airborne Divisions.
They were to take bridges at Eindhoven and at Nijmegen.
However, Market Garden failed because the Allies weren't able to take the bridge at Arnhem.
The British 1st Airborne was supposed to take the bridge, but failed because there was a Panzer unit guarding it.

\subsubsection*{Ardennes Counteroffensive}

In December 1944, the Germans launched an attack into the Ardennes, Operation Autumn Fog,
trying to take Antwerp.
They attacked at St.\ Vith, Elsenborn Ridge, Houffalize, and \textbf{Bastogne}.
During the opening of the offensive, the Blowtorch Brigade under Joachim Peiper committed the Malmedy Massacre,
killing 113 Allied POWs from the 285th Field Artillery Observation Battalion.

While the Germans moved west, Eisenhower ordered troops into Bastogne to set up a defense.
When Heinrich Luttwitz asked for the surrender of the defenders in the town,
General Anthony McAuliffe replied, ``Nuts!''
While troops held out in the town,
Patton took his army from Luxembourg and pushed through to end the siege at Bastogne.

On December 31, the Germans launched Operation Nordwind, their last major offensive on the Western Front,
into Alsace and Lorraine.
The battle was called many things.
The Germans called it Operation Watch on the Rhine,
the Allies called it the Ardennes Counteroffensive,
and press referred to it as the \textbf{Battle of the Bulge} because of the way the Allied front line was shaped.

The German offensive was doomed at the end of the battle,
which ended up being the largest and bloodiest battle the Americans fought in the Second World War.
Victory was achieved largely using surprise tactics and the fact that the Allies had broken the Enigma Machine,
thus giving them the ability to read German coded messages.

\subsubsection*{Taking Berlin}

In February 1945, FDR, Churchill, and Stalin met at the \textbf{Yalta Conference} in the Livadia Palace on the Crimea.
The idea was to discuss how to put together the countries that had fallen apart during the war.
The conference was codenamed Argonaut, and it had been preceded by the \textbf{Tehran Conference} in 1943.
Stalin agreed to help invade Japan,
and the Big Three called for a democratic government in Poland,
establishing the Curzon Line as the eastern border of the country.
They also agreed to divide Germany into four zones following victory over the Nazis.

After they won in the Ardennes, the Allies pushed the Germans back towards the Rhine.
They crossed after capturing the Ludendorff Bridge at Remagen.
They executed a pincer movement, trapping the Germans in the Ruhr Pocket.
By the time they got to Berlin, they met up with the Soviets and they forced a surrender.
\textbf{V-E Day} came on May 8, 1945.

Hitler had killed himself in a bunker on April 30,
and Mussolini was killed and strung up on display two days earlier.
In the US, Harry S. Truman had replaced FDR, who had died in office (more on this later),
and \textbf{Clement Atlee} had succeeded Churchill as Prime Minister in London.

Allied leaders met again at the \textbf{Potsdam Conference} July 11
to discuss and confirm the earlier agreements that they'd made about Germany.
They reaffirmed their goal of ``unconditional surrender'' with Japan,
issuing the Potsdam Declaration to that effect.

\subsection*{Eastern Front}

Remember that the Russians had been forced to retreat and used a scorched earth policy to destroy their own land.
When we left off, the Germans had laid siege to Leningrad, but succumbed to the harsh Russian winter.
The Soviets referred to the war as the ``Great Patriotic War''.

Having endured the winter, the Germans needed some oil.
So, they went south in the summer of 1942, focusing on the Caucasus.
The Soviets were quickly pushed back hundreds of miles to the east.
However, Hitler then split the Wehrmacht into Army Group A (Caucasus) and Army Group B (Stalingrad).
This major mistake would lead to a drawn out and bloody conflict in the city on the Volga.

The \textbf{Battle of Stalingrad} is perhaps the biggest, baddest, bloodiest battle in history.
The Germans under Friedrich Paulus started the attempt to capture the city in the late summer.
Fighting quickly became extremely close-quarters, and there was quite a bit of building-to-building combat.
Yakov Pavlov fortified an apartment building and used it as a fort for himself.
Mamayev Kurgan led fighting at the top of a hill now named in his honor.
Colonel Raiynin led the 1077th Anti-Aircraft Regiment, composed entirely of women just out of high school,
in an attempt to stop the Luftwaffe from succeeding in their runs.

In November, Alexandr Vasilevsky and \textbf{Georgy Zhukov} devised themselves a counteroffensive,
codenamed Operation Uranus.
They set up a two-pronged attack on Romanian and Hungarian forces protecting the German 6th Army.
Hitler ordered Paulus not to retreat,
although he did allow Erich von Manstein to try to fight their way into Stalingrad.
The encircled Germans tried to get out, and fighting continued until February 1943.
At that point, they ran out of supplies and surrendered.
The five month battle was the turning point of the war on the Eastern Front.

The Nazis focused on a heavily fortified salient near the Battle of Kharkov,
and as a result, the Germans' retreat was forced at the \textbf{Battle of Kursk} in July.
Hitler's offensive, codenamed Operation Citadel,
led to the Battle of Prokhorovka, one of the biggest armored battles ever.
The Soviet T-34 was countered by the Germans using Tiger and Panther tanks.
Soviet victory at Kursk started the Wehrmacht on a retreat that would take them all the way back to Berlin.

\section{War in the Pacific}
% TODO Tokyo Rose

\subsection*{Island Hopping}

On March 30, 1942, \textbf{Admiral Chester Nimitz} was appointed Supreme Allied Commander in the Pacific.
In the Southwest Theater, \textbf{Douglas MacArthur} was put in charge.
Japanese forces were under the command of \textbf{Isoroku Yamamoto}, who would be succeeded by Mineichi Koga.
Let's discuss the most important battles of the war in the Pacific.

\subsubsection*{Coral Sea}

Nimitz met the Japanese fleet near the Great Barrier Reef at the \textbf{Battle of the Coral Sea}
in early May 1942.
The battle was fought over control of Port Moresby, and it notably was fought entirely between aircraft carriers.
There was no ship-to-ship combat; the whole battle was fought exclusively with planes.

Before the battle started, Japan invaded Tulagi,
attempting to establish dominance over the island for use as a base.
On the first day of the battle, the \textit{Kikuzuki} and \textit{Shoho} were destroyed,
and on the next day, the \textit{Shokaku} was forced to retreat.
USS \textit{Yorktown} escaped damage, but \textit{Lexington} was destroyed.

\subsubsection*{Midway}

A month after the Coral Sea, Nimitz met Yamamoto at the \textbf{Battle of Midway},
a little over a thousand miles northwest of Oahu.
The battle would be the first time Japan had lost a naval battle since the Battle of Shimonoseki Straits in 1863.
Nimitz was aided by a Task Force, led by Frank Fletcher and Raymond Spruance.

During the battle, the Yorktown was destroyed, but the Japanese fleet lost many carriers.
The attack on the \textit{Hiryu} forced Vice Admiral Yamaguchi to sink with his ship.
The SBD Dauntless Dive Bombers helped sink the \textit{Mikuma} during an action known as the Famous Four Minutes.

\subsubsection*{Guadalcanal Campaign}

The first major Allied offensive in the Pacific was the \textbf{Guadalcanal Campaign},
codenamed Operation Watchtower.
In August 1942, Marines landed on the islands of Guadalcanal, Tulagi, Tanambogo, and Florida in the Solomon Islands.
The main invasion included fighting along the Matanikau River,
and eventually Operation Ke resulted in full Japanese withdrawal, rendering the campaign a major Allied success.
Supply lines used during the battle included the Tokyo Express, and the Allies set up the ``Cactus Air Force''.

\subsubsection*{Leyte Gulf}

The largest naval battle of World War~II was fought in October 1944,
on the waters off of the Philippine islands of Leyte, Samar, and Luzon,
in the \textbf{Leyte Gulf}.
The battle saw the first use of the kamikaze attacks,
and the Third Fleet, commanded by \textbf{William Halsey}, destroyed Kurita's ``Center Force''.
Fighting center around Samar and the Surigao Strait,
and at one point, Halsey followed a decoy fleet, leaving the main landing force unprotected from the sea.

\subsubsection*{Iwo Jima}

In February 1945, American Marines executed Operation Detachment,
invading and capturing the Japanese-controlled island of \textbf{Iwo Jima}.
Iwo Jima was referred to as a ``God-forsaken island'',
filled with heavy Japanese fortifications and an extensive tunnel system under Mount Suribachi.
A Marine general, Holland ``Howling Mad'' Smith, was stuck on his ship for the duration of the battle.

The Japanese commander Tadamichi Kuribayashi opted out of a suicidal banzai charge on the beach,
instead choosing to fight in the tunnel system that he'd created,
modeling the defense on the prior Battle of Peleliu.
Much of the fighting took place on Meatgrinder Hill,
until the Marines took Mount Suribachi.
\textbf{Joe Rosenthal} took a picture of the iconic flag raising on the mountain,
making people like Ira Hayes and John Bradley famous.
Kuribayashi's body was never found.

\subsubsection*{Okinawa}

The largest amphibious assault of the Pacific theater, Operation Iceberg,
was launched in April 1945 on the islands of \textbf{Okinawa}.
The intended outcome of the invasion was a base from which to launch an invasion of mainland Japan.

The ensuing battle has been called the ``typhoon of steel'',
due to the intense and ferocious fights and the massive numbers of \textit{kamikaze} attacks launched by Japan.
Japan lost over 77,000 soldiers, while the Allies lost 14,000.
Notable deaths include Ernie Pyle, and the Japanese commanders Isamu Cho and Mitsuru Ishajima.
Fighting occurred at Wana Ridge, the Kiyan Peninsula, and Shuri Castle.

\subsection*{Ending the War}

\subsubsection*{Manhattan Project}

In 1942, General Leslie Groves directed an initiative to create the first nuclear weapons.
The project, codenamed the \textbf{Manhattan Project}, took place at \textbf{Los Alamos National Laboratory},
under the direction of physicist \textbf{J. Robert Oppenheimer}.
Along the way, the Manhattan Project absorbed the British project Tube Alloys.
The Smyth Report chronicled the history of the project, which was located at places like Hanford and Oak Ridge.

On July 16, 1945, the \textbf{Trinity Test} was conducted at Alamogordo Bombing and Gunnery Range in New Mexico.
The successful test led to the creation of two weapons: Little Boy and Fat Man.

\subsubsection*{Hiroshima \& Nagasaki}

Recall that at this point, the US army was all set up on Okinawa ready to invade by land.
Then, Truman ordered the use of the bombs that the Manhattan Project had developed.
The B-29 \textit{Enola Gay} was set up to deliver the bombs from Tinian in the Mariana Islands.

The plane dropped Little Boy on Hiroshima on August 6, 1945.
Three days later, Fat Man was dropped on the city of Nagasaki.
During the following months, many thousands would die of radiation sickness, burns, etc.

\subsubsection*{A Second Victory}

On August 15, Japan surrendered to the Allies.
Hirohito read the \textbf{Jewel Voice Broadcast},
in which he announced that Japan had accepted the Potsdam Declaration and unconditionally surrendered.
It was probably the first time that the emperor had spoken to the common people,
and Hirohito told people to ``endure the unendurable'', and he renounced his divinity.
After the war, Hirohito avoided the war crimes trials that would condemn Tojo to death.
Tojo tried to kill himself, and ended up hanged in 1946.


