\chapter{Asian Dynasties}

\epigraph{%
  When future generations look back to my time, it will probably be similar to how I now think of the past.
}{Wang Xizhi}

\section{Ancient India}

Recall that there was some burgeoning civilization in India.
Around the 4th century BC, more centralized dynasties with influential impact started taking power.
Here we go over some of the more important ones.

\subsection*{Maurya Dynasty (324 -- 184 BC)}

The dynasty was founded by \textbf{Chandragupta Maurya}.
He defeated the Nanda Empire and the Macedonian Seleucid Empire,
and having defeated Alexander the Great, he was able to unify India.
The Iron Pillar of Delhi is sometimes attributed to him, but that theory was probably discredited.

Chandragupta's grandson, Bindasura's son,
\textbf{Ashoka} (273--232 BC), is considered perhaps the greatest ruler of ancient India.
He took the throne after killing some of his relatives (but spared his brother, Tissa).
He conquered the Kalinga empire, an action that changed him forever,
since he adopted Buddhism as his new religion.
After his conversion, he put up edicts on columns all over the place to promote non-violence and animal rights,
and he became mostly a pacifist ruler.
One of those columns at Sarnath has four lions on it,
along with a 24 spoke wheel of righteousness, a symbol which would be placed on the Indian flag.

\subsection*{Gupta Dynasty (240 -- 550)}

\textbf{Chandra Gupta II} was the most important ruler during this time,
and the Gupta Empire was biggest under him.
Besides that, not a lot of hugely notable things happened around this time.
There would follow periods of India being invaded repeatedly by people from
all over the place, including Persia, Scythia, Parthia, and more.

Until we get to a point in history where India starts to interact with the rest of the world more,
let's move up a little bit and take a look at China.

\section{Chinese Dynasties}

The history of China until modern times is largely one of dynasties.
Here are the most important ones.

\subsection*{Shang Dynasty (1600 -- 1046 BC)}

The Shang is the first real Chinese dynasty with written records, succeeding the lesser known Xia dynasty.
The writings are found on ``oracle bones'',
which were bones that were heated and used to tell the future in a practice called pyromancy.
Their capital was near Anyang.
The emperor Wu Yi was killed by a bolt of lightning,
which is funny because he was quite vocal about not liking the god of lightning and thunder very much.
Other notable emperors include Wu Ding and Fu Hao.
Then, the emperor Xi Din lost the battle of Muye and killed himself, leading to the Zhou dynasty ruling.

\subsection*{Zhou Dynasty (1046 -- 256 BC)}

The Zhou was the longest lasting period of Chinese history.
During the Zhou, the use of iron began.
When the Zhou capital was sacked, the Eastern Zhou and the \textbf{Spring and Autumn period} began.
Intellectualism flourished at this point;
Sun Tzu wrote \textit{The Art of War}
and the Hundred Schools of thought (one of which was \textbf{Confucianism}) were introduced.

However, during the Zhou dynasty, real power rested with the feudal lords, so
by the end of this really long dynasty,
everything fell apart into seven separate warring states.
When the fighting had settled, the state of Qin was victorious.

\subsection*{Qin Dynasty (221 -- 206 BC)}

A short, but important dynasty, consisting almost entirely of one important emperor: \textbf{Qin Shi Huangdi}.
He burned a lot of books, so people tend to not think too highly of him.
He also standardized weights, unified Chinese writing,
and begin building a Great Wall which you may have heard of.
Along with his minister Li Si, he implemented \textbf{Legalism}.
When he died, he was buried near Xi'an with a massive and famous terra cotta army.

Qin Shi Huangdi had created the first truly unified China,
but his death, the dynasty was mismanaged into failure, and it collapsed quickly.

\subsection*{Han Dynasty (206 BC -- AD 220)}

A golden age of China that had a lasting impact, the Han dynasty names China's predominant ethnic group today.
The emperor \textbf{Gaozu} (aka Liu Bang) founded the dynasty, even though he was born as a peasant.
He reunited China and made a new capital at Chang'an (now Xi'an).

One of the greatest rulers of China was \textbf{Wudi}, who expanded China greatly.
He reformed government and he made Confucianism the new official doctrine.
An official named \textbf{Wang Mang} temporarily overthrew the Han dynasty
to try and make a new Xin dynasty, but it didn't take.
Wang Mang died at Chang'an, and Emperor Guangwu took back the throne.

By the end, the Han dynasty had faced the \textbf{Yellow Turban Rebellion},
a Taoist peasant revolt during the reign of Emperor Ling.
The Yellow Turban Rebellion was the opening to the Period of the Three Kingdoms.

\subsection*{Period of the Three Kingdoms (184 -- 280)}

A short and violent period in history.
The classic novel \textit{Romance of the Three Kingdoms} ensured that it's still well known today.
The three kingdoms were Wei, Shu, and Wu:
Cao Wei north of the Yangtze,
Eastern Wu in the lower Yangtze,
and Shu Han in Sichuan.

The \textbf{Battle of Red Cliffs} (208) was fought between Liu Bei and Sun Quan,
(although admittedly it overlapped a bit with the Han dynasty).
It set up the basis for the states of Shu and Wu.

Eventually, Cao Wei managed to defeat the other kingdoms, but failed to reunify China.
China ended up going through a short period of Southern and Northern dynasties,
but not a lot of things I care about happened during that time, so let's skip to the Tang dynasty.

\subsection*{Tang Dynasty (618 -- 907)}

The Tang dynasty succeeded in reuniting China after the Sui dynasty (very short dynasty) fell.
It brought another golden age in China:
important people such as \textbf{Li Po} and Du Fu worked and the printing press was invented.

The Tang dynasty moved the capital back to Chang'an and was also ruled by \textbf{Gaozu}.
His son, emperor Taizong (aka Li Shimin), deposed him and became a great ruler,
consolidating power fairly effectively and conquering a lot more of western China.
He was succeeded by Empress \textbf{Wu}, the only woman to be emperor of China,
presiding over the ``second Zhou dynasty''.
The Tang also fought three wars against the Three Kingdoms of Korea (Gorguyeo, Beakje, and Silla).

Later, during the reign of Xuanzong, the \textbf{An Lushan} rebellion tore apart the Tang.
Even though it was quelled, the revolt had lasting effects that ended up beginning the
periods of the five dynasties and the Ten Kingdoms.

\subsection*{Song Dynasty (960 -- 1279)}

Soon afterwards, the Song dynasty began.
This was a time dedicated to culture, not warfare; to engineering, not killing.
Gunpowder was discovered, the first compass was made,
a standing navy was established, and paper money was circulated.

The first ruler was \textbf{Taizu}, and he realized that he could save his own neck if he
asked all the generals and military people around China to retire.
This resulted in scholars being far more dominant during the Song dynasty.

The capital was moved to Kaifeng, which was captured by the Jin dynasty from the north.
So, the Song fled down south and made a new Southern Song dynasty with capital at Hangzhou,
and the Jin didn't bother them any more.
But then the Mongols came down and destroyed the Jin, with the help of the Song (more on that later).
The Song didn't realize that the Mongols wanted their territory too,
so even though they lasted a while under Mongol attacks, they eventually were defeated.

\subsection*{Yuan Dynasty (1271 -- 1368)}

The Mongols established the Yuan dynasty, most notable for the emperor \textbf{Kublai Khan}.
The dynasty was created after the Battle of Yamen, and the capital was at Dadu.
We'll talk more about the Mongols and the Khans later in the book.

In the 1350's, the White Lotus Society (Buddhists) created an army to go against the Yuan dynasty.
This \textbf{Red Turban Rebellion}, led by Zhu Yuanzhang,
would bring about the demise of the Yuan and the rise of the Ming within a few years.

\subsection*{Ming Dynasty (1368 -- 1644)}

The rulers of the Ming dynasty were the Zhu family.
Everyone knows about their porcelain work, notably their vases.
This dynasty originated the use of the word ``china'' to describe high quality porcelain.

The emperor \textbf{Hongwu}, who had led the Red Turbans, founded the dynasty.
The eunuch Zheng He led fleets on treasure voyages to show off how rich they were.

This is also when China's capital was moved to Beijing, and Yongle built the \textbf{Forbidden City}.
The Ming dynasty collapsed as a result of a failing economy
coupled with the invasion of the Manchu people from the north.

\subsection*{Qing Dynasty (1644 -- 1911)}

The Manchurian Qing dynasty was the last dynasty of China.
They created the banner system, and the emperor Kangxi quelled the Revolt of the Three Feudatories.
The dowager empress Cixi weakened the dynasty, and at one point,
they were also threatened by a \textbf{White Lotus Rebellion} against Emperor Chia Ch'ing;
the rebellion was almost successful because of a corrupt government under Ho-shen.
The three great emperors were Kangxi, Yongzheng, and Qianlong.
Other notable emperors include Guangxu.
However, the dynasty ended because the later emperors weren't very effective.

The \textbf{Taiping Rebellion} lasted from 1850 to 1864.
The Christian Hong Xiuquan led a fairly bloody revolt against the Qing,
aided by his God Worshiper's Society.
Hong Xiuquan said that he was Jesus's younger brother
(the genealogy might not side with him on that particular issue).
The rebels set up a base at Nanking.
The Qing were aided by foreign powers,
which helped create the \textbf{Ever Victorious Army},
under the command of Frederick Townsend Ward, and later Charles George ``Chinese'' Gordon.
The rebellion ended with the fall of Nanking and a Qing victory.

In 1899, foreign spheres of influence were growing in China.
Locals such as the Society of Righteous and Harmonious Fists (``Boxers'') weren't particularly thrilled.
They decided to fight back against the Unequal Treaties and the Open Door policy
that other countries had created in China, starting the \textbf{Boxer Rebellion}.
During the rebellion, Boxers killed foreign missionaries during the Taiyuan Massacre.
Boxers converged on Beijing and laid siege to the Legation Quarter,
and the Empress Dowager Cixi decided to support them.
Thus began a conflict between Cixi and the Boxers and the Eight-Nation Alliance against them.
The siege of the legations was lifted in 1900 and the Boxer Protocol was implemented.
