\chapter{Towards a Modern Era}

After the Middle Ages, a rebirth of culture and achievement began.
It started in Italy, and it would provide a transition from Medieval Times into the Early Modern Era.

\section{Italian Renaissance}

The \textbf{Renaissance} began in Tuscany, on the west coast of Italy.
It would spread to Venice, and the \textbf{humanist} scholars were happy with the new discoveries,
and they enjoyed interest in the classics that had been forgotten in what they called the Dark Ages.

The Italian Renaissance would end up peaking in the 16th century, when the Italian Wars threw the region into turmoil.

We'll leave the numerous cultural achievements of the Renaissance to literature experts and art historians.
Suffice it to say that there was a lot of very good art and literature that came from the era,
including Petrarch, Bocaccio, Michelangelo, Leonardo da Vinci, Botticelli, Raphael, and Titian.

\subsection*{Background}

Remember that the papacy had been moved to Avignon by Philip the Fair,
so the papal states were fairly loosely held together, and didn't do a lot of huge things.
However, the Fourth Crusade had done much to improve the strength of trade of the northern Italian city-states,
like Genoa, Venice, and Pisa, by weakening the Byzantine Empire.

Most of the outside kingdoms left the city-states of Italy alone,
and Florence grew in strength, resulting in their florin becoming the \textit{de facto} currency of trade.
Merchants were very powerful people in the 13th century.

But, in the 14th century, the economy fell apart, party as a result of the Hundred Years' War.
The banks of Bardi and Peruzzi in Florence collapsed.
The \textbf{Black Death} swept through Europe and took out nearly a third of Europe's people.
Florentine textile workers rebelled in the \textit{ciompi} in 1378.

\subsection*{Warring City-States}

Northern Italy was divided between city-states,
and they were embattled between the forces of the papacy and the Holy Roman Empire.
Internally, they were also divided between the factions of the Guelphs and the Ghibellines.
Mercenaries were employed by many city-states, and many would have a sizable military force.
On land, the mercenaries were called \textit{\textbf{condotierri}}, and they came from all over Europe.

Pisa, Genoa, and Venice fought many battles on water, and eventually Pisa's became weakened.
On land, Florence, Milan, and Venice became dominant,
and they signed the Peace of Lodi (1454) and agreed to stop fighting so much.

\subsection*{Medici Florence}

For much of the 14th century, the House of Albizzi led Florence.
Florence became the central banking hub of Europe when Siena's Bonsignori banks failed.

The main family that rivaled the Albizzi was the \textbf{Medici}.
The first of the Medici was \textbf{Giovanni de' Medici},
followed by his son, \textbf{Cosimo di Giovanni de' Medici} (Cosimo the Elder).
They controlled the largest bank in Europe, and they had a lot of money.
They would stay in power for three centuries.
Cosimo negotiated the peace of Lodi with \textbf{Francesco Sforza}, leader of Milan.
Cosimo's son Piero succeeded him, and promptly died.

And so power passed to Cosimo's grandson, \textbf{Lorenzo the Magnificent}.
Lorenzo would become a big patron of the arts, a very important thing in Renaissance Italy.
He formed the Council of Seventy, which formalized rule of Florence, with Lorenzo himself at the head.
However, his relationship with others slowly decayed, and the papacy didn't like him very much.
At the Castello del Trebbio, the Pazzi family was nudged by Pope Sixtus IV to conspire against Lorenzo.
On April 26, 1478, they tried to kill Lorenzo after High Mass at the Santa Maria del Fiore
(the Florence Cathedral, the one with the big dome).
They managed to stab Giuliano, Lorenzo's brother, but Lorenzo escaped alive, but wounded.
Florentines took it upon themselves and killed off many members of the Pazzi family
(perhaps Ezio Auditore had something to do with it?)
and the Pazzis were forced out of Florence.

Near the end of the 15th century, \textbf{Girolamo Savonarola}, a Dominican monk, took Florence,
after the French, led by Charles~VIII, invaded the city.
He took power because people didn't like the secularism that had been flourishing,
and his followers were called the \textit{Piagnoni}, or the ``Weepers''.
He destroyed many artworks and books in the Piazza della Signoria in his \textbf{Bonfire of the Vanities},
notably making Botticelli ``desert his painting'' and Pico della Mirandolla abandon writing.
Savonarola wrote works such as
\textit{Infelix Ego},
\textit{On the Ruin of the Church},
and \textit{On the Downfall of the World}.
He was asked to walk through fire, but a rainstorm canceled the event, causing a riot.
Eventually, Pope Alexander~VI excommunicated him and he was burned at the stake,
having confessed to heresy while being tortured with Domenico de Pescia and Fra Silvestro.

\subsection*{Religion}

In 1417, the papacy was able to return to Rome, but it mostly stayed a ruined city.
By 1447, though, Pope Nicholas V was able to make it a better place.
Pope Pius~II was a humanist scholar, and the papacy would become controlled by powerful families,
such as the Medici and the Borgias.
Pope Sixtus~IV ordered the construction of the Sistine Chapel.
The Papal States became more centralized, and the popes became warriors.

\textbf{Pope Alexander~VI} was pope for the last decade of the 15th century.
His uncle was Alfonso Borgia, or Pope Callixtus~III\@.
He was originally called Rodrigo Borgia, and he had children even though he was pope.
His important children were \textbf{Lucrezia Borgia} and \textbf{Cesare Borgia}.
Claims of incest surrounded the family, which wasn't exactly traditional.
Cesare, who conquered Forli, would be praised in Machiavelli's \textit{The Prince},
being used as an example of a great and harsh ruler.

\section{Tudor England}

\subsection*{Henry~VII}

Last time we were in England, \textbf{Henry~VII Tudor} had won the Battle of Bosworth Field,
and he had become the first Tudor king of England.
The Tudors would rule England for the entirety of the 16th century.
It's perhaps interesting to note that the actual Tudor monarchs didn't like being called ``Tudor'',
because the family had been a very modest one before Henry~VII\@.

Henry~VII married Elizabeth of York, and unified the warring houses of Lancaster and York,
as symbolized by the Tudor rose, a white and red rose.
He planned to make a gold plated statue of himself and put it on the tomb of Edward the Confessor.
A plot against him by John de la Pole (Richard~III's nephew) involved a boy named Lambert Simnel,
who posed as Edward, Earl of Warwick,
and led some mercenaries into England, but they were defeated at the Battle of Stoke.
Another threat came from Perkin Warbeck, who pretended to be Edward~IV's son.
Henry married his son Arthur to \textbf{Catherine of Aragon}, daughter of Ferdinand and Isabella,
but Arthur quickly died, leaving his brother Henry as the new heir.

\subsection*{Henry~VIII}

When Henry died in 1509, his son became \textbf{Henry~VIII}, and the position of the Tudors was secured.
He married Catherine of Aragon, but the only child of theirs that survived was Mary.
Henry started an incursion into France, but the only notable battle was the Battle of the Spurs,
where the king didn't even show up, and the war wasn't particularly effective.

At this point, James~IV of Scotland activated the Auld Alliance with France and declared war.
In 1513, the English met 10,000 Scots at the \textbf{Battle of Flodden Field}.
The battle actually took place at Milfield and Braxton Hill.
The English were victorious, and James~IV was defeated, as chronicled by Walter Scott in his \textit{Marmion}.

When Catherine wasn't able to have any more children than Mary, Henry got nervous.
The last time a female monarch was on the throne was when Matilda was queen,
and that hadn't worked out so well.
He decided to divorce Catherine, but Pope Clement~VII would not allow it.
So, Henry got mad and decided to make his own church.

The new \textbf{Church of England} was mostly just Catholic, except with Henry at the head of it.
In 1530, he declared his marriage to Catherine invalid, and Mary was declared illegitimate.
Henry then married \textbf{Anne Boleyn} in 1533, and Anne had a daughter, Elizabeth.
Henry got annoyed again, and when the queen failed to have a son, he beheaded her, and he married \textbf{Jane Seymour}.
Jane gave birth to Edward, and Henry was finally happy.
But, Jane promptly died of sepsis, and Henry was actually upset about this one.
Henry then married the German \textbf{Anne of Cleves} for the political advantages.
But, he didn't like her very much, so he divorced her as fast as he could.
He then married \textbf{Catherine Howard}, but since she wasn't faithful,
she ended up on the chopping block as well.
Henry's last marriage was to \textbf{Catherine Parr}, and his health fell into decline.
If you haven't been counting, that's six marriages total.

Henry did do things other than get married.
His dissolution of the monasteries resulted in Robert Aske leading a revolt known as the Pilgrimage of Grace.
His advisors included Cardinal Thomas Wolsey and \textbf{Thomas Cromwell},
who helped him with the dissolution of the monasteries.
He got into an argument with \textbf{Thomas More}, Lord Chancellor, who opposed Henry's creation of the new church.
True to form, Henry had him convicted of treason and beheaded, and Pope Pius~XI would canonize More as a martyr.

Henry died in 1547 as his paranoia and insanity got even worse.
He was buried next to Jane Seymour, and he was succeeded by their son, Edward.

\subsection*{Edward and Mary}

\section{Central Europe}

\section{Age of Discovery}

\section{Russia}

\section{India}

