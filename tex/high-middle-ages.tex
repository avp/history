\chapter{The High Middle Ages}

After the fall of the Western Roman Empire, the Middle Ages began.
They would last until the start of the Renaissance in the 15th century,
and they were characterized by massive population growth and war in Europe.
Let's look at some areas of Europe in particular.

\section{England}

\subsection*{Anglo-Saxon England}

Near the beginning of the Middle Ages, the Anglo-Saxons lived in England,
a result of Germanic tribes moving north across the English Channel.
Anglo-Saxon England was traditionally considered a heptarchy, consisting of:
Northumbria, Mercia, East Anglia, Essex, Kent, Sussex, and Wessex.
By the 7th century, much of the Anglo-Saxons had converted to Christianity,
and the Venerable Bede chronicled this era, earning him the title ``Father of English History''.

The kingdom of Mercia rose to prominence in the 7th century, building massive earthworks such as
Offa's Dyke, which separated modern-day Wales from the rest of England.
By the late 8th century, however, Viking raiders began to invade the island.
These Danes were led by Guthrum, whom we don't know much about before he seized control of the Danes.
They subdued East Anglia, capturing York.

\textbf{Alfred the Great}, king of Wessex, son of \AE{}thelwulf, fought back against Guthrum.
He created a network of fortified towns called burhs (these would be turned into boroughs eventually)
to defend Wessex and defeat the invaders, at battles such as Eddington and Ashdown.
Alfred confined Guthrum in a region that became known as the Danelaw, and proclaimed himself king of all the English.
His coronation was performed by Pope Leo IV, and
he compiled a census of England, called the Doom Book
(not to be confused with the Domesday Book that would be made later).
He remains the only English king to ever be called ``the Great''.

Alfred was eventually succeeded by Edgar the Peaceful, and by the end of Edgar's reign,
Wessex had taken control of the rest of England, including York and the Danelaw.
But, when Edgar died only two years after being crowned at Bath,
the issue of who should succeed to the throne was thrown into doubt.
His son Edward was crowned king, but he was quickly assassinated.

He was followed by Edgar's half-brother, \textbf{\AE{}thelred the Unready}.
As his epithet may suggest, his reign didn't go very well, even though he did rule for a long time.
It's now believed that he was simply the victim of bad counsel, and wasn't actually particularly ``unready''.
He had married Emma, a princess of Normandy.
While he was king, England was invaded by Danes again, led by Sweyn Forkbeard.
The Vikings pillaged through Essex, an action that resulted in the Battle of Maldon,
where the Vikings easily crushed the English.
From that point forward, the Vikings roamed the countryside, apparently free to take whatever they pleased,
while \AE{}thelred probably just hid in his basement.
Sweyn found \AE{}thelred and forced him into exile.
Soon after, though, Sweyn died, and \AE{}thelred was able to come back to England.

Angered at \AE{}thelred, \textbf{Cnut}, son of Sweyn, invaded England.
\AE{}thelred's son, Edmund Ironside, abandoned his father, and multiple English nobles took Cnut's side.
But, \AE{}thelred died, and Edmund was forced to fight the Danes.
At the battle of Ashingdon, the Danes overpowered the English, and Cnut and Edmund split England;
Edmund would take Wessex, and Cnut would take everything else.
When Edmund died (he was probably murdered), the English witan council declared Cnut king of England.
Cnut married \AE{}thelred's widow Emma,
and she agreed only because he agreed to allow her children to be heirs to the throne.
However, Cnut had had a previous wife, and her children also wanted the throne.

So, when Cnut died in 1035, the throne was disputed.
The claimants were Harald Harefoot, and Emma's son by Cnut, Harthacnut.
Emma's son by \AE{}thelred, Edward, tried to raid Southampton, but failed, and his brother Alfred was killed later.
Harthacnut became king, but he was unpopular, and nobody wanted his heirs to become king.
So, when he died, Edward came back from exile and was recognized as king.

Edward, who would come to be known as \textbf{Edward the Confessor},
married the daughter of Earl Godwin of Wessex, in order to secure an alliance.
But, when the citizens of Dover killed Edward's relative Eustace of Bolougne
(probably a response to Edward's appointment of Robert of Jumieges as Archbishop of Canterbury)
and Godwin refused to punish some people who killed Edward's relatives,
Edward had a falling out with Godwin, and the Godwins ran away to Normandy.
During the conflict, Edward was regularly working with and for the earls Siward and Leofric.
At this time, Edward probably offered the succession to his cousin, William, duke of Normandy.
However, soon after, the king reconciled with Godwin, and when Godwin died in 1053,
his son, \textbf{Harold~Godwinson}, acceded to the earldom in Wessex.
Godwin's other sons were given East Anglia, Mercia, and Northumbria.

In 1066, Edward the Confessor died, and Harold~Godwinson was declared king of England.
In the following year, England would be in turmoil because,
even though Harold had taken the crown, there were those who believed it wasn't his to take.
Remember that Edward had given a promise of succession to his cousin, William, duke of Normandy.
Also, Harald~Hardrada of Norway, descendant of Cnut, staked a claim to the throne.

Harold's brother Tostig, leader in Northumbria, allied himself with Hardrada.
Hardrada invaded Northern England, fighting the Battle of Fulford near York.
However, Hardrada was killed at Stamford Bridge by Harold~Godwinson's forces.

While Harold was celebrating his victory, William, Duke of Normandy, sailed for England.
Harold marched back south, where he met William at the \textbf{Battle of Hastings} on September 27, 1066.
William crushed Harold's forces, and Harold himself probably got an arrow through the eye.
The battle was chronicled on the \textbf{Bayeux Tapestry}, which also notably shows Halley's comet.
As a result of his achievement, William is now known to us as \textbf{William the Conqueror}.

\subsection*{Anglo-Norman England}

\section{Eastern Europe}

\section{Southern Europe}
