\chapter{Recent World History}

Keep in mind that we haven't looked at some parts of the world in quite a long time.
As a result, ``recent'' in some of these places may go back quite a while\ldots{}

\section{South America: Wars and Dictators}

\subsection*{Bolivar's Fight for Independence}

For this part, we go all the way back to the first part of the 19th century.
Spain and Portugal ruled South America, and in the first quarter of the 1800's, the wars of independence began.

In 1819, \textbf{Simon Bolivar} came up with a plan to take New Granada by crossing the Andes.
He'd previously given his Decree of War to the Death during his Admirable Campaign through M\'erida.
Bolivar aimed to create an army himself and invade Venezuela with the help of Antonio Jose de Sucre.
After winning at the Battle of Carabobo, the Battle of Pinchincha, and finally at the Battle of Boyaca,
Bolivar created \textbf{Gran Colombia} in northern South America and became its first president.
In 1828, an assassination attempt on Bolivar was stopped by his mistress Manuela Saenz,
and Bolivar called her ``Libertadora del libertador'' (liberator of the liberator).
At the \textbf{Guayaquil Conference}, Bolivar met with Jose de San Martin and convinced him to retire to France.

\subsection*{Brazil}

When King Joao~VI of Portugal left Brazil in 1821, he left behind his son,
who became \textbf{Emperor Pedro~I of Brazil}, aka ``Dom Pedro~I''.
He abdicated in 1831, leaving behind his young son as the new emperor.

When \textbf{Pedro~II} took power, he was only five, so regents ruled until 1840.
At this time, there were loads of revolutions and wars, including:
the Sabinada,
the Ragamuffin War or War of Tatters (recall Garibaldi?),
the Male Revolt,
Cabanagem and Balaida,
and others.
Then, Pedro started a new parliamentary reign.
He passed the Law of Free Birth, and his daughter Isabel passed the Golden Law which abolished slavery.
Pedro was ousted in 1889 by a coup led by Deodoro da Fonseca, and a republic was established in Brazil.

The Old Republic lasted from 1889 to 1930, but I'm not going to talk about it much.

After 1930, \textbf{Getulio Vargas} led a new military junta.
He was called the ``Father of the Poor'', and he set up an Estado Novo in Brazil,
inspired by the Estado Novo that Antonio Salazar had set up in Portugal.
He replaced the ``cafe com leite'' (coffee with milk) politics that had dominated the old republic,
which had placed a lot of power in the hands of the very rich and particularly alert coffee industry.
He created the fake communist ``Cohen Plan'' to foment hatred of communists,
and he supported Plinio Salgado's far-right political organization until Salgado targeted him in the Pajama Revolt.
Vargas was deposed in 1945 and replaced with Jose Linhares,
but he became a democratically elected president in 1951.
When his bodyguard tried to kill his rival Carlos Lacerda on Rua Tonelero, people called for Vargas's resignation.
He killed himself in 1954 and was replaced by his Vice President Caf\'e Filho.

\subsection*{Chile}

Chile's independence movement began in 1810, when a new junta was established.
The new government was led by Jose Miguel Carrera, who ruled with an iron fist.
Unhappy about this, \textbf{Bernardo O'Higgins} led a group of rivals that aimed to get full independence.
After a loss at Rancagua, he and his friend \textbf{Jose de San Martin}, the ``Bolivar of the South'',
led the Army of the Andes to victories at Maipu and \textbf{Chacabuco}.
O'Higgins served as the second Supreme Directory of Chile from 1817 to 1823.
Having contextualized Chilean independence in the early 1800's\ldots{}

In 1970, Marxist \textbf{Salvador Allende} was elected president.
He was part of the new Popular Unity Party, and he defeated Jorge Alessandri to win the election.
He dealt with the El Teniente copper mine strike,
and he went through the Cuban packages smuggling scandal during which arms were discovered in the packages.

In August 1973, Carlos Prats, a member of Allende's cabinet, was replaced by \textbf{Augusto Pinochet}.
A month later, on September 11, 1973, Pinochet staged a CIA-backed coup against Allende.
Allende apparently committed suicide in the presidential palace, La Moneda,
and Pinochet seized power in Chile.

Having come to power, Pinochet ordered the assassination of everyone he didn't like by the Caravan of Death.
He arrested Defense Minister Orlando Letelier, imprisoned him for a year, and tortured him;
Letelier was eventually assassinated on Pinochet's orders.
Pinochet killed Carlos Prats, utilized the DINA secret police, and he was supported by the Carabineers.
He also employed the \textbf{Chicago Boys}, a group of Chilean economists,
who had trained at the University of Chicago under Milton Friedman and Arnold Harberger,
and who lowered tariffs and privatized industry following Allende's Marxist regime.
Pinochet led \textbf{Operation Condor} in the 1970's, suppressing and killing leftists in South America.
The campaign was what targeted Orlando Letelier, as well as the Group of Seven.
During Operation Condor, the Night of the Pencils resulted in the kidnapping and torture and murder of many students.
The operation expanded into the Dirty War under Jorge Videla in Argentina.

\subsection*{Argentina}
% Peron

\subsection*{Conflict}
% Triple Alliance
% War of the Pacific
% Falkland War

\section{Eastern Asia: Independence and Leaders}

\subsection*{Indian Independence}

\subsection*{Communism in China}
% Mao
% Chinese Civil War
% Chiang Kai-shek
% Sun Yat-sen
% Zhou Enlai

\subsection*{Cambodia: Khmer Rouge}

\subsection*{Sri Lanka}

\subsection*{Indonesia}

\subsection*{Burma}
% Operation Dracula
% fight sword with sword
% way to socialism
% 8888 Uprising
% SLORC
% Aung San Suu Kyi
% Than Shwe

\section{Middle East}

\subsection*{Turkey}
% Ataturk

\subsection*{Iran}
% Shah
% Ayatollah
% Iran Hostage Crisis
% Canadian Caper

\subsection*{Israel}

\section{African States}

\subsection*{Egypt}
% Yom Kippur War

\subsection*{Rwanda}

\subsection*{Uganda}

\subsection*{Ethiopia}

\subsection*{Somalia}

\subsection*{Angola}
% Lusaka Accord
% Jonas Savimbi
% MPLA, UNITA
% Augustinho Neto, Holden Roberto

\subsection*{Zimbabwe}
% Mugabe
% Canaan Banana
% Lancaster House Agreement

\subsection*{South Africa}
% Apartheid
% Soweto Riots
% Sharpeville
% Mandela
% Hendrik Verwoerd
% Bantustan
