\chapter{Recent World History}

\epigraph{%
  Judgment comes from experience, and experience comes from bad judgment.
}{Simon Bolivar}

Keep in mind that we haven't looked at some parts of the world in quite a long time.
As a result, ``recent'' in some of these places may go back quite a while\ldots{}

\section{South America: Wars and Dictators}

\subsection*{Bolivar's Fight for Independence}

For this part, we go all the way back to the first part of the 19th century.
Spain and Portugal ruled South America, and in the first quarter of the 1800's, the wars of independence began.

In 1819, \textbf{Simon Bolivar} came up with a plan to take New Granada by crossing the Andes.
He'd previously given his Decree of War to the Death during his Admirable Campaign through M\'erida.
Bolivar aimed to create an army himself and invade Venezuela with the help of Antonio Jose de Sucre.
After winning at the Battle of Carabobo, the Battle of Pinchincha, and finally at the Battle of Boyaca,
Bolivar created \textbf{Gran Colombia} in northern South America and became its first president.
In 1828, an assassination attempt on Bolivar was stopped by his mistress Manuela Saenz,
and Bolivar called her ``Libertadora del libertador'' (liberator of the liberator).
At the \textbf{Guayaquil Conference}, Bolivar met with Jose de San Martin and convinced him to retire to France.

\subsection*{Brazil}

When King Joao~VI of Portugal left Brazil in 1821, he left behind his son,
who became \textbf{Emperor Pedro~I of Brazil}, aka ``Dom Pedro~I''.
He abdicated in 1831, leaving behind his young son as the new emperor.

When \textbf{Pedro~II} took power, he was only five, so regents ruled until 1840.
At this time, there were loads of revolutions and wars, including:
the Sabinada,
the Ragamuffin War or War of Tatters (recall Garibaldi?),
the Male Revolt,
Cabanagem and Balaida,
and others.
Then, Pedro started a new parliamentary reign.
He passed the Law of Free Birth, and his daughter Isabel passed the Golden Law which abolished slavery.
Pedro was ousted in 1889 by a coup led by Deodoro da Fonseca, and a republic was established in Brazil.

The Old Republic lasted from 1889 to 1930, but I'm not going to talk about it much.

After 1930, \textbf{Getulio Vargas} led a new military junta.
He was called the ``Father of the Poor'', and he set up an Estado Novo in Brazil,
inspired by the Estado Novo that Antonio Salazar had set up in Portugal.
He replaced the ``cafe com leite'' (coffee with milk) politics that had dominated the old republic,
which had placed a lot of power in the hands of the very rich and particularly alert coffee industry.
He created the fake communist ``Cohen Plan'' to foment hatred of communists,
and he supported Plinio Salgado's far-right political organization until Salgado targeted him in the Pajama Revolt.
Vargas was deposed in 1945 and replaced with Jose Linhares,
but he became a democratically elected president in 1951.
When his bodyguard tried to kill his rival Carlos Lacerda on Rua Tonelero, people called for Vargas's resignation.
He killed himself in 1954 and was replaced by his Vice President Caf\'e Filho.

\subsection*{Chile}

Chile's independence movement began in 1810, when a new junta was established.
The new government was led by Jose Miguel Carrera, who ruled with an iron fist.
Unhappy about this, \textbf{Bernardo O'Higgins} led a group of rivals that aimed to get full independence.
After a loss at Rancagua, he and his friend \textbf{Jose de San Martin}, the ``Bolivar of the South'',
led the Army of the Andes to victories at Maipu and \textbf{Chacabuco}.
O'Higgins served as the second Supreme Directory of Chile from 1817 to 1823.
Having contextualized Chilean independence in the early 1800's\ldots{}

In 1970, Marxist \textbf{Salvador Allende} was elected president.
He was part of the new Popular Unity Party, and he defeated Jorge Alessandri to win the election.
He dealt with the El Teniente copper mine strike,
and he went through the Cuban packages smuggling scandal during which arms were discovered in the packages.

In August 1973, Carlos Prats, a member of Allende's cabinet, was replaced by \textbf{Augusto Pinochet}.
A month later, on September 11, 1973, Pinochet staged a CIA-backed coup against Allende.
Allende apparently committed suicide in the presidential palace, La Moneda,
and Pinochet seized power in Chile.

Having come to power, Pinochet ordered the assassination of everyone he didn't like by the Caravan of Death.
He arrested Defense Minister Orlando Letelier, imprisoned him for a year, and tortured him;
Letelier was eventually assassinated on Pinochet's orders.
Pinochet killed Carlos Prats, utilized the DINA secret police, and he was supported by the Carabineers.
He also employed the \textbf{Chicago Boys}, a group of Chilean economists,
who had trained at the University of Chicago under Milton Friedman and Arnold Harberger,
and who lowered tariffs and privatized industry following Allende's Marxist regime.
Pinochet led \textbf{Operation Condor} in the 1970's, suppressing and killing leftists in South America.
The campaign was what targeted Orlando Letelier, as well as the Group of Seven.
During Operation Condor, the Night of the Pencils resulted in the kidnapping and torture and murder of many students.
The operation expanded into the Dirty War under Jorge Videla in Argentina.

\subsection*{Argentina}

Argentina was placed into turmoil in 1943 by a coup led by Pedro Pablo Ramirez, Arturo Rawson, Edelmiro Farrell,
and \textbf{Juan Peron}.
They marched on the Casa Rosada and demanded that President Castillo step down,
which he did, and ended the Infamous Decade of Argentina.
Ramirez took power, and broke relations with the Axis.
He was replaced by Farrell, and Peron helped manage relations with laborers until he was deposed.
Peron was detained on Martin Garcia Island, until he was brought back in 1945 and won elections.

Peron's second wife was \textbf{Eva ``Evita'' Peron}, who helped out a lot and then died of cancer in 1952.
She did the Rainbow Tour to try and improve relations with Europe.
When Evita died, Peron's third wife, Isabela, was Vice President.

Juan Peron censored lots of things and the number of unionized workers went up.
His government was isolationist, and he tried to avoid the Cold War using his ``Third Way'' policy,
which ended up being targeted by the Dirty War mentioned earlier.
Peron was supported by the descamisados (shirtless ones).
When a group of his Monteneros supporters went to meet him at the airport,
they were killed in the Ezeiza massacre.
Peron was deposed in 1955 and exiled by Eduardo Lombardi, but he came back to power in 1973.

\subsection*{Conflict}

\subsubsection*{War of the Triple Alliance}

The \textbf{War of the Triple Alliance} took place from 1864 to 1870.
The popular theory of the cause of the war is that Paraguayan President \textbf{Francisco Solano Lopez}
wanted to get control of the Platine basin.
Paraguay started fighting Brazil in 1864 when Paraguay invaded the Mato Grosso.
By the next year, Argentina and Uruguay entered the war,
allying with Brazil and greatly outnumbering Paraguay.
During the war, Bartolome Mitre led Argentinian forces and Pedro~II was in charge in Brazil.
Immediately prior to the start of the war,
Venancio Flores and the Blanco Party took power in Uruguay, overthrowing Atanasio Aguirre and the Colorado Party.

The initial offensive of Paraguay stalled after they lost at the Battle of Yatay.
Marshal Caxias of Brazil retrained the Allied army for a few months.
The turning point of the war was the naval battle of Riachuelo.
The war ended at Cerro Cora, where Solano Lopez was killed.
Paraguay, needless to say, was utterly demolished by this uneven war,
and the country lost 60\% of its population, leaving a woman-man ratio of 4 to 1.

\subsubsection*{War of the Pacific}

The \textbf{War of the Pacific} was fought, as you may have guessed, along the Pacific coast of South America,
between 1879 and 1883.
On one side was Chile, and on the other, the allied nations of Bolivia and Peru.
In February 1878, Bolivia imposed a tax on a Chilean mining company,
even though they'd previously said they wouldn't.
So Chile occupied Antofagasta, a port city.
Peru was drawn into the whole ordeal because of a secret treaty they'd made with Bolivia.
It's worth noting that countries were also after the valuable guano and saltpeter in the Atacama.
On March 1, Bolivia declared war and Chile declared war right back on both countries.

Chile's land war eventually defeated the allied nations.
Bolivia retreated after the Battle of Tacna (1880),
and the Peruvians were defeated at the Battle of Arica (1881).
Other important battles include the Battle of Angamos,
where Captain Miguel Grau was killed when his ship \textit{Huascar} was captured;
he'd earlier helped sink \textit{Esmeralda} at the Battle of Iquique.
Chile occupied Lima in January 1881, while Peru tried to stop the Chileans with a guerrilla campaign.
The war ended with the Treaty of Ancon;
Bolivia became landlocked, and the Tarapaca province was given to Chile.

\section{Eastern Asia: Independence and Leaders}

\subsection*{Indian Subcontinent}

\subsubsection*{Independence}

The British Raj had ruled over India since 1858.
There had been some militant movements for independence in Bengal but that didn't really work out.
So, let's forward a bit to the 20th century.

\textbf{Mohandas ``Mahatma'' Gandhi} was born in India and went to learn law in London.
He then spent some time as a lawyer in South Africa,
helping the Indian community there with civil rights using nonviolent civil disobedience.
While in South Africa, he founded Phoenix Farm and Tolstoy Farm.
He came back to India in 1915, and he started to organize a peasant movement to protest high taxes and injustice.
He took control of the Indian National Congress and started his whole campaign for independence.
During the early twenties, he led the Non-Cooperation Movement against the Rowlatt Act.

Gandhi decided to try for \textit{swadeshi}, boycotting foreign stuff,
trying to increase use of homespun cotton fabric.
After the Purna Swaraj declaration of independence of India in 1930,
Gandhi led the \textbf{Dandi Salt March} to protest the salt tax.
On March 12, he started the march with less than 80 people,
and by the end of the march on April 6, the group was almost two miles on.
On reaching Dandi, Gandhi picked up some mud and boiled it,
claiming to shake the very foundations of the British Empire.
For this, he was put in Yerwada Jail for eight months along with about 60,000 other Indians.
Gandhi would later be assassinated by \textbf{Nathuram Godse}.

Of course, there were other notable people in the period that India was working for independence.
People like \textbf{Subhas Chandra Bose} and \textbf{Bhagat Singh} wanted armed resistance.
During the Second World War, Gandhi's Quit India Movement and Bose's Indian National Army peaked.
In 1947, the \textbf{Indian Independence Act} was signed, and the new countries of India and Pakistan were formed.

\subsubsection*{A New India}

The first prime minister of the new India was \textbf{Jawaharlal Nehru},
who had split the country with Pakistan in 1947 very soon after independence.
Nehru had given the \textbf{Tryst with Destiny} speech right before independence:
\begin{quote}
  Long years ago we made a tryst with destiny, and now the time comes when we shall redeem our pledge,
  not wholly or in full measure, but very substantially.
  \textbf{At the stroke of the midnight hour, when the world sleeps, India will awake to life and freedom.}
  A moment comes, which comes but rarely in history, when we step out from the old to the new,
  when an age ends, and when the soul of a nation, long suppressed, finds utterance.
\end{quote}
He created the States Reorganization Commission, headed by Faizal Ali,
in 1953 to reorganize the state boundaries in the country (the process ended up taking 2 years).
Nehru instituted the Forward Policy, in which he set up a bunch of forts along the Chinese border.
This started the \textbf{Sino-Indian War}, a border dispute with China that included the Battle of Thag La Ridge.
People didn't like Nehru after this war, and he lost office in 1964.

The third prime minister was \textbf{Indira Gandhi}, Nehru's daughter.
She won after defeating Morarji Desai.
After losing power for a bit, she came back in 1971 along with her Congress party.
India intervened in the Bangladesh Independence War,
and Bangladesh (formerly East Pakistan) became independent as well.
Gandhi started a two year state of emergency in the country,
and was forced out in the late seventies by the Janata Party.

In the start of 1980, however, Indira Gandhi came back to power.
Insurgency in Punjab became a problem quickly, and Gandhi decided to order \textbf{Operation Blue Star},
in which Indian troops raided militant headquarters in the Sikh Golden Temple of Amritsar.
This didn't go so well, and a lot of civilians died.
Sikh people across the country were infuriated with Gandhi.
Her own Sikh bodyguards Beant Singh and Satwant Singh assassinated Gandhi in 1984.

\subsubsection*{Pakistan}

The first leader of the new Pakistan was \textbf{Muhammad Ali Jinnah}.
He'd issued Fourteen Points that dealt with religious minority protections.
When Congress Party walked out of government during World War~II, he called for a ``Day of Deliverance''.
He broke with Nehru over the two-state idea and the Lahore Resolution.
He set up the Lucknow Pact and preceded Muhammad Iqbal as leader of the Muslim League.
When he called for ``Direct Action'' (the Great Calcutta Killing), riots started in Bengal.

\subsection*{Communism in China}
% Mao
% Zhou Enlai

In the early 20th century, people were getting fed up with the Qing dynasty, and they revolted.
The revolutionaries were led by \textbf{Sun Yat-sen},
and the Wuchang Uprising led to the new provisional government of the Republic of China, headquartered in Nanjine.
Sun Yat-sen was declared the first president of the new republic.

Sun was succeeded by \textbf{Yuan Shikai},
who was made president because he was able to get Emperor Puyi to step down.
Meanwhile, Sun Yat-sen and Song Jiaoren formed the new \textbf{Kuomintang} (KMT), China's nationalist party.
Yuan started persecuting the KMT, who ended up winning elections in 1912--1913.
The Second Revolution took place when Sun and KMT forces tried to overthrow Yuan,
but they failed and Sun ran to Japan while Yuan likely ordered a successful hit on Song Jiaoren.
Yuan created the Empire of China in 1915, and he set himself up as emperor.
But he died in 1916 right after abdicating, and this led to the Warlord Era of China.
China fell apart.

In the 1920s, Sun Yat-sen set up a base in the south to try and unite the broken country.
He set up alliances with the USSR and the Communist Party in China (CPC),
talking to Mikhail Borodin at a conference in Penang.
He delivered a speech talking about the Three Principles of the People:
nationalism, democracy, and the people's livelihood.
He also put forward the Five-Power Constitution.

On Sun's death, his prot\`eg\`e \textbf{Chiang Kai-shek} took control of the KMT\@.
He'd been the first commander of the KMT Whampoa Military Academy that Sun had set up,
and in 1926 he led the Northern Expedition against the \textbf{Beiyang Government}
(which had been created by Yuan Shikai)
to try and unify the country.
Chiang became the president of China.
He set up the New Life Movement, and with the help of the Blue Shirts Society,
he tried to spread education of a new vaguely fascist set of ideals.
During the Xi'an incident in 1936, Chiang was kidnapped by Marshal Zhang Xueliang,
and Soong Mei-ling (Madame Chiang Kai-shek) helped negotiate an end to the incident.
Chiang set up a truce with the CPC to fight Japan before the Second World War,
but truces between enemies aren't really made to last.

After Japan was defeated, the civil war started up again between the Nationalist forces and the CPC\@.
The Marshall Mission, in which the Americans tried to help set up a coalition government,
failed in 1946 and the war continued.
By 1949, the Nationalists were defeated because the CPC were simply better at war,
as well as the fact that Chiang had managed to antagonize a good chunk of the country.

Upon defeat, Chiang and the Nationalist forces retreated to Taiwan (called Formosa).
Chiang set up a new martial law under the KMT,
going after people who criticized him with the help of the Green Gang.
This period was known as the White Terror.
Chiang died in 1975.

With the KMT out of the way in Taipei,
the CPC was free to do whatever they wanted on the mainland.
On October 1, 1949, \textbf{Mao Zedong} proclaimed the People's Republic of China.
Mao had become head of the CPC during the \textbf{Long March},
when they had been forced to retreat from the KMT\@,
He'd also joined forces with Zhu De in order to create the Red Army.
For the next few years after taking power, he set up his Marxist/Leninist ideals in the country,
which would later be detailed in his \textbf{Little Red Book}.

Mao launched the Hundred Flowers Campaign in 1956, encouraging people to openly express their ideas.
Of course, when he decided that he didn't like people disagreeing with him,
he quickly changed tack and cracked down on dissidents as part of his Anti-Rightist Campaign.
In 1957, he launched his \textbf{Great Leap Forward},
trying to turn China from an old-fashioned agrarian state into a modern industrial one.
This had the particularly impressive effect of transforming it into neither of the above, and leaving it in famine.

Mao started the \textbf{Cultural Revolution}, in which he tried to get rid of Chinese traditionalism.
The main objective was to destroy the Four Olds (Old Customs, Old Culture, Old Habits, and Old Ideas).
Mao issued the ``Bombard the Headquarters'' document, which incited reactions from the public.
The Red Guards attacked many people in the country, killing and pillaging as they saw fit.
During the 1968 ``Down to the Countryside Movement'',
Mao sent a bunch of privileged city kids to go learn from farmers in the middle of nowhere.
While the Cultural Revolution was going on, the \textbf{Gang of Four} came to prominence,
led by Mao's last wife, \textbf{Jiang Qing}.

Under Mao, the first Premier of the PRC was \textbf{Zhou Enlai},
who helped out with foreign policy related issues.
He had previously helped negotiate Chiang Kai-shek's release during the Xi'an incident.
After Nehru was elected, Zhou talked with him and came up with the ``Five Principles of Peaceful Coexistence''.
He administered China during the Cultural Revolution.
John Foster Dulles, Eisenhower's Secretary of State, didn't want to shake Zhou Enlai's hand,
but Kissinger smoothed over relations when discussing Nixon's visit to China.

Mao died in 1976, and a power vacuum was created.
The Gang of Four were blamed for the craziness that was the Cultural Revolution.
Mao's appointed successor was \textbf{Hua Guofeng}, but he wasn't going to stay in power for long.
Before long, \textbf{Deng Xiaoping} defeated Hua Guofeng
and took control as the Paramount Leader of China from 1978 to 1992.
Deng advocated the Four Modernizations, and he created Special Economic Zones to help the economy,
pushing for his ``one country, two systems'' policy.

The death of former general secretary Hu Yaobang in 1989
resulted in a funeral that ended in \textbf{Tiananmen Square}.
Students and protesters called for governmental reform and about a million people stood in the square.
Initially, the government didn't do anything, but the protests kept going.
Then, Deng decided to use force to resolve the protests, sending in tanks and declaring martial law.
Jeff Widener took a picture of a man standing in front of a column of tanks.
The incident resulted in really hurting Deng's popularity,
who had actually been doing pretty well up until that point.

\subsection*{Cambodia: Khmer Rouge}

In 1945, the Japanese were occupying Cambodia.
The young king \textbf{Norodom Sihanouk} declared the existence of the independent Kingdom of Kampuchea,
and the Japanese ratified it and the new government set up shop in Phnom Penh.
He stayed around until he was ousted by Prime Minister \textbf{Lon Nol} in 1970 while he was in Beijing.
He quickly moved to ally Cambodia with the US\@.

In April of that year, Nixon announced that the Cambodian Incursion of US and South Vietnamese troops had begun.
The aim was to destroy the NVA bases in the country.
This came as no surprise; the US had been bombing Cambodia for a while at this point.
In 1972, Lon Nol became president.

All this while, the \textbf{Khmer Rouge} (CPK) insurgency was growing within Cambodia.
It was led by \textbf{Pol Pot} and Ieng Sary, who were backed by North Vietnam.
They slowly weaned off Vietnamese dependency and on 1975, they attacked in full.
In 117 days, the Khmer Republic fell and Lon Nol surrendered.

Not one to be outdone by Chairman Mao, Pol Pot set up the Super Great Leap Forward,
instituting a bunch of purges after declaring that this was Year Zero.
The CPK subsequently ordered everyone out to go work on farms,
trying to rebuild the country to conform to Pol Pot's ideas.
The farming collectives soon came to be known as the ``killing fields''.
Pol Pot set up a prison camp codenamed S-21\ in a high school and killed 20,000 people.
In 1979, the Vietnamese army invaded Cambodia,
creating the new People's Republic of Kampuchea and forcing Pol Pot to the Thai border.
He killed himself when his party decided to turn him over to international authorities.

\subsection*{Indonesia}

In 1942, Japan was occupying Indonesia.
They offered \textbf{Sukarno} some power in order for him to make everyone else in the country support Japan.
Sukarno instituted ``guided democracy'' in the country when he became the first president.
He set up the Pancasila (Five Principles) in a 1945 speech.
After the 30 September Movement, he was deposed and replaced by \textbf{Suharto}.

Suharto, head of the Golkar Party, came to power in 1967.
He was apposed by Megawati Sukarnoputri, Sukarno's daughter, who led the PDI\@.
His New Order government \textbf{invaded East Timor} in Operation Komodo,
occupying it and beginning a bloody era that laster from 1975 until 1999.
He was forced to resign in 1998 following a riot at Triskati University and his failure to manage the economy.
He was succeeded by Bacharddin Jusuf Habibie, but he only lasted until 1999 because nobody liked him.

\subsection*{Burma}

Here's a country with history that comes up sometimes, but often in fragments.
So, I'm just going to go over the bits that matter.

During World War~II, Operation Dracula was an Allied attack on Burma.
This was part of the Burma Campaign, and the operation's goal was to attack Rangoon and reoccupy Burma.
The Japanese had already left, so occupation came swiftly.

The \textbf{8888 Uprising} took place on August 8, 1988, which lends it its name.
Note that until this point, the Burma Socialist Programme Party had reigned supreme under Ne Win.
The Way to Socialism had wrecked Burma's economy, and people were not happy.
The goal of the uprising was democracy, and it was violently suppressed,
but it did manage to get Ne Win out of power.
In 1988, the State Law and Order Restoration Council (SLORC) took power.

In 1990, elections resulted in a win for the National League for Democracy (NLD),
notably including \textbf{Aung San Suu Kyi}.
But the military quickly put a stop to that and put her under house arrest.
Everyone in the world got mad at them, and Aung San Suu Kyi was given the Nobel Peace Prize in 1991.
The military put Than Shwe in power in 1992.
He eventually released Aung San Suu Kyi.

\section{Middle East: War and Turmoil}

\subsection*{Turkey}

In the early 20th century, the Ottoman Empire was an entity rooted in the past,
the last remnants of a bygone era of kings and sultans that had peaked with Suleiman the Magnificent.
Time had eroded the borders of the empire, resulting in the loss of Greece, Algeria, and Tunisia.
Beset on all sides with attackers, the Ottomans built a new alliance out of necessity with Germany.
They went into the First World War and managed to hold their own (certainly better than Austria-Hungary did)
but were ultimately defeated by the Allied Powers.
The \textbf{Treaty of Sevres} broke up the empire and distributed it amongst Greece, Italy, Britain, and France.

Into this came Mustafa Kemal, a soldier who had made a name for himself in the Gallipoli Campaign,
having been saved from certain death by a pocket watch that had blocked a piece of flying shrapnel.
Under him, the Turkish War of Independence raged in the country, aiming to revoke the Treaty of Sevres.
By the end of 1922, the new army expelled the occupying forces, overthrew the sultanate,
and set up a new parliament.
The \textbf{Treaty of Lausanne}, signed by Mustafa Kemal in 1923,
resulted in the formation of a new Republic of Turkey.
Mustafa Kemal was made first president of the new republic,
and he was rewarded with the honorific \textbf{Ataturk} and hailed as the ``father of the Turks''.

Ataturk's philosophy was characterized by the ``Six Arrows'':
Republicanism, Nationalism, Populism, Revolutionism, Secularism, and Statism.
His policies included adoption of the Latin alphabet and the metric system.
He prohibited civil servants from wearing the Ottoman fez, a relic that was not welcome in Ataturk's new republic.
Ataturk also established full rights for women politically.
When he died in 1938, he designated Ismet Inonu as his successor.

\subsection*{Iran}

Following the end of the Second World War, the Tehran Declaration allowed Iran to have independence.
The young Shah \textbf{Mohammad Reza Shah Pahlavi} came to power,
and his constitutional monarchy started to work well.
He had a hands-off role in government, and by 1950 he had created the new Senate,
which elected \textbf{Mohammad Mosaddeq} as prime minister in 1953.
Mosaddeq nationalized British oil interests,
and he forced the shah into exile following an attempted coup by intelligence chief Nematollah Nassiri.

Now, the CIA and MI6, under the direction of Kermit Roosevelt, decided to organize a coup under Fazlollah Zahedi
(codenamed Operation Ajax and Operation Boot, respectively).
The coup took Mosaddeq out of power, and installed Pahlavi as shah again.
Pahlavi introduced the White Revolution, modernizing the country.
He was a secular Muslim, and he tried to westernize Iran.
His new ideas and reforms didn't leave him without enemies,
and Pahlavi used his SAVAK secret police to crush opposition to his regime.

In 1979, opposition to the shah came to a head in the Islamic Revolution.
A huge outpouring of support ushered \textbf{Ayatollah Ruhollah Khomeini} into power.
Khomeini had been saved from execution a few years prior by Hassan Pakravan,
a member of SAVAK, whom Khomeini promptly executed \textit{because} he was a member of SAVAK,
even though Pakravan had saved his life.
Iran quickly shifted into an Islamic state,
replacing a capitalist economy and social structure with a nationalized, heavily regulated one.
Khomeini's ideology was called the ``Guardianship of the Jurist''.

The first major event worth noting in Khomeini's Iran was the \textbf{Iran Hostage Crisis}.
When the US admitted the former shah into America for cancer treatment,
Iranian students overran the US embassy, taking 52 hostages on November 4, 1979.
Carter then launched \textbf{Operation Eagle Claw} to try and rescue the hostages.
Helicopters launched from USS \textit{Nimitz} ran into issues
with obscure sand-based weather phenomena called \textit{haboobs},
in which sand is suspended in the air, and they were forced to abort.
The failure of Eagle Claw damaged American reputation across the world.
Notably, the Canadian government launched the \textbf{Canadian Caper} to rescue six Americans trapped in Tehran,
who had evaded capture with the rest of the embassy.
They pretended that they were scouting locations for shooting a scene of a fictional movie called \textit{Argo},
and Canadian ambassador Ken Taylor helped pull off a successful rescue (watch the Oscar winning movie).

It's also worth discussing the \textbf{Iran-Iraq War}, which ran from 1980 to 1988.
\textbf{Saddam Hussein} decided to invade Iran because of the unstable state it was in as a result of the revolution.
This was the war in which the Iran Contra affair took place,
and the US ran Operations Earnest Will and Prime Chance during the ``Tanker War'' phase of the war.
During \textbf{Operation Praying Mantis}, the US launched attacks on oil platforms.
Iraq authorized the Halabja poison gas attack, which attacked the Kurdish population.
A ceasefire was signed in 1988.

\subsection*{Israel}

Israel gained independence in 1949, signing armistices and ceasefires with friends and neighbors.
The \textbf{Knesset} (parliament) was convened and moved to Jerusalem.
The first elections resulted in the Socialist-Zionist parties of Mapai and Mapam winning most of the seats.
The leader of Mapai was \textbf{David Ben-Gurion}, who became Israel's first prime minister.

Ben-Gurion ordered Operation Magic Carpet, which successfully migrated lots of people out of Yemen.
Ben-Gurion also founded Ahdut HaAvoda, which would eventually become the Labour Party.
In 1956, he launched the Sinai War when Nasser nationalized the Suez Canal.
The Lavon Affair, in which Operation Susannah failed to blame the Muslim Brotherhood for some bombs being planted.
The affair resulted in Defense Minister Pinhas Lavon resigning,
and Ben-Gurion soon left the party, being succeeded by Moshe Sharett and Levi Eshkol.

The 1967 \textbf{Six Day War} pitted Israel against Egypt, Syria, and Lebanon.
It was started by Israeli preemptive strikes on airfields in the opposing countries.
During the brief war, Israel took the Gaza Strip, the Sinai Peninsula, and the West Bank.
It was ended by UN Resolution 242, and it led to the Khartoum Resolution,
declaring that there would be ``no peace, no recognition, and no negotiation''.

In 1969, Eshkol died in office, giving the office to Labour leader \textbf{Golda Meir},
winning 56 of 120 seats in the Knesset.
Meir had previously been Foreign Minister under Ben-Gurion and Eshkol.
She was the first female prime minister of Israel, and like Thatcher, she was called the ``Iron Lady''.
During the War of Attrition, the Israelis shot down some Soviet fighters because they were helped Egypt.
During the 1972 Summer Olympics in Munich, 11 Israeli team members were taken hostage by Palestinian terrorists,
and they were killed as a result of a failed German rescue.
In response, Meir authorized Operation Wrath of God,
in which Mossad assassinated people that were involved in the incident.

The \textbf{Yom Kippur War} with Sadat's Egypt started on October 6, 1973.
Syrian and Egyptian forces launched a surprise attack on Israeli troops
at Golan Heights and across the Suez Canal (Operation Badr), respectively.
Israel hadn't mobilized, partially because Meir and Moshe Dayan rightly believed
that the US wouldn't help Israel if it instigated a war.
Of course, American aid to Israel resulted in an OPEC oil embargo against the US\@.
Other Israeli commanders included David Elazar and Israel Tal.
UN Resolutions 338 and 339 called for ending the war.
After the war, Kissinger negotiated troop disengagements using ``shuttle diplomacy''.
Meir was succeeded by \textbf{Yitzhak Rabin}, who was notably assassinated by Yigal Amir.

\section{Newly Independent African States}

\subsection*{A New Egypt}

The Egyptian Republic was declared in June 1953, with Muhammad Naguib as the first president.
He was forced to resign the following year and \textbf{Gamal Abdel Nasser} came to power.
Nasser had headed the Free Officers Movement that had dethroned King Farouk in the first place.
Nasser nationalized the Suez Canal, leading to the Suez Crisis,
and he allied with Syria in the United Arab Republic.
Recall that while he was president, Egypt fought in the Six Day War.
He also built the Aswan High Dam, resulting in the creation of Lake Nasser.
After he was almost assassinated (recall the Lavon affair), he banned the Muslim Brotherhood,
executing Sayyid Qutb.

Nasser died in 1970 and he was succeeded by \textbf{Anwar Sadat}.
Sadat launched the Yom Kippur War (October War).
His \textit{Infitah} economic policy wasn't particularly popular, leading to the Bread Riots.
He signed the Camp David Accords with \textbf{Menachim Begin} (with help from Jimmy Carter),
winning the Nobel Peace Prize.
He was assassinated by Khalid Islambouli for the same accords and succeeded by \textbf{Hosni Mubarak}.

\subsection*{Rwandan Genocide}

On April 6, 1994,
a plane carrying President \textbf{Juv\'enal Habyarimana} and Burundian President Cyprien Ntaryamira
was shot down on descent into the Rwandan capital of Kigali, and everyone on the plane died.
Habyarimana was the head of the \textbf{Hutu}-led Rwandan government,
which had come to power after overthrowing Gregoire Kayibanda.
His death prompted genocidal killings against \textbf{Tutsi} opposition to start.
The \textbf{Interahamwe} and Impuzamugambi militias
went around recruiting other Hutus to pick up whatever weapons they could find and help out.
Their RTLM broadcast messages telling Hutus to ``cut down the tall trees''.
One of the first actions taken was when some ``cockroaches'' were killed in a Polish church in Gikondo.

The UN launched UNAMIR (UN Assistance Mission for Rwanda), and some Belgian soldiers were killed.
Force Commander Romeo Dallaire was criticized for inaction,
and after the genocide he went to Canada and wrote \textit{Shake Hands with the Devil}.
The diplomatic officer was Jacques Roger Booh Booh.
The French tried to intervene in Operation Turquoise, setting up a safe zone (Zone Turquoise).

The genocide ended with \textbf{Paul Kagame} and his RPF taking control of Kigali.
The Tanzanian city of Arusha held a criminal tribunal for the genocide.

\subsection*{Uganda: Idi Amin}

Uganda got independence from Britain in 1962, and the first Chief Minister was Benedicto Kiwanuka.
In 1966, \textbf{Milton Obote} became prime minister and suspended the constitution.
On January 25, 1971, \textbf{Idi Amin}, a commander in the army, led a military coup,
taking Obote from power and putting himself in his place.
He would rule for the next eight years.

Amin set out to kill everyone who opposed him (as is customary for evil dictators).
He killed over 300,000 people during his time in office,
and when he pushed out all the Indian entrepreneurs from the country during ``Africanization'',
the economy fell to bits.

On June 27, 1976, an Air France flight was hijacked by Palestinian terrorists,
whom Amin allowed to land at Entebbe airport (the biggest in Uganda).
They proceeded to let all the non-Israeli passengers go, and kept the crew and the rest of the passengers.
The Israel Defense Forces (IDF) put Operation Thunderbolt into effect.
Led by Lt. Col. \textbf{Yonatan Netanyahu}, commandos secured 102 hostages,
resulting in 5 wounded commandos and the death of Netanyahu.
Amin proceeded to order the death of many Kenyans in Uganda because Kenya had supported Israel.

When Amin tried to annex the Kagera province of Tanzania in 1978, he started the Uganda-Tanzania War.
The war led to the end of his rule, and Amin went into exile, dying in 2003.



\subsection*{South Africa}

\textbf{Apartheid} was a South African system of segregation that started in 1948 and stayed around until 1991.
It was staunchly advocated by H.F. Verwoerd, who would end up being killed in parliament in 1966.
Apartheid party members created Bantustans, which were territories set aside for black residents of the country.
The UN declared it a crime against humanity in 1966.
Apartheid led to the Sharpeville Massacre,
and during the Soweto Uprising, many people including Steve Biko were killed.
During the Rivonia Trial, ten members of the African National Congress (ANC) were charged and some were convicted
for fighting against the state, essentially.

Among those convicted was the Xhosa ANC leader \textbf{Nelson Mandela},
who was placed in prison on Robben Island for the next 27 years.
During the Rivonia Trial, he had said:
\begin{quote}
  During my lifetime I have dedicated myself to this struggle of the African people.
  I have fought against white domination, and I have fought against black domination.
  I have cherished the ideal of a democratic and free society in which all persons live together
  in harmony and with equal opportunities.
  It is an ideal which I hope to live for and to achieve.
  But if needs be, it is an ideal for which \textbf{I am prepared to die}.
\end{quote}
Mandela served 27 years of his life sentence, and after being freed,
he succeeded F.W.\ de Klerk as president of South Africa.
One of his important acts of unity was presenting the winners' trophy at the Rugby World Cup.
He established the Truth and Reconciliation commission.
Mandela offered South Africa as a site for the trial of the Lockerbie Bombers (Pan Am Flight 103).
