\chapter{Medieval Times}

After the fall of the Western Roman Empire, the Middle Ages began.
They would last until the start of the Renaissance in the 15th century,
and they were characterized by massive population growth and war in Europe.
Let's look at some areas of Europe in particular.

\section{England: A New Monarchy}

\subsection*{Anglo-Saxon England}

Near the beginning of the Middle Ages, the Anglo-Saxons lived in England,
a result of Germanic tribes moving north across the English Channel.
Anglo-Saxon England was traditionally considered a heptarchy, consisting of:
Northumbria, Mercia, East Anglia, Essex, Kent, Sussex, and Wessex.
By the 7th century, much of the Anglo-Saxons had converted to Christianity,
and the Venerable Bede chronicled this era, earning him the title ``Father of English History''.

The kingdom of Mercia rose to prominence in the 7th century, building massive earthworks such as
Offa's Dyke, which separated modern-day Wales from the rest of England.
By the late 8th century, however, Viking raiders began to invade the island.
These Danes were led by \textbf{Guthrum}, whom we don't know much about before he seized control of the Danes.
They subdued East Anglia, capturing York.

\textbf{Alfred the Great}, king of \textbf{Wessex}, son of \AE{}thelwulf, fought back against Guthrum.
Other Viking leaders of the invading forces included Ivan the Boneless, son of Ragnar Lodbrok.
He created a network of fortified towns called burhs (these would be turned into boroughs eventually)
to defend Wessex and defeat the invaders, who had formed the \textbf{Great Heathen Army},
at battles such as Eddington and Ashdown.
Alfred confined Guthrum in a region that became known as the Danelaw, and proclaimed himself king of all the English.
His coronation was performed by Pope Leo IV,
and he compiled a census of England called the Doom Book
(not to be confused with the Domesday Book that would be made later).
He introduced a system of shires, hundreds, and tithings to divide England into governable portions.
Alfred remains the only English king to ever be called ``the Great''.

Alfred was eventually succeeded by \textbf{Edgar the Peaceful}, and by the end of Edgar's reign,
Wessex had taken control of the rest of England, including York and the Danelaw.
When Edgar died only two years after being crowned at Bath,
the issue of who should succeed to the throne was thrown into doubt.
His son Edward was crowned king, but he was quickly assassinated.

He was followed by Edgar's half-brother, \textbf{\AE{}thelred the Unready}.
As his epithet may suggest, his reign didn't go very well, even though he did rule for a long time.
It's been written that he defecated in his baptismal font,
leading to a prophecy that the English crown would fall under his reign;
the story is probably entirely a fabrication, though.
It's now believed that he was simply the victim of bad counsel, and wasn't actually particularly ``unready''.
He had married Emma, a princess of Normandy.
While he was king, England was invaded by Danes again, led by Sweyn Forkbeard.
The Vikings pillaged through Essex, an action that resulted in the Battle of Maldon,
where the Vikings easily crushed the English.
From that point forward, the Vikings roamed the countryside, apparently free to take whatever they pleased,
while \AE{}thelred probably just hid in his basement.
Sweyn found \AE{}thelred and forced him into exile.
Soon after, though, Sweyn died, and \AE{}thelred was able to come back to England.

Angered at \AE{}thelred, \textbf{Cnut}, son of Sweyn, invaded England.
\AE{}thelred's son, Edmund Ironside, abandoned his father, and multiple English nobles took Cnut's side.
But, \AE{}thelred died, and Edmund was forced to fight the Danes.
At the battle of Ashingdon, the Danes overpowered the English, and Cnut and Edmund split England;
Edmund would take Wessex, and Cnut would take everything else.
When Edmund died (he was probably murdered), the English witan council declared Cnut king of England.
Cnut married \AE{}thelred's widow Emma,
and she agreed only because he agreed to allow her children to be heirs to the throne.
However, Cnut had had a previous wife, and her children also wanted the throne.

When Cnut died in 1035, the throne was disputed.
The claimants were Harald Harefoot, and Emma's son by Cnut, Harthacnut.
Emma's son by \AE{}thelred, Edward, tried to raid Southampton, but failed, and his brother Alfred was killed later.
Harthacnut became king, but he was unpopular, and nobody wanted his heirs to become king.
When he died, Edward came back from exile and was recognized as king.

Edward, who became known as \textbf{Edward the Confessor},
married the daughter of Earl Godwin of Wessex, in order to secure an alliance.
When the citizens of Dover killed Edward's relative Eustace of Bolougne
(probably a response to Edward's appointment of Robert of Jumieges as Archbishop of Canterbury)
and Godwin refused to punish some people who killed Edward's relatives,
Edward had a falling out with Godwin, and the Godwins ran away to Normandy.
During the conflict, Edward was regularly working with and for the earls Siward and Leofric.
At this time, Edward probably offered the succession to his cousin, William, duke of Normandy.
However, soon after, the king reconciled with Godwin, and when Godwin died in 1053,
his son, \textbf{Harold~Godwinson}, acceded to the earldom in Wessex.
Godwin's other sons were given East Anglia, Mercia, and Northumbria.

In 1066, Edward the Confessor died, and Harold~Godwinson was declared king of England.
In the following year, England would be in turmoil because,
even though Harold had taken the crown, there were those who believed it wasn't his to take.
Remember that Edward had given a promise of succession to his cousin, William, duke of Normandy.
\textbf{Harald~Hardrada} of Norway, descendant of Cnut, also staked a claim to the throne.

Harold's brother Tostig, leader in Northumbria, allied himself with Hardrada.
Hardrada invaded Northern England, fighting the Battle of Fulford near York.
However, Hardrada was killed at Stamford Bridge by Harold~Godwinson's forces.

While Harold was celebrating his victory, William, Duke of Normandy, sailed for England.
Harold marched back south, where he met William at the \textbf{Battle of Hastings}, near Senlac Hill and Telham Hill.
Harold used housecarls and shield walls to defend from William's archers.
William crushed Harold's forces at Hastings, and Harold himself probably got an arrow through the eye.
The battle was chronicled on the \textbf{Bayeux Tapestry}, which notably shows Halley's comet.
The site of the battle is now home to a church called Battle Abbey.
As a result of his achievement, William is now known to us as \textbf{William the Conqueror}.

\subsection*{Anglo-Norman England}

William the Conqueror, now William~I, king of England, quickly moved to consolidate his power.
His people were revolting all over the place, and he crushed each one,
building a set of castles to control the populace.
These revolts included one led by the Earls Edwin and Morcar.
He led a massive and bloody expedition through the north of England,
killing those who stood in his way, called the \textbf{Harrying of the North}.
He commissioned a census of the country, and recorded it in the \textbf{Domesday Book}.
At the Trial of Penenden Heath, William investigated Odo of Bayeux for stealing lands.
His archbishop was Lanfranc, and he stopped Edward Atheling, Harold Godwinson's successor.
Orderic Vitalis, when chronicling William's rule, says that William admitted to being a ``barbarous murderer''
when he died in 1087 after falling off his horse at the Siege of Mantes.

\textbf{William Rufus} (William~II) inherited his father's throne,
but he also faced revolts from his brother Robert or his cousin Stephen.
When William~II died in a hunting accident (we don't know if he was murdered),
his brother, Henry~I, quickly moved to take the crown.

\textbf{Henry~I}, ``Beauclerc'', defeated his brother Robert Curthose, duke of Normandy, at Tinchebrai.
He recalled St.\ Anselm to Archbishop of Canterbury, but exiled him again.
His son William Atheling drowned in the \textit{White Ship} disaster of 1120.
On Henry's death 1128, yet another power struggle broke out;
Matilda, Henry's daughter, disputed the claim of Stephen of Blois, Henry's nephew.
The civil war that broke out was called the Anarchy.

\subsection*{The Plantagenets}

Eventually, Matilda's son, \textbf{Henry~II} was crowned after a peace settlement.
Henry had been born in Anjou, making him the first of the Angevins,
and the first member of the House of \textbf{Plantagenet} to be king.
Henry married Eleanor of Aquitaine after her marriage to Louis~VII of France was annulled.
Henry got into disputes with the Church, and wanted power over them.
To do this, he appointed \textbf{Thomas Becket} to the position of Archbishop of Canterbury,
because Henry believed that Becket was a man that would be loyal to him.
But, Becket became a thorn in Henry's side, fighting to give the Church unrestricted power and freedom.
Henry didn't like this at all, and he issued the Constitutions of Clarendon
to restrict the power of Church courts and the power of the pope (Alexander~III) in England.
Becket resisted, and Henry exiled him to France.

However, by 1169, Henry wanted to crown his son Young Henry as king along with himself.
This required Becket to agree, so Henry tried to reconcile with him, but failed.
Henry crowned his son by himself, and Becket laid an interdict on England,
halting useful Church activity like weddings.
Eventually, Henry got fed up with Becket, and he got really mad, saying things like,
``Will no one rid me of this turbulent priest?''
In response to this, four knights went secretly to Canterbury and killed Becket,
an act which would result in Becket being declared a martyr and being declared a saint.

Henry wanted some more territory for his sons, so he invaded Ireland, conquered it,
and set up local fiefdoms to rule over the island.
He then faced the Great Revolt in which elder sons, supported by France's Louis~VII, rebelled.
The only son that didn't rebel against Henry~II was John.

Henry~II would be succeeded by \textbf{Richard~I, the Lionheart} in 1189,
because Henry's three eldest children, William, Henry, and Geoffrey, had died as young men.
Richard spent his reign mostly protecting his territory and fighting the Third Crusade in the East.

Richard's youngest brother, \textbf{King John Lackland}, inherited the throne in 1199.
He negotiated the Treaty of Le Goulet with Philip Augustus of France in 1200.
When John abandoned Isabel of Gloucester in order to marry Isabella of Angul\^eme, the peace was broken.
John had claimed that he had not obtained papal consent to marry Isabel in the first place.
John took William de Roches, his seneschal, and defended his lands in Normandy, and won the Battle of Mirebeau.
However, Philip would later win the Battle of Bouvines against John and his allies.
The nobles would eventually lose faith in John, and at Runnymede,
they forced him to sign \textbf{Magna Carta},
which gave them permission to overpower him if they didn't like what he did.
Magna Carta also ensured that raising taxes for scutage (ability to raise ransom) and other acts
could not be done without ``common counsel of the realm''.
John eventually reneged on Magna Carta, and the barons gave up and decided to revolt.
Pope Innocent~III excommunicated John, and John lost at Poitou.
This resulted in the First Barons' War, and John would be succeeded by his son Henry.

\textbf{Henry~III} came to power as a child, and his reign was filled with rebellion.
Simon de Montfort became a de facto ruler, and he drastically reduced power of the king,
going into the Second Barons' War, as well as establishing a more modern Parliament.
Henry's son, Edward, would stop the rebellion and restore Henry to power.

\textbf{Edward~I, Longshanks}, would rebuild the monarchy,
which had fallen into shambles as a result of the ineffective reigns of John and Henry~III\@.
Edward aimed to consolidate his power in Scotland,
which had previously had a slightly fuzzy relationship with the English crown.
He made \textbf{John Balliol} king in Scotland, thinking that John would be loyal to him, but he wasn't.
Balliol would be succeeded by John Comyn.
Edward invaded Scotland.
At the Battle of Stirling Bridge, he was repulsed by the Scots,
who were led at that point by \textbf{William Wallace}.
But, he struck back and the English defeated the Scots at the \textbf{Battle of Falkirk}.
When Edward captured and killed Wallace, \textbf{Robert the Bruce} rose to power in Scotland.
Robert killed John Comyn by stabbing him to death on an altar.
Edward would be known to posterity as ``The Hammer of the Scots'',
and when he died, he was succeeded by his son.

\textbf{Edward~II} would proceed to have a fairly pathetic time as king.
He married Isabella of France, daughter of Philip the Fair.
His favorite, Piers Gaveston, was so heavily favored by him that the populace forced Gaveston into exile.
Edward tried to continue his father's subjugation of Scotland, but he failed
when Robert the Bruce completely humiliated English forces at the \textbf{Battle of Bannockburn}.

\textbf{Edward~III} was a courageous and great ruler of England.
He defeated the Scots at Halidon Hill, and took back a substantial portion of Scotland.
He began a war with France, and through outmaneuvering, won the \textbf{Battle of Cr\'ecy} in 1346.
The English lost at the disastrous naval Battle of La Rochelle.

Edward's sons were Edward the Black Prince and John of Gaunt.
The Black Prince won the Battle of Poitiers, and he was the father of Richard~II, who would be king.
The prince's death, however, prevented him from ever becoming king himself.

When \textbf{Richard~II} came to power, he was only ten years old.
He faced a \textbf{Peasants' Revolt}, led by Wat Tyler, for taxes that were too high.
The rebellion was stopped when the Lord Mayor William Walworth killed Tyler.

\textbf{Henry IV} of Bolingbroke claimed that he was descended through Edmund Crouchback,
who may have been the oldest son of Henry~III\@.
Not a lot of people really believed him.
He decided to go on the offensive and took the crown by force, and imprisoned Richard.
Richard would later die for some (unknown) reason.
Henry restarted the Hundred Years War, and he stopped rebellions
in Scotland (by the Percys) and in Wales (by Owain Glyndwr).
Henry~IV died in 1413.

He was succeeded by \textbf{Henry~V}, a ruthless king who liked war.
He invaded France, taking advantage of Charles~VI of France's illness,
and took Caen, Rouen, and went to Calais.
He participated in the Siege of Harfleur, and after that,
his army, outnumbered three to one, won the \textbf{Battle of Agincourt} on St.\ Crispin's Day, 1415.
At Agincourt, the Duke of Brabant showed up late but he was killed quickly.
The French were led by Charles d'Albert, and they stole a crown from the English during the battle
before d'Albert led them straight into a line of pikes that the English had driven into the ground for defense.
The English won in large part because of their Welsh longbows.
The battle led to the Treaty of Troyes.
He took back Normandy, and he married Catherine of Valois.
Henry~V died in 1422, and was succeeded by his infant son.

\subsection*{Wars of the Roses}

Henry~IV, Henry~V, Henry~VI were members of the House of Lancaster,
which would come to fight the House of York in the Wars of the Roses,
so named because of the roses that were the insignia of the warring houses.

\textbf{Henry~VI} started his reign young, but even when he grew up, he would be a weak king.
Many people would rise up against him, including Richard Neville, Earl of Warwick, ``Kingmaker''.
Warwick's cousin was a naval commander --- Thomas Neville, the Bastard of Fauconberg.
He had a mental breakdown, and Richard, Duke of York, was named regent in his stead.

When Henry became sane again, Richard of York and the Nevilles decisively defeated
him and the House of Lancaster at the \textbf{First Battle of St.\ Albans}.
It wasn't a big battle, and few men died, but some of those that died were fairly important people.
However, popular support wasn't really with York, and they fled.
When they returned, they won the Battle of Northampton, and captured Henry.
They reached an agreement called the Act of Accord, according to which Henry would stay king,
and when he died, York would take the crown.

\textbf{Edward~IV} of York took the throne after he won at the Battle of Towton.
However, his marriage would lead to him being deposed and Henry~VI restored to the throne.
Edward returned, and his victory as the Battle of Tewkesbury
and subsequent murder of Henry wiped out the Lancastrians.

At this point, the House of York seemed to be fairly solidly set as the new ruling family.
When Edward~IV died, he was succeeded by his brother Richard,
who took the throne and became \textbf{Richard~III}, under the statute \textit{Titulus Regius}.
Richard had supposedly imprisoned Edward's children Edward~V and his brother,
Richard had married Anne Neville after an argument with George, Duke of Clarence, and Edward~IV\@.
Edward's sons were never seen again,
leading to the legend of the ``Princes in the Tower''.

By 1483, Lady Margaret Beaufort of the House of Tudor was actively promoting her son, Henry,
as a better alternative to the rule of Richard~III\@.
Henry tried to take England, but his plans fell apart and he ran away to France,
and he lived in exile in Brittany for a year.
He put together an army, getting help from the Welsh by using a dragon flag to show his ancestry.
In 1485, he landed at Milford Haven to try again,
and he realized his best chance was to attack Richard and defeat him quickly.

Henry met Richard at the \textbf{Battle of Bosworth Field}, near Ambion Hill.
The Yorkist army deployed on the hilltop, so that Richard could see the whole area.
Henry's army had very few Englishmen in it --- many were Scottish and French.
His army was commanded by the Earl of Oxford, who was assisted by the Lords Stanley
(who initially hung back but then decided to join when they felt like it).
When the battle began, Henry was maneuvering around trying to get into a better position.
The first charge was led by John Howard, Duke of Norfolk.
At one point, when Henry rode to join the Stanleys, Richard decided to charge and kill him quickly.
He killed Henry's standard-bearer Sir William Brandon, and unhorsed John Cheyne,
but Henry was still very well guarded.
Richard had come close to Henry, but his group was surrounded gradually,
and Henry's Welshmen hacked him to death.
After the battle, Richard was buried at Grey Friars monastery.
Richard's circlet was brought to Crown Hill,
where Henry Tudor was crowned, and he became \textbf{Henry~VII}\@.

Henry VII was the first of the Tudors to rule England,
and we'll see more of him and his wild and crazy descendants later.

\section{France: The Capetians}

When we last looked at France, Charlemagne was in power.
In 814, he died, and his heirs were incompetent, so the empire started to break.
In 843, the grandsons of Charlemagne (sons of Louis the Pius) signed the \textbf{Treaty of Verdun}.
The treaty split the Carolingian empire between Louis the German, Charles the Bald, and Lothair.
Louis the German received the east, which would eventually become the kingdom of Germany,
Charles the Bald would rule over modern-day France,
and Lothair would get the Low Countries, including Italy.

\subsection*{Capetians}

Viking advances and infighting amongst French royals would result in the downfall of the Carolingians.
In 987, an assembly in Reims elected \textbf{Hugh Capet} to the position of King of the Franks.
The dynasty he founded, the \textbf{Capetians}, would rule France for over eight centuries,
along with its friends the Valois and Bourbons.
Hugh's son, Robert the Pious, was elected king before Hugh died, in order to ensure succession.

In the land of the Franks, \textbf{Philip~I} came to power in 1060 and ruled until 1108.
His reign included the First Crusade, which I'll talk about in another section soon.
This was also the time that William the Conqueror was invading England.

Starting with the reign of \textbf{Louis~VI}, royal authority in France became more accepted.
Louis liked war, and he was assisted by Abbot Suger regarding non-military things.
He was succeeded by his son \textbf{Louis~VII}, who people tended to like.
He married Eleanor of Aquitaine, and also saw a crusade.
However, his marriage was annulled, and Eleanor would marry Henry~II of England.

\textbf{Philip~II Augustus} recovered much of Normandy from England (under John's rule at the time);
recall the Battle of Bouvines, England was mostly ousted from France.
He founded the Sorbonne, and he made Paris a more intellectual city.

France would become even more centralized under \textbf{Louis IX} (1226--1270).
His mother was was Blanche of Castile.
He fought Henry~III in the Saintonge War, in which Henry tried to take back England's old territory.
He defeated Hugh of Lusignan and Henry~III of England at the Battle of Taillebourg.
On completion of the war, he supported new forms of art, and he commissioned the Saint-Chapelle, a Gothic building.
He participated in the Seventh and Eighth Crusades.
Louis's nephew was \textbf{Robert of Artois}, an experienced soldier who participated in the Aragonese Crusade.
He also won the Battle of Furnes against the Flemings,
and in 1302, he went into Flanders and won the Battle of the Golden Spurs.
On his death in 1270, he became the only canonized king of France, St.\ Louis.

Philip~III and \textbf{Philip~IV} (Philip the Fair) followed Louis~IX\@.
Philip~IV tried to invade Aragon, and failed miserably because of an epidemic.
The latter elevated his monarchy to a glorious position.
He focused on the north of France, and he forced England out of Gascony.
He got into an argument with Pope Boniface~VIII for taxing the clergy in 1296.
Eventually he would put his own pope into the papacy that got moved to Avignon.
He also destroyed the Knights Templar; we'll deal with them more during the Crusades section.
Philip~IV died in 1314, and his son Louis~X ruled shortly before he died of sickness.

In 1218, on Sicily, the bad rule of Charles~I of Anjou forced a rebellion, in favor of Peter~III of Aragon.
This revolt was known as the \textbf{Sicilian Vespers}.
A phase of the War of the Sicilian Vespers was the Aragonese Crusade.
After the war, the Sicilians won and Anjou rule was kicked off the island.

The throne then passed to his brother \textbf{Philip~V}.
He made peace with Flanders, and he continued to fight with Edward~II of England regarding Gascony.
He put down the Pastoreux uprising when his hand was forced by Pope John~XXII\@.
He was succeeded by his brother, Charles~IV\@.

\subsection*{Hundred Years' War}

When Charles died in 1328, it signaled the end of the main line of the Capetians.
Thus, the crown passed down to Philip~VI, son of Charles of Valois.
The standing enmity between England and France turned into the Hundred Years' War, discussed above.
Just as England experienced a peasant revolt in 1381, France experienced the \textbf{Jacquerie} in 1358.

France was pushed back in the beginning of the war (1337--1360), and they began to push back later.
However, when Henry~V of England won at Agincourt, France fell into chaos.
Seven years later, Henry~VI was declared king in Paris by the Treaty of Troyes, and the Valois were pushed back.

In 1429, \textbf{Joan of Arc} was seemingly given instructions from Saints Michael, Catherine, and Margaret.
She went and met the Dauphin Charles~VII at Orleans, and he was impressed by her.
She then went to the siege of Orleans, and she helped lift the siege (claiming she had seen a sign),
leading to her being known as ``the Maid of Orleans''.
The victory helped crown \textbf{Charles~VII} king in Rheims.
She went on to rout John Fastolf at the Battle of Patay, aided by La Hire.
She wielded an old sword that was found behind St.\ Catherine's altar.
However, she was captured by Burgundians at the Compiegne, due to a mistimed drawbridge.
Then, she was sold to the English John of Luxembourg, tried by Bishop Chaucon, and burned at the stake for witchcraft.
She would be sainted in 1920.

The French went on to drive England back out of France.
Other notable battles during the Hundred Years' War included:
Bauge, where the French won;
Poitiers, where the British crushed the French;
and Formigny, where the French won.
The last battle of the war is considered to have been the Battle of Castillon.

\subsection*{House of Valois}

The first king of France of the House of Valois was \textbf{Francis~I}, son of Charles of Angoul\^eme.
He succeeded Louis~XII, who had died without an heir.
Francis fought in the \textbf{War of the League of Cambrai} during the Italian Wars.
During the final stage of that war, Francis routed the Papal States and the Old Swiss Confederacy
at the Battle of Marignano, defeating Pope Leo~X.
Pope Clement~VII became a big ally of Francis.

\section{Germany: Holy Roman Empire}

The Germans were slowly being united under such leaders as Henry the Fowler.
In 936, Henry's son, \textbf{Otto I, the Great} was crowned as king at Aachen.
He led Burchard~III of Swabia and Conrad the Red into battle
against the Magyars at the Battle of Lechfeld in 955.
The Magyars had also lost the Battle of Pressburg.

After receiving a plea for help, led the First Italian Expedition to defend Adelaide, queen of Italy,
who was imprisoned on Lake Garda by the usurper Berengar~II,
Otto married Adelaide and took control of Italy, becoming King of the Lombards.
In 962, he was declared Holy Roman Emperor by Pope John XII\@.

Later, \textbf{Henry~IV} (1050--1106) would begin a clash with the Church.
He disagreed with \textbf{Pope Gregory VII} over appointments to Catholic offices,
in what came to be known as the lay \textbf{Investiture Controversy}.
Eventually, after the pope decided to excommunicate him, Henry gave in near the snows of Canossa in 1077.
This continued until the \textbf{Concordat of Worms} in 1122 resulted in at least a temporarily cease of dispute.
The Investiture Controversy resulted in weakening the church.

At this point, trading in the Baltic Sea increased, and the \textbf{Hanseatic League} was formed.
The League was a set of major trading towns, led by \textbf{L\"ubeck}, and they would dominate trade across the coast.
The cities also included Cologne, Hamburg, Bremen, and Visby.
The League established trading posts, called \textit{kontors},
and it forced the Treaty of Stralsund on Valdemar~IV of Denmark.

In 1226, the \textbf{Teutonic Knights} began their conquest of Prussia.
Prussia would evolve from the set of towns that the Knights established after their war.
Their greatest foes were probably a society called the Lizard League.
They lost the Thirteen Years War, and their leaders included Hermann Von Salza.
The Knights fought against the Cumans for Andrew~II of Hungary.
They would remain in power in Prussia until the Battle of Tannenberg in 1410.

Eventually, the Hohenstaufen dynasty under \textbf{Frederick Barbarossa} (1152--1190)
would grant Bavaria to Henry the Lion, duke of Saxony.
Austria became a separate entity, and Frederick tried to take Italy.

Starting in 1438, the \textbf{Habsburgs} from the south and east parts of the empire
would maintain a grip on the position of the Holy Roman Emperor.
We'll see a lot more of the Habsburgs as time goes on.

\section{Crusades}

The \textbf{Crusades} were religious military campaigns with the goal of restoring Christianity to places that had lost it.

The \textbf{First Crusade} (1096--1099) was declared by \textbf{Pope Urban~II} in 1095
at the \textbf{Council of Clermont}.
The main goal originally was to help \textbf{Alexius~I Komnenos}, the Byzantine emperor at the time.
At the Council of Piacenza, he asked for help repelling the Seljuk Turks in Anatolia.
However, it quickly became about regaining Jerusalem and the Holy Land,
and the crusaders expected help from Alexius.
During the crusade, the Byzantine general Tatikios won a victory at Dorylaeum.
Walter the Penniless (Walter Sans Avoir) and Peter the Hermit were defeated at Nicea.
Other leaders in the conflict included Raymond of Toulouse, Bohemond, and Godfrey of Bouillon.
It was ultimately successful in taking back the Holy Land.
The prelude to the First Crusade was known as the People's Crusade.

The \textbf{Second Crusade} (1145--1149) was started as a reaction to Zengi's capture of Edessa.
It was encouraged by Pope Eugenius~III\@, the bull \textit{quantum praedecessores},
and it was championed by the preacher St.\ Bernard of Clairvaux (called in by the Archbishop of Mainz).
The major generals in the crusade were Louis~VII of France and Conrad~III of Germany.
The crusade ended when the siege of Damascus failed, and unlike the First Crusade,
it failed to take back the Holy Land.

The \textbf{Third Crusade} (1189--1192) was another attempt to take the Holy Land.
It was also known as the Kings Crusade,
as Philip~II of France, Frederick Barbarossa, and Richard the Lionheart all led forces.
Frederick died when he decided to ride his horse across the Saleph River,
but the horse couldn't handle it, and his armor was too heavy to swim with.
The major leader for the Muslim side was \textbf{Saladin}.
Saladin had previously defeated the Assassins at Masyaf
(he notably covered the ground around his tent with chalk so he could see their footprints).
The assassins were led by the ``old man of the mountain'',
and their opiate-abusing ways would be stopped by Hulagu Khan.
Richard and Saladin made peace at the Treaty of Ramla after the Battle of Arsuf,
mostly because they were tired of fighting and they didn't feel like continuing.
The crusade ended mostly successfully, capturing Acre and Jaffa, but failed to take Jerusalem.
On the way home, Richard's disguises failed, and he was captured by Leopold~V of Austria,
and was kept in various states of captivity for a couple years.

The \textbf{Fourth Crusade} (1202--1204) tried to take Jerusalem by way of Egypt.
Pope \textbf{Innocent~III} called for the crusade, but nobody really cared.
So Theobald~III of Champagne held a tournament where various people, such as Fulk of Neuilly,
attended and took the cross, to go on the crusade.
Before the crusade started, the Venetians had been in deep debt.
The Doge, Enrico Dandolo, wanted to pay it off, and Alexius~IV, the Byzantine prince, offered to do so.
At a meeting between Boniface of Montferrat and Philip of Swabia,
they agreed to take down the current Byzantine emperor, Alexius~III\@.
Because of this, the crusade ended up going to Constantinople.
The crusaders attacked Galata Tower to break the chain across the Golden Horn.
Eventually, they were able to enter the city, and they sacked it for three days.
At the end, Alexius~III had been toppled, and Alexius~IV was in power.
Only about a tenth of the crusaders ever made it to the Holy Land,
so the very Christian intentions of the crusade simply turned into the very much non-Christian sacking of Constantinople.

The \textbf{Albigensian Crusade} (1209--1229), also called the Cathar Crusade, was a Crusade that didn't go east.
Pope Innocent~III called for this one too.
It tried to eliminate Catharism in Languedoc in the south of France.
The Cathars were a sect of Christians that became known as the Albigensians.
This crusade ended in far less Cathars being alive, and the French crown grew in power.
The crusade ended with a Treaty of Paris.

The \textbf{Children's Crusade} (1212) which tried to take Jerusalem peacefully.
It failed pretty badly because it recruited children, and some of them were sold into slavery.
Notable leaders included Stephen of Cloyes, and the crusade was sent home by Philip~II after reaching Saint-Denis.

The \textbf{Fifth Crusade} (1213--1221) tried to take Jerusalem through Egypt (again).
They lost to the Ayyubid state in Egypt and failed.

The \textbf{Sixth Crusade} (1228) tried to take Jerusalem (do you see a pattern?).
Frederick~II, Holy Roman Emperor, delayed the start of the crusade;
when they did set out, there was an outbreak of malaria.
Very little fighting actually occurred, and the crusaders won a diplomatic victory,
and Pope Gregory~IX, an enemy of Frederick, didn't like that Frederick declared himself king in Jerusalem.

During the \textbf{Seventh Crusade} (1248--1254) and \textbf{Eighth Crusade} (1270), Louis~IX of France participated.
The crusaders didn't take Jerusalem.

The last crusade was the \textbf{Ninth Crusade}.

\section{Khans \& Conquerors}

\subsection*{Mongol Empire}

The Mongolian plateau was home to the Khereid, Khamag Mongol, Naiman, Mergid, and Tatar tribal confederations.
The Jin dynasty emperors worked to keep them feuding among themselves.
During the 13th century, the harsh, dry, and cold steppes of central Asia would have very mild, wet conditions,
which would contribute to a rise of Mongol military strength.
One of the sons of the Mongol chieftain Yesugei was Temujin.
By 1206, Temujin was crowned as the Khaghan of the Mongol Nation,
and he became known as \textbf{Genghis Khan}.

According to legend, Genghis had been born with a clot of blood in his hand.
Genghis Khan married Borte,
and he appointed his brother Shigi-Khuthugh to the position of supreme judge
to oversee the Yassa law code that he implemented.
He also was tolerant of religious freedom and didn't tax those who couldn't take it.
He encouraged literacy and adopted the use of the Uyghur script.
Genghis's life was documented by Rashid al-Din and Juvaini.

Genghis united the warring Mongol tribes into a single mighty force.
His army was divided into arbans, zuuns, Mingghans, and tumens.
He forbade looting of the enemy without permission, and he set up a system to share spoils of war;
he also set up a practice of holding victory feasts on a platform on top of captured nobles.
He expanded the Mongol empire, and pushed into central Asia, with the help of his general Subotai.
In 1220, he sacked Samarkand.
Before he died fighting the Tangut peoples in 1227,
he named his son \textbf{Ogedei}, in place of his other son Tolui, heir to his empire.

During Ogedei's reign, a grandson of Genghis, \textbf{Batu Khan},
overran the Bulgars and other peoples of southern Russia.
When Ogedei died in 1241, his widow took over control of his empire.
She was able to control most of it.
But, Batu's \textbf{Golden Horde} did not show up to the
\textit{kurultai} that was held to choose the next khan of the Mongols,
and the empire was thrown into imbalance.

The Golden Horde was composed of the Blue and White Hordes,
and they were named for the colors of tents that they used.
There was much more turmoil between the khans that I won't get into right now,
but other leaders of the Golden Horde included Nogai, Uzbek, Berke, Jani Beg, and Hulagu Khan.
Also recall that Kublai Khan invaded China and set up the Yuan dynasty in the mid-late 13th century.

\subsection*{Tamerlane}

Sometime in the 1320s, a man named Timur the Lame, or \textbf{Tamerlane}, was born.
He founded the Timurid dynasty in Central Asia, and he referred to himself as the \textit{Sword of Islam}.
He led his army around Asia, wrecking and plundering all towns he saw,
and creating pyramids of skulls outside of places he sacked.
His capital was at Samarkand, he put down a tax revolt in Isfahan,
and he built the White Palace in his birthplace of Shahrisabz.
At Ankara, he captured Bayezid the Thunderbolt,
and he defeated Mahmud Tughluq and sacked Delhi.
Tokhtamysh of the Golden Horde, one of Tamerlane's biggest rivals, lost Sarai to him at the Battle of Kur River.
Tamerlane's other victories include the Battle of the Terek River and the Battle of Ankara.

Tamerlane's life was chronicled by Ruy Gonzales de Clavijo, and he was buried in a tomb called Gur-e-Amir,
which would be opened before Operation Barbarossa (which we'll discuss when we get to World War II).
Tamerlane was be succeeded by his son, Shah Rukh.
His grandson was Ulugh Beg, and one of his descendants was Babur, who would found the Mughal Empire.
