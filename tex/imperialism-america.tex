\chapter{America as a World Power}

\section{A Gilded Age}

Mark Twain coined the term \textbf{Gilded Age} to talk about the fact that there were some pretty bad social problems
in the late 19th century, but they were masked by economic growth.

\subsection*{Rutherford Hayes}

The Election of 1876 was one of the most controversial elections ever.
Republican \textbf{Rutherford B. Hayes} was up against Democrat Samuel Tilden, who won the popular vote.
Hayes was down by 19 electoral votes, but Florida, Louisiana, South Carolina, and Oregon were in dispute.
An elector in Oregon was declared illegal and replaced.
As a result, the \textbf{Compromise of 1877} was created, giving all 20 remaining votes to Hayes.
In return, Republicans agreed to end Reconstruction and take the troops out of the South.
Power in the South went to eh Democratic \textbf{Redeemers},
and Hayes was sometimes called ``His Fradulency''.

Hayes's First Lady was \textbf{Lucy Hayes},
now known as ``Lemonade Lucy'' because she was a big supporter of the \textbf{temperance movement} to ban alcohol.
The president banned alcohol from the White House while in office.
Hayes's Secretary of the Interior was Carl Schurz, who helped determine Indian policy,
and his Attorney General was Carl Devens.
His Treasury Secretary was John Sherman,
and his Vice President was William Wheeler
(Hayes apparently asked ``Who is Wheeler?'' when told about his running mate).

Hayes vetoed the \textbf{Bland-Allison Act},
in which the Treasury would have been required to put \$2 million in silver into circulation.
Congress passed the act over his veto anyway.
He also sent Winfield Scott Hancock to Pittsburgh and Baltimore to deal with the Great Railroad Strike.
Hayes removed Alonzo Cornell and Chester Arthur from power in the Port of New York,
an action that was opposed by his nemesis Roscoe Conkling, leader of the Stalwarts.

Hayes also helped arbitrate the end of the War of the Triple Alliance in South America.
He got Paraguay some territory.

\subsection*{Garfield and Arthur}

Hayes didn't run for reelection, and Republican \textbf{James Garfield} took office,
having gotten the nomination as a result of a compromise with the Stalwarts and Half-Breeds.
Garfield had previously won Ex Parte Milligan,
and his rags-to-riches story was written by \textbf{Horatio Alger}.
While in office, Garfield had Postmaster General Thomas James investigate the Star Route frauds.

In 1881, \textbf{Charles Guiteau}, a Stalwart, asked Garfield for a consulship in Paris.
He had printed lots of copies of a speech about Winfield Scott Hancock called ``Garfield vs. Hancock''.
Garfield refused Guiteau and Guiteau got angry.
Yelling ``I am a Stalwart!'', Guiteau shot Garfield.
In an attempt to help Garfield, \textbf{Alexander Graham Bell} brought a metal detector to find bullets,
although at first it was only detecting springs in the bed that Garfield was on.
The event resulted in the Pendleton Civil Service Act being passed.
Garfield died and Guiteau was convicted of his murder.

Garfield's Vice President \textbf{Chester A. Arthur} took office.
Arthur had previously been supported by Stalwart Roscoe Conkling (during the Hayes administration).
He signed the Pendleton Act, and he rebuilt the navy.
During Arthur's administration, the Chinese Exclusion Act was passed,
and he signed a compromise tariff called the Mongrel Tariff.
He also signed the Edmunds Act, banning polygamists from taking public office.

\subsection*{Cleveland, Harrison, and Cleveland}

\section{Spanish-American War}

\subsection*{William McKinley}

\subsection*{Course of War}
% T Roosevelt
% USS Maine
% De Lome Letter
% San Juan Hill
% Manila Bay
% splendid little war

\section{Progressivism and Imperialism}

\subsection*{Theodore Roosevelt}
% Assistant Secretary of the Navy
% Panama Canal
% Corollary Doctrine

\subsection*{William H.\ Taft}
% Progressivism
% Dollar Diplomacy

\subsection*{Mexican Revolution}
% Zapata
% Pancho Villa
% Bandit War

\subsection*{Panama Canal}
% Hay-Bunau-Varilla

