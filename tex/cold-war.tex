\chapter{Cold War}

\epigraph{%
  From Stettin in the Baltic to Trieste in the Adriatic, an iron curtain has descended across the Continent.
}{Winston Churchill}

\section{Germany Divided}

The Second World War had killed almost 10\% of the German population.
The place was in shambles.
At Potsdam, the country had been divided into four zones by the Allies.
The country didn't regain independence until 1949.

In East Germany, Soviets ruled with an iron fist, employing the \textbf{Stasi} as the official state security.
From 1950 to 1971, the party boss in the area was \textbf{Walter Ulbricht}.
In 1953, Ulbricht's government had to put down an uprising using Soviet troops and tanks.
The General Secretary after Ulbricht was \textbf{Erich Honecker} (1971--1989).

Berlin, deep in the heart of East Germany, was itself divided into separate occupation zones.
Stalin then decided to blockade West Berlin, disallowing any Allied supplies from reaching the city.
The Soviets introduced the Ostmark in their zone to combat the announcement of the new Deutsche Mark.
As a result, Lucius Clay, deputy governor of Germany, put Operation Vittles into effect.
The \textbf{Berlin Airlift}, as it was called,
aimed to fly two million tons of supplies into the city and drop them to the people below.
Construction for the bases occurred at places near Tempelhof and Lake Tegel.
Albert Wedemeyer, a general who had commanded the airlift from India over the Himalayan Hump, helped the effort.
The British contribution was codenamed Operation Plainfare,
and the Australian operation was designated Operation Pelican.

By 1961, 2.6 million people had fled from East Germany,
and Ulbricht created the \textbf{Berlin Wall} to try and stop them.
The wall was crossable in few locations, including Checkpoint Charlie,
and abandoned subway stops past the wall were called ``ghost stations''.
During the Berlin Wall crisis in 1961, Lucius Clay ordered tanks to a standoff to back up Albert Hemsing.

\section{Cold War Begins}

\subsection*{Truman and his Doctrine}

Following the victory in World War~II, Truman and friends helped set up the \textbf{United Nations}.
He also issued the \textbf{Truman Doctrine}, giving aid to Turkey and Greece, in an attempt to contain communism.
Truman's policy of containment was inspired partly by \textbf{George F.\ Kennan} (Mr.\ X)
and his writings talking about \textbf{containment} of communism.
Kennan sent the \textbf{Long Telegram} from Soviet Russia,
claiming that the Soviets were expansionist and needed to be contained in places that mattered to the US\@.

In order to help rebuild Europe, Secretary of State \textbf{George Marshall} came up with a strategy.
The \textbf{Marshall Plan} consisted of using \$13 billion to bring European economies back on track,
and William Clayton, Kennan, and Marshall unveiled the plan at a graduation speech at Harvard in 1947.
To counter the Marshall Plan, the Soviets put their own Molotov Plan into effect to help Eastern European countries.

\subsection*{NATO \& the Warsaw Pact}

In 1949, the former Allies formed the \textbf{North Atlantic Treaty Organization (NATO)}.
Headquartered in Brussels, NATO's goal was to stop the growing Soviet threat.
They set up yearly readiness exercises named after Able Archer and Dean Rusk.

In response, the Soviets set up the \textbf{Warsaw Pact} in 1955.
Included in the organization were China and North Korea,
and during Zapad 81, they would carry out the largest military exercise ever.

\section{Korean War}

Korea had been ruled by Japan since 1910 until the later parts of World War~II\@.
Then, the Soviets took control of the peninsula north of the \textbf{38th parallel},
and US forces occupied the south, and Japan surrendered, packed up, and left.

Two separate governments were set up.
The south elected \textbf{Syngman Rhee} president in July 1948.
Rhee had converted to Christianity in High School, and he had spent time in America.
He opposed Communism, and as soon as he took power,
he started instituting laws against political dissent, smashing his leftist opponents.

In Communist North Korea, a new regime was established under \textbf{Kim Il-Sung}.
His symbol was a blue orchid that had been given to him by Sukarno, and he made it the official flower of the country.
He set up the kwan li so labor camps,
and he introduced the Juche (self-reliance) political ideology, displacing Stalin and Mao.
In 1968, his government would capture \textbf{USS Pueblo}, commanded by Captain Lloyd Bucher.

Now, both Rhee and Kim Il Sung wanted to unite Korea under their respective governments.
Stalin set up the North Koreans with weapons,
and they crossed the 38th parallel on June 25, 1950, fighting on the Ongjin peninsula.
They came south, forcing Rhee to leave Seoul and drastically reducing the headcount of South Korean troops.

At this point, the Truman administration wasn't really well prepared for military action in Korea.
There were worries as to what the Soviets would do if the US intervened in Korea.
Truman wrote:
\begin{quote}
  There was complete,
  almost unspoken acceptance on the part of everyone
  that whatever had to be done to meet this aggression had to be done.
  There was no suggestion from anyone that either the United Nations or the United States could back away from it.
\end{quote}
There was fear of a chain reaction allowing the Soviets to disregard the UN
and create Communist aggression in any number of other places across the world.
On June 27, after the UN Security Council ruled that the North Korean invasion was bad,
Truman sent the 8th Army into the peninsula to help the South Koreans.

The first significant American engagement came at the \textbf{Battle of Osan}.
The battle was unsuccessful,
and the North Korean KPA pushed the South Korean ROK and the US army back to \textbf{Pusan}.
Battles around this time included Onjong.
Kim Il-sung thought he would end the war by the end of August.

To relieve the Pusan perimeter, General MacArthur ordered the planning of a landing at \textbf{Inchon}.
The battle began with USS \textit{Mansfield}, USS \textit{Swenson}, and USS \textit{Collett} bombarding a fortress.
The X Corps then landed at Green, Blue, and Red beaches, and captured the Kimpo airstrip,
allowing for a breakout from the Pusan perimeter.
By the end of September, Seoul was back in South Korean hands.

Continuing their momentum, MacArthur and the ROK went up western Korea and took Pyongyang in October.
MacArthur then thought it was going to be a good idea to keep going straight on into China, but Truman disagreed.
When China entered the war, it led to major Allied losses,
and MacArthur was seriously considering using nuclear weapons to end the war.
MacArthur's Home-by-Christmas Offensive was met with the Chinese Second Phase Offensive,
ambushing and pushing the allied forces back from the Yalu River.

While MacArthur wanted complete surrender from the North Koreans,
Truman wasn't as optimistic about the land war in Asia (he realized it was a classic blunder).
So, Truman relieved MacArthur of his command, and put \textbf{General Matthew Ridgway} in charge.

The rest of the Korean War would mostly be spent in stalemate.
Combat continued during armistice negotiations, which would take a very long time to go through.
Negotiations were held at Kaesong and at Panmunjom.
Battles during the stalemate included:
Pork Chop Hill,
Heartbreak Ridge,
White Horse Mountain,
Hill Eerie,
and the siege of Outpost Harry.
An armistice was agreed on in July 1953, setting up a \textbf{Demilitarized Zone (DMZ)} at the 38th parallel.
The area is still patrolled by members of both sides, and clashes occur sometimes.

\section{Soviet Union After Stalin}

\subsection*{Khrushchev's USSR}
% TODO Gary Powers

In the US, the transfer of leadership from Truman to Eisenhower resulted in policy not changing much,
even though the transition crossed party lines.
The change from Stalin to his successor, \textbf{Nikita Khrushchev}, however, was big.
Stalin died in 1953, and the Communist Party named Khrushchev, his right hand man, First Secretary.

During some collective leadership, Khrushchev consolidated power.
He gave his \textbf{secret speech}, ``On the Personality Cult and Its Consequences'',
to the 20th Party Congress of the Communist Party,
denouncing Stalin and the crimes that Stalin had committed:
\begin{quote}
  We have to consider seriously and analyze correctly [the crimes of the Stalin era]
  in order that we may preclude any possibility of a repetition in any form whatever
  of what took place during the life of Stalin,
  who absolutely did not tolerate collegiality in leadership and in work,
  and who practiced brutal violence, not only toward everything which opposed him,
  but also toward that which seemed to his capricious and despotic character, contrary to his concepts.
\end{quote}
Even though his speech was supposed to be secret, it ended up getting leaked.
Khrushchev ended up being named Premier of the USSR in 1958.

While in power, Khrushchev used the Virgin Lands Campaign to try and improve agricultural productivity.
He oversaw some liberalization and de-Stalinization, in a period known as the Khrushchev Thaw.
At one point, Lazar Kaganovich and Vyacheslav Molotov, the Anti-Party group, tried to overthrow him
and replace him with his Prime Minister, Nikolai Bulganin.

\subsection*{The Cuban Situation}
% Cuban Missile Crisis

To contextualize this part,
we need to go back a bit and take a look at where Cuba was at this point in time, and how it got there.
In 1952, \textbf{Fulgencio Batista} became dictator of Cuba.
Batista had been elected President after overthrowing Carlos Prio Socarras in 1940.
Around this time,
a young lawyer named \textbf{Fidel Castro} decided that he didn't like Batista and wanted to depose him.
He teamed up with his brother \textbf{Raul Castro} and led an attack on Moncada Barracks in July 1953.
They failed badly and ended up in prison in Model Prison on the Isle of Pines.
In his own defense, Castro gave the ``History Will Absolve Me'' speech in court:
\begin{quote}
  I know that imprisonment will be harder for me than it has ever been for anyone,
  filled with cowardly threats and hideous cruelty.
  But I do not fear prison,
  as I do not fear the fury of the miserable tyrant who took the lives of 70 of my comrades.
  Condemn me.
  It does not matter.
  \textbf{History will absolve me.}
\end{quote}

When they got out, the Castros went to Mexico, and met \textbf{Ernesto ``Che'' Guevara}.
While they were there, they set up the \textbf{26th of July Movement} and prepared to take down Batista again,
named after the date of the failed Moncada Barracks attack.

In December 1956, Fidel and less than a hundred men took the yacht \textit{Granma}
(incidentally, the newspaper of the Cuban Communist Party would be named \textit{Granma} in its honor)
to Cuba, landing with the intention of taking the island.
They were promptly decimated by Batista's men, but Castro escaped into the Sierra Maestra mountains.
From there, he and Guevara ran a guerrilla campaign against Batista.
Every time Batista tried to take down the rebellion, he failed.
At this point, the US imposed trade restrictions on Cuba and told Batista to get out of town,
and Batista left the island in January 1959, allowing Castro to take over.

Of course, the Americans weren't particularly happy with the fact that Cuba (only 90 miles from Key West)
was under the dictatorial control of a Communist.
So, the US severed diplomatic ties with Castro's Cuba.
In April 1961, the CIA executed a plan to take down the Castro regime.
The idea for the \textbf{Bay of Pigs invasion} (Operation Zapata)
had been created under the Eisenhower administration,
but it was executed four months into Kennedy's administration (more on Kennedy later).

The CIA idea was to arm 1,400 Cuban exiles known as Brigade 2506
and empower them to establish ``a new government with which the United States can live in peace''.
On April 15, the brigade got on a boat from Guatemala and landed on Playa Gir\'on, initially making some gains.
Then, everything proceeded to go south.
During Operation Falcon, the drop of 150 paratroopers failed to take their objective,
and retreat into the Escambray Mountains was impossible.
So, most of the brigade was captured and interrogated.

Kennedy took full responsibility for the failure,
and the invasion was a contributing factor to the dismissal of Allen Dulles as director of the CIA in late 1961.
Following the Bay of Pigs, the Kennedy administration began Operation Mongoose,
a more covert sabotage campaign against Castro, making Castro hate the US even more.

Castro was fairly closely allied with Khrushchev at this point, them both being Communists.
After the failure of the Bay of Pigs invasion,
Khrushchev agreed to Castro's request for nuclear missiles on the island.
Khrushchev was pleased with the arrangement,
especially in light of the fact that the US had missiles in Italy and Turkey within range of Moscow.

This was October 1962, so a US election was underway,
The US confirmed missile preparations when Richard Heyser flew a U-2 plane over Cuba,
taking pictures of new bases under construction at San Cristobal.
Kennedy's administration was criticized about the nuclear missiles sitting 90 miles off the Florida coast.
At this point, the US blockaded Cuba and demanded that all the existing missiles there should be dismantled.

Negotiations in the United Nations Security Council were mediated by Secretary General U Thant,
and Ambassador \textbf{Adlai Stevenson} discussed the situation in Cuba with Soviet Ambassador Valerian Zorin:
\begin{quote}
  Do you, Ambassador Zorin, deny that the USSR has placed
  and is placing medium and intermediate range missiles and sites in Cuba?
  Yes or no?
  Don't wait for the translation: yes or no?
  \textbf{I am prepared to wait for my answer until hell freezes over, if that's your decision.}
\end{quote}
Stevenson then proceeded to present a series of pictures taken by the U-2 to the Security Council.
During the Cuban Missile Crisis, Kennedy put together EXCOMM,
a group of National Security Council members and friends, to help advise him during the situation.
The crisis was resolved by the Hotline Agreement,
in which the US agreed to dismantle the Jupiter Missiles threatening Moscow,
and a new hotline was set up between Kennedy's and Khrushchev's offices.

\subsection*{Brezhnev in Power}

When the Soviets got tired of Khrushchev in 1964,
they ousted him and placed \textbf{Leonid Brezhnev} in his place as First Secretary.
He'd previously been in charge of the Virgin Lands Campaign, and now he took full power.
Brezhnev would consolidate power, setting up his Brezhnev Doctrine,
allowing intervention if socialist countries decided to turn to capitalism.
A prime example of the use of his doctrine is the invasion of Czechoslovakia during Prague Spring
(more on the Czech situation later).
He also ordered the Soviet invasion of Afghanistan.

When Brezhnev took over, the Era of Stagnation began.
The economy, politics, and social change all more or less came to a standstill,
and they would stay that way until Gorbachev took over in the mid-80s.

\section{Vietnam War}

Before we get to how this proxy war was fought, let's first take a look at history in southeast Asia.
France had begun to conquer Indochina in the 1850s, and in 1888,
\textbf{French Indochina} was colonized in present-day Cambodia and Vietnam.
Vietnamese forces tried to oppose French rule for quite a while,
but the most successful were the \textbf{Viet Minh}, founded in 1941,
and controlled by the Communist Party.
While France was occupied by Germany, the Indochinese government sided with the Vichy regime.
The Viet Minh opposed French occupation, and they fought the Japanese who tried to take over as well.

When Japan lost World War~II, Japanese forces in Vietnam were inactive,
and Viet Minh forces took over the country in the \textbf{August Revolution}.
\textbf{Ho Chi Minh} led meetings to create the independent Democratic Republic of Vietnam in Hanoi.
At this point, the Viet Minh were fairly popular.
However, all the Allies who had won World War~II thought the area should belong to France.
British and Nationalist Chinese forces took the country and then ceded control to France.
Ho Chi Minh wasn't particularly happy with this development,
and the Viet Minh began a guerrilla campaign, resulting in the \textbf{First Indochina War}.
The war fanned out, including Laos and Cambodia, home to the Pathet Lao and the Khmer Serei respectively.

By 1950, the People's Republic of China and the USSR were helping the Viet Minh take back their country.
The French and Americans began to seriously consider using tactical nuclear weapons,
but Eisenhower decided against it.
During negotiations, the US sailed recon flights over Dien Bien Phu based off carriers in the Gulf of Tonkin.

The \textbf{Battle of Dien Bien Phu} would signal the end of French occupation of Indochina.
Viet Minh commander Vo Nguyen Giap served up a decisive defeat against the French,
partly because the French underestimated the Viet Minh following a French victory at Na San.
Following a good amount of trench warfare, Giap ordered a full assault, capturing 10,000 prisoners.

After Dien Bien Phu, Vietnam was partitioned at the \textbf{17th parallel},
and under the Geneva Accords, people were allowed to move between the two new countries for a short period of time.
North Vietnam instituted agrarian reform, setting up a campaign against land owners.

In the south, \textbf{Bao Dai} was instituted as emperor, and \textbf{Ngo Dinh Diem} was appointed prime minister.
Diem quickly moved to crush all his opponents in his pursuit of the destruction of communism,
and he set himself up to win elections by the end of 1955.
Winning as much as 133\% of the vote in some places\footnote{Yes, he rigged the election.},
Diem was elected president of a new independent Republic of Vietnam on October 26, 1955.
Supported by the Americans (Eisenhower had no real alternative),
Diem instituted his ``Denounce the Communists'' campaign, killing and torturing communists across the country.

By 1957, insurgency was starting to take root.
People were getting fed up with Diem's government, but they were disorganized.
In late 1960, the National Liberation Front, aka Viet Cong, was created in an attempt to organize the activists.
The Viet Cong wanted American influence out of Vietnam.
By 1960, North Vietnam decided to start helping, and they invaded Laos.
In 1962, multiple countries got together and agreed to respect Laos's neutrality.

The South Vietnamese ARVN army was more or less incompetent in the face of the Viet Cong.
At the Battle of Ap Bac, the Viet Cong defeated the ARVN, despite having many disadvantages.
In the middle of 1963, the US started discussing a change of leadership in South Vietnam.
The CIA looked the other way when some generals overthrew Diem and killed him.
After the coup, chaos reigned supreme.
Military regimes in South Vietnam came and went quickly.

In August 1964, USS \textit{Maddox} fired on some torpedo boats in the Gulf of Tonkin.
Later, USS \textit{Turner Joy} and \textit{Maddox} were both fired on in gulf.
This \textbf{Gulf of Tonkin incident} led to some debate in Congress.
In response to the incident, the US began Operation Pierce Arrow, in which the Navy engaged North Vietnamese ships.
Johnson also used the incident to pass the \textbf{Gulf of Tonkin Resolution} on August 7,
giving the president power to conduct military operations in the area without actually declaring war.
The only person who opposed the bill in the House was Eugene Siler.

The bombing in North Vietnam escalated in three stages.
It included Operation Flaming Dark, Operation Rolling Thunder, and Operation Arc Light.
Operation Commando Hunt targeted places such as the Ho Chi Minh trail, a supply route through Laos and Cambodia.
The war proceeded to escalate, with the incompetence of the South Vietnamese military becoming more apparent.
\textbf{William Westmoreland}, deputy commander of Military Assistance Command Vietnam said:
\begin{quote}
  I am convinced that U.S. troops
  with their energy, mobility, and firepower can successfully take the fight to the NLF\@.
\end{quote}

In late 1967, Communists baited American troops out at Dak To
and at \textbf{Khe Sanh Marine Base} on the border with Laos.
Americans, perfectly willing to take their massive military might into the middle of nowhere,
where they didn't have to worry about civilian collateral casualties, obliged.
But, at the end of January 1968, the Viet Cong broke the truce that went along with the Tet (New Year),
and they began the \textbf{Tet Offensive}.
Enemy troops attacked over 100 cities, including Westmoreland's headquarters and the US Embassy in Saigon.
Initially staggered by the surprise and scale of the assault, the Americans responded in kind.
At the Battle of Hue, the US recaptured a former capital from Viet Cong forces.

Many civilians died in the Tet Offensive, but the US was able to take out lots of Viet Cong troops as well.
In fact, it was a military victory for US forces.
However, the Tet Offensive was notable for its side effects.
Westmoreland, who had been well known back home,
having been featured on the cover of \textit{Time} and named Person of the Year,
suddenly found himself losing all his popularity.
American support for the war started to wane, and people started to turn on the Johnson administration.
This was an intelligence failure, some say, as big as not seeing the Japanese attack on Pearl Harbor.
In an editorial, Walter Cronkite said:
\begin{quote}
  To say that we are closer to victory today is to believe, in the face of the evidence,
  the optimists who have been wrong in the past.
  To suggest we are on the edge of defeat is to yield to unreasonable pessimism.
  To say that we are mired in stalemate seems the only realistic, yet unsatisfactory, conclusion.
\end{quote}
Westmoreland was promoted out of Vietnam, and he was succeeded by \textbf{Creighton Abrams}.
Peace talks started in Paris, and the US stopped bombardment of North Vietnam.

When Nixon was elected, he started to withdraw US military from the area.
The idea behind his ``Nixon Doctrine'' was to make sure the ARVN could hold on their own,
a policy known as ``Vietnamization''.
Nixon also worked to try and make peace with the USSR and China.

On March 16, 1968, Army soldiers raped and killed hundreds of unarmed civilians in South Vietnam.
The \textbf{My Lai massacre} occurred in a supposed Viet Cong stronghold in the area, codenamed Pinkville ---
the incident was originally called the Pinkville massacre.
Air support flown by Warrant Officer Hugh Thompson, Lawrence Colburn, and Glenn Andreotta helped stop the massacre,
earning themselves the Soldier's Medal in the process.
Public awareness of the carnage was stirred up when Ron Ridenhour wrote a letter to Mo Udall;
Ridenhour also investigated the event.
Second Lieutenant William Calley was court martialed
and was the only person convicted of any crimes in the aftermath of My Lai.

Incidents such as My Lai really helped to spur opposition to the war in the US\@.
The situation was bad enough that Governor of Ohio Jim Rhodes threatened martial law.
In 1970, students at \textbf{Kent State} protested Nixon's decisions regarding the war.
In the Prentice Hall parking lot near Blanket Hill, the Ohio National Guard fatally shot four students.
The Scranton Commission was assembled to investigate the Kent State shootings.

The \textbf{Paris Peace Accords} on ``Ending the War and Restoring Peace in Vietnam'' were signed in January 1973,
having been negotiated by \textbf{Henry Kissinger} and Le Duc Tho,
who both won the Nobel Peace Prize for their efforts (though Tho declined it).
A ceasefire was declared, and elections were to be held in both North and South Vietnam.

During the ceasefire, the North Vietnamese kept building infrastructure,
preparing for one final massive invasion of the south.
The success of a dry season offensive in 1973-1974 inspired the North Vietnamese to try again.
In 1975, they launched Campaign 275, prompting President of South Vietnam Nguyen Van Thieu to order a retreat.
By the end of March, Hue and fallen, and Da Nang and its airport were in ruins.

The North Vietnamese then pushed for Saigon.
The Ho Chi Minh campaign aimed to take the city by May 1.
On April 27, they encircled the city.
The NVA shelled the airport, closing it.
Martial law was declared.
Operation Frequent Wind was launched for evacuation.
On April 30, NVA troops entered Saigon.
They took Independence Palace.
President Duong Van Minh surrendered.

\section{Space Race}

The goal of annihilating the enemy with planes and guns and bombs
fueled an initiative on both sides of the Cold War to get things further up into the sky.
The German scientist \textbf{Wernher von Braun} had been director of Nazi Germany's missile program,
and during the 1950s, he was brought over to the US to help the Americans make rockets.
His team went to White Sands Proving Ground in New Mexico and started working on rockets
that would eventually become the basis for the Jupiter and Saturn rocket family.

The Soviets decided to try building a satellite.
Sergei Korolev founded the Soviet space program and he got permission in 1957 to put a \textit{sputnik} in the air.
On October 4, 1957, the new \textbf{Sputnik 1} satellite was launched into orbit.
In light of this new development, Eisenhower ordered the US Vanguard project to launch early.
However, when Project Vanguard failed to launch and exploded,
newspapers called it Flopnik, Stayputnik, Kaputnik, Dudnik, etc.
By the end of January 1958, von Braun and his team launched Explorer 1 successfully into orbit.

On April 12, 1961, the USSR launched \textbf{Yuri Gagarin} into orbit in \textbf{Vostok 1}.
Gagarin, the first cosmonaut, was the first human to make it into space.
He became a hero of the Soviet Union.
When the US created NASA, the Air Force's attempts to put a man in space was renamed \textbf{Project Mercury}.
On May 5, 1961, they launched \textbf{Alan Shepard} into space.
Shepard didn't achieve orbit, but he did manage to exercise control over his attitude and rockets.
The Mercury suborbital flight was repeated in July by \textbf{Virgil ``Gus'' Grissom}.
In February 20, 1962, \textbf{John Glenn} became the first American to orbit the Earth.
Meanwhile, the Soviets launched more Vostok flights, including Vostok 6,
which launched \textbf{Valentina Tereshkova}, the first woman in space.

The Americans were fairly humiliated that they had been beaten twice in a row at this point.
Kennedy decided to look into the space program himself, with the new goal being the moon.
On September 12, 1962, he gave a speech at Rice University:
\begin{quote}
  \textbf{We choose to go to the Moon} in this decade and do the other things,
  not because they are easy, but because they are hard,
  because that goal will serve to organize and measure the best of our energies and skills,
  because that challenge is one that we are willing to accept,
  one we are unwilling to postpone, and one which we intend to win\dots{}
  It is for these reasons that I regard the decision last year
  to shift our efforts in space from low to high gear
  as among the most important decisions that will be made during my incumbency in the office of the Presidency.
\end{quote}

NASA then announced \textbf{Project Gemini},
a two-man craft that would help develop technologies for the future three-man Apollo program.
Gemini ran for 10 piloted missions.
The Soviets worked on the Soyuz spacecraft and the Voskhod program.

As we know, eventually Apollo 11 was successful.
Crewed by \textbf{Neil Armstrong}, \textbf{Buzz Aldrin} and \textbf{Michael Collins},
the mission landed on the moon on July 21, 1969.
There's more space travel stuff that can be talked about, but let's move on for now.

\section{Last Days of War}

\subsection*{Mikhail Gorbachev}

Brezhnev left office in 1982.
He was succeeded by Yuri Andropov, former head of the KGB, and Konstantin Chernenko.
Their leadership didn't last very long,
and in 1985, \textbf{Mikhail Gorbachev} was elected General Secretary by the Politburo.
Gorbachev had a big wine-colored birthmark on his forehead that cartoonists were quite fond of mocking.

Gorbachev began to turn around the stagnant economy.
His agenda was called \textbf{perestroika}, or restructuring.
It relaxed the production quota system and let people own their own businesses through the Law on Cooperatives.
He also started \textbf{glasnost}, or openness, which introduced transparency and improved freedom of the press.
Gorbachev implemented the ``Sinatra Doctrine'',
which let Warsaw Pact nations determine for themselves how they would work
(the name's an allusion to the song ``My Way'').

An August 1991 Communist coup tried to target Gorbachev after the Cold War ended,
and it was suppressed by \textbf{Boris Yeltsin}, who then gave a speech while standing on a tank.

\subsection*{Raising the Iron Curtain}

Because of the fact that the USSR was giving in a bit,
Reagan decided that it would be OK to start talks with the Russians on thawing the Cold War.
The first meeting was held in Geneva,
and the second Reykjavik Summit was set up in Iceland.
The third summit resulted in the Intermediate-Range Nuclear Forces Treaty (INF).

In 1987, Reagan gave a speech at the Brandenburg Gate.
In challenge to Gorbachev, Reagan said:
\begin{quote}
  We welcome change and openness; for we believe that freedom and security go together,
  that the advance of human liberty can only strengthen the cause of world peace.
  There is one sign the Soviets can make that would be unmistakable,
  that would advance dramatically the cause of freedom and peace.
  General Secretary Gorbachev, if you seek peace, if you seek prosperity for the Soviet Union and Eastern Europe,
  if you seek liberalization:
  Come here to this gate!
  Mr.\ Gorbachev, open this gate!
  \textbf{Mr.\ Gorbachev, tear down this wall!}
\end{quote}

In 1989, the Berlin Wall fell, and the iron curtain across the continent was lifted.
With the end of the Cold War, the 1989 revolutionary wave washed over Central and Eastern Europe.
