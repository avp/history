\chapter{Cold War}

\epigraph{%
  From Stettin in the Baltic to Trieste in the Adriatic, an iron curtain has descended across the Continent.
}{Winston Churchill}

\section{Germany Divided}

The Second World War had killed almost 10\% of the German population.
The place was in shambles.
At Potsdam, the country had been divided into four zones by the Allies.
The country didn't regain independence until 1949.

In East Germany, Soviets ruled with an iron fist, employing the \textbf{Stasi} as the official state security.
From 1950 to 1971, the party boss in the area was \textbf{Walter Ulbricht}.
In 1953, Ulbricht's government had to put down an uprising using Soviet troops and tanks.

Berlin, deep in the heart of East Germany, was itself divided into separate occupation zones.
Stalin then decided to blockade West Berlin, disallowing any Allied supplies from reaching the city.
The Soviets introduced the Ostmark in their zone to combat the announcement of the new Deutsche Mark.
As a result, Lucius Clay, deputy governor of Germany, put Operation Vittles into effect.
The \textbf{Berlin Airlift}, as it was called,
aimed to fly two million tons of supplies into the city and drop them to the people below.
Construction for the bases occurred at places near Tempelhof and Lake Tegel.
Albert Wedemeyer, a general who had commanded the airlift from India over the Himalayan Hump, helped the effort.
The British contribution was codenamed Operation Plainfare,
and the Australian operation was designated Operation Pelican.

By 1961, 2.6 million people had fled from East Germany,
and Ulbricht created the \textbf{Berlin Wall} to try and stop them.
The wall was crossable in few locations, including Checkpoint Charlie,
and abandoned subway stops past the wall were called ``ghost stations''.
During the Berlin Wall crisis in 1961, Lucius Clay ordered tanks to a standoff to back up Albert Hemsing.

\section{US-Soviet Relations}
% Bay of Pigs

\subsection*{Truman and his Doctrine}
% Election 1948

Let's talk about Harry S.\ Truman, Roosevelt's final Vice President.
Truman was born in Missouri, and he joined Tom Pendergast's political machine in Kansas City.
As discussed before, Truman assumed the presidency just before the Nazis surrendered,
and he ordered dropping the bombs on Hiroshima and Nagasaki.

Following the victory in World War~II, Truman and friends helped set up the \textbf{United Nations}.
He also issued the \textbf{Truman Doctrine}, giving aid to Turkey and Greece, in an attempt to contain communism.
Truman's policy of containment was inspired partly by \textbf{George F.\ Kennan} (Mr.\ X)
and his writings talking about \textbf{containment} of communism.
Kennan sent the \textbf{Long Telegram} from Soviet Russia,
claiming that the Soviets were expansionist and needed to be contained in places that mattered to the US\@.

In order to help rebuild Europe, Secretary of State \textbf{George Marshall} came up with a strategy.
The \textbf{Marshall Plan} consisted of using \$13 billion to bring European economies back on track,
and William Clayton, Kennan, and Marshall unveiled the plan at a graduation speech at Harvard in 1947.
To counter the Marshall Plan, the Soviets put their own Molotov Plan into effect to help Eastern European countries.

Truman ran for reelection in 1948.
His opponent was \textbf{Thomas Dewey}, a New York Republican.
Also running was \textbf{Strom Thurmond}, who had shown up from South Carolina with the support of Dixiecrats,
and formed the States' Rights Democratic Party with his running mate Fielding Wright.
During the election, Truman criticized the ``Do-Nothing Congress'' for opposing his Fair Deal.
On election day, the Chicago Tribune preemptively announced election results: ``Dewey Defeats Truman!''
However, they were wrong because some polls weren't available, and Truman ended up winning reelection.
Truman's new cabinet included Secretary of State \textbf{Dean Acheson}.

\subsection*{NATO \& the Warsaw Pact}

In 1949, the former Allies formed the \textbf{North Atlantic Treaty Organization (NATO)}.
Headquartered in Brussels, NATO's goal was to stop the growing Soviet threat.
They set up yearly readiness exercises named after Able Archer and Dean Rusk.

In response, the Soviets set up the \textbf{Warsaw Pact} in 1955.
Included in the organization were China and North Korea,
and during Zapad 81, they would carry out the largest military exercise ever.

\subsection*{Korean War}
% MacArthur -> Ridgeway

Korea had been ruled by Japan since 1910 until the later parts of World War~II\@.
Then, the Soviets took control of the peninsula north of the \textbf{38th parallel},
and US forces occupied the south, and Japan surrendered, packed up, and left.

Two separate governments were set up.
The south elected \textbf{Syngman Rhee} president in July 1948.
Rhee had converted to Christianity in High School, and he had spent time in America.
He opposed Communism, and as soon as he took power,
he started instituting laws against political dissent, smashing his leftist opponents.

In Communist North Korea, a new regime was established under \textbf{Kim Il-Sung}.
His symbol was a blue orchid that had been given to him by Sukarno, and he made it the official flower of the country.
He set up the kwan li so labor camps,
and he introduced the Juche (self-reliance) political ideology, displacing Stalin and Mao.
In 1968, his government would capture \textbf{USS Pueblo}, commanded by Captain Lloyd Bucher.

Now, both Rhee and Kim Il Sung wanted to unite Korea under their respective governments.
Stalin set up the North Koreans with weapons,
and they crossed the 38th parallel on June 25, 1950, fighting on the Ongjin peninsula.
They came south, forcing Rhee to leave Seoul and drastically reducing the headcount of South Korean troops.

At this point, the Truman administration wasn't really well prepared for military action in Korea.
There were worries as to what the Soviets would do if the US intervened in Korea.
Truman wrote:
\begin{quote}
  There was complete,
  almost unspoken acceptance on the part of everyone
  that whatever had to be done to meet this aggression had to be done.
  There was no suggestion from anyone that either the United Nations or the United States could back away from it.
\end{quote}
There was fear of a chain reaction allowing the Soviets to disregard the UN
and create Communist aggression in any number of other places across the world.
On June 27, after the UN Security Council ruled that the North Korean invasion was bad,
Truman sent the 8th Army into the peninsula to help the South Koreans.

The first significant American engagement came at the \textbf{Battle of Osan}.
The battle was unsuccessful,
and the North Korean KPA pushed the South Korean ROK and the US army back to \textbf{Pusan}.
Battles around this time included Onjong.
Kim Il-sung thought he would end the war by the end of August.

To relieve the Pusan perimeter, General MacArthur ordered the planning of a landing at \textbf{Inchon}.
The battle began with USS \textit{Mansfield}, USS \textit{Swenson}, and USS \textit{Collett} bombarding a fortress.
The X Corps then landed at Green, Blue, and Red beaches, and captured the Kimpo airstrip,
allowing for a breakout from the Pusan perimeter.
By the end of September, Seoul was back in South Korean hands.

Continuing their momentum, MacArthur and the ROK went up western Korea and took Pyongyang in October.
MacArthur then thought it was going to be a good idea to keep going straight on into China, but Truman disagreed.
When China entered the war, it led to major Allied losses,
and MacArthur was seriously considering using nuclear weapons to end the war.
MacArthur's Home-by-Christmas Offensive was met with the Chinese Second Phase Offensive,
ambushing and pushing the allied forces back from the Yalu River.

While MacArthur wanted complete surrender from the North Koreans,
Truman wasn't as optimistic about the land war in Asia (he realized it was a classic blunder).
So, Truman relieved MacArthur of his command, and put \textbf{General Matthew Ridgway} in charge.

The rest of the Korean War would mostly be spent in stalemate.
Combat continued during armistice negotiations, which would take a very long time to go through.
Battles during the stalemate included:
Pork Chop Hill,
Heartbreak Ridge,
White Horse Mountain,
Hill Eerie,
and the siege of Outpost Harry.
An armistice was agreed on in July 1953, setting up a \textbf{Demilitarized Zone (DMZ)} at the 38th parallel.
The area is still patrolled by members of both sides, and clashes occur sometimes.

\subsection*{Missiles in October}
% Khrushchev

\subsection*{Brezhnev in Power}

\section{Vietnam War}
% Tet Offensive
% My Lai
% Gulf of Tonkin
% Kent State
% William Westmoreland

\section{The Wall Comes Down}

\subsection*{Reagan \& Thatcher}
% Gorbachev
% Honecker
% Reagan

% The General Secretary after Ulbricht was \textbf{Erich Honecker} (1971--1989).
