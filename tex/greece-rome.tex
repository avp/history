\chapter{Ancient Europe}

\section{Persian Wars}

In modern day Iran, a large war machine was massing in the 6th century BC\@.
The Greeks would end up not liking them very much.

\textbf{Cyrus the Great}, grandson of Astyages, founded the \textbf{Achaemenid} Empire.
He ended the Babylonian Captivity, in which many Jews were held captive in Babylon,
and his Edict of Restoration allowed them to return home.
As far as war went, he was victorious at Thymbra and he beat Croesus of Lydia at Sardis,
Cyrus also overthrew the Medes (Median Empire).
His ``cylinder'' is considered an important declaration of human rights.
Xenophon immortalized the man in his text \textit{Cyrus the Great}, and Cyrus's capital was at Pasargadae.

Now, having built this empire, some Persian kings wanted to also conquer Greece.
Obviously, the Greeks weren't very pleased with this proposition, and fought back.

\textbf{Darius I (the Great)} first attempted to conquer Greece.
To gain power, he overthrew Gaumata, and bragged about it in the Behistun Inscription.
Following the Ionian Revolt, he sent troops to Marathon.

The Battle of \textbf{Marathon} (490 BC) coincided with the Carneia festival, so the Spartans didn't fight.
The Persians led by Artaphernes (Darius's brother) and Datis.
At Marathon (490 BC), the Athenians, led by \textbf{Miltiades}, won,
and \textbf{Pheidippides} the runner ran back to Athens and died, making the marathon.

Darius also began construction of Persepolis, the new Persian capital.
He increased the number of satraps (governors) in Persia to 23.
When he died, he was succeeded by \ldots

\textbf{Xerxes I}, who kept trying to conquer Greece.
He enhanced the city of Persepolis, and ordered the building of the Gate of all Nations.
He faced multiple rebellions for melting down a big statue of Marduk in Babylon.
At one point, a bridge spanning the Hellespont broke, and Xerxes decapitated the engineers.
He then had a bridge of boats built to cross the Hellespont,
at which time he whipped the sea for insolence.
During the second Persian invasion of Greece, there were some important battles:

\textbf{Thermopylae} (480 BC) was a famous battle in which \textbf{Leonidas} and his 300 Spartans
made a final stand against Xerxes in a narrow pass.
They actually had a decent chance of surviving,
but a traitor named Ephialtes decided to tell Xerxes II about a path that
was hidden in the mountains, and the Spartans returned home on their shields.

\textbf{Salamis} (480 BC) was a naval battle near the Isthmus of Corinth, close to Piraeus.
Themistocles led a force of triremes against the navy of Xerxes.
Xerxes places men on Psyttaleia, a small island near the exit of the straits,
to try and kill/capture any Greeks that dared pass.
However, Aristides led an Athenian detachment and killed the Persians on Psyttaleia.
Other people at the battle included Artemesia of Caria (Queen of Halicarnassus).
Xerxes reportedly saw 200 of his triremes sink into the sea from Egaleo hill, and he wasn't happy.
This was the high point of the Persian invasion, and after this, they began to retreat.
Salamis has been called ``the battle that saved Western civilization''.

\textbf{Plataea} was the final decisive defeat of the Persian army.
After this, the Persians gave up and decided that Greece wasn't worth it.

\section{Greek City-States}

\textbf{Athens} was a very forward-thinking Greek city-state.
Democracy originated in Athens, pioneered by reformers like Solon and Cleisthenes.
Another notable leader of Athens was \textbf{Draco}, who was quite strict in his laws.
They practiced ostracism, a practice started by Cleisthenes ---
every year, the citizens would vote to exile someone for five years.
They built the Parthenon on their Acropolis, where the Elgin Marbles would be taken from years later.
Notably, Athens's harbor and port was \textbf{Piraeus}, where the ships that won at Salamis were built.

\textbf{Sparta} (also called Lacedaemon) was a militaristic Greek city-state.
Children were taught to fight from a very young age, and soldiers were told to return with their shield --- or on it.
Slaves and of Sparta were called helots, and the ruling class were known as ephors.
Future leaders of Sparta were allowed to kill helots to prove themselves in a ritual called the \textit{krypteia}.
When Philip II of Macedon sent a message to Sparta regarding its province of Laconia,
saying that if he invaded Laconia, Sparta would be destroyed, never to rise again.
The ephors responded with the message, ``If''.
So, Philip II and his son Alexander never tried to conquer the city.

\subsection*{Peloponnesian War}

After the thwarted invasion of Persia and the ensuing Greco-Persian wars,
Athens and Sparta fought during the \textbf{Peloponnesian War} (431--404 BC).
Athens brought in some allies in the form of the Delian League.
Briefly during the war, the Council of Four Hundred took power in Athens.
At another time, \textbf{Pericles}, the leader of Athens, delivered a funeral oration.

The Athenians, in an attempt to win, launched the Sicilian Expedition, which attempted to take Syracuse.
It didn't end very well, because the expedition's commander, \textbf{Alcibiades}, was recalled to Athens,
and Sicily didn't really care about Athens's threats.
Alcibiades would go on to be accused of profaning the Eleusinian mysteries,
and he notably switched sides multiple times during the Peloponnesian War.
Notably, when the Athenians were going to go home, there was a lunar eclipse.
Being superstitious, Nicias and friends decided not to sail, and were attacked by the Syracusians.
Other battles include Cyzicus, where the Athenians won under the command of Thrasybulus.

The last major battle of the war was \textbf{Aegospotami}, where a Spartan fleet,
under Lysander, crushed the Athenian navy.
Athens also had to endure a bad plague, during which Pericles died.
Other notable battles include Notium, Sphacteria, and Delium.
Halfway through the war, the Peace of Nicias interrupted the war.
Thucydides ended up writing a history of the war.

\subsection*{Macedonia}

The Sacred Band of Thebes was an elite force of 150 male lovers under Epimonandas.
The band fought at the battle of Leuctra against Cleombrotus of Sparta, during one of the post-Peloponnesian conflicts.

Also of note was \textbf{Philip II} of Macedon, son of Eurydice and Amyntas~III\@.
He rose to the throne when his brother, Perdiccas~III,
died in battle against the Illyrians, leaving his son, Amyntas~IV\@.
He united all of Greece, using both iron and gold to achieve his goals,
conquering Paeonia and Thessaly (disregarding warnings by Demosthenes).
He would also destroy the Sacred Band of Thebes at the battle of Chaeronea.

Philip's son was \textbf{Alexander the Great}, who would become one of the greatest conquerors of his time.
Alexander the Child was tutored by \textbf{Aristotle}, and he became a brilliant tactician and leader.
When he was young, he helped his father at Chaeronea.
When he was declared king, he killed some rivals, including Amyntas~IV\@.
He had defeated King Porus at the Hydaspes River (but had lost his horse, Bucephalus),
and he cut the Gordian Knot because he couldn't untie it.
Alexander decided to try and conquer the massive Persian Empire, which was now ruled by Darius III\@.
At the battle of Issus, Alexander met the forces of Darius, and pushed him back, making Darius mad.
So, on the plains of \textbf{Gaugamela}, Darius flattened the grass to make it easier for his
large numbers of chariots to outmaneuver Alexander's forces.
But, Alexander was smarter than Darius, and ended up winning anyway, taking over the Persian Empire.
Alexander executed the general Permenion, as well as Permenion's son, Philotas.
After his death, his generals fought over his vast conquered lands in the wars of the Diadochi,
and his massive empire fell apart.

\section{Rome: From Romulus to Constantine}

While Persia and Greece were having wars and conquering each other, another important city was growing in Italy.
The Roman empire would grow to encompass much of the Western world,
and greatly influence the rest of history.

\subsection*{Monarchy}

There were seven kings of Rome, and here are the important ones:

\textbf{Romulus} founded the city of Rome in 753 BC, along with his brother Remus.
They were raised by a she-wolf, and they overthrew their evil grandfather Amulius and restored Numitor.
Romulus staked his claim on Rome after seeing birds from the Palatine Hill, while Remus was on the Aventine.
However, they got into a brotherly argument, and Remus ended up dead.
As king, he hosted a festival of Neptune, and invited the Sabines, led by Titus Tatius.
There, he authorized his men to kidnap and rape the Sabine Women.
He organized a personal guard called the Celeres,
and when he died, he was deified as Quirinus.

\textbf{Numa Pompilius}, second king of Rome,
established a more organized religion and the office of Pontifex Maximus (biggest religion position).

\textbf{Tarquinius Superbus} (Tarquin the Proud) was the seventh and final king of Rome.
After the rape of Lucretia, wife of Collatinus, he was ousted by Brutus (not the one who killed Caesar).

Also notable during the monarchy:

The \textbf{Etruscans} lived in Italy at the same time (preceding the Romans).
Their king was Lars Porsenna, and one of their major cities was Veii.
At one point, Horatius Cocles defended a bridge against an Etruscan attack.
Mucius Scaevola put his hand into a fire to prove to Porsenna that he was strong.

\subsection*{Republic}

With the death of Tarquinius Superbus, the Roman Republic began.
It was ruled by consuls and a Senate, and lasted until about 14 BC\@.
Collatinus and Brutus were the first two consuls.

Some more notable people during the Roman Republic:

Gaius \textbf{Marius} was known as the ``third founder of Rome'' for defeating the Cimbri and Teutones,
at battles such as Vercellae and Aquae Sextiae.
He defeated Jugurtha, the Numidian, and enacted various military reforms.
He was elected consul for six years running, and his arch-nemesis was \ldots

Lucius Cornelius \textbf{Sulla}, a general who captured Jugurtha after Marius defeated him,
thus ending the Jugurthine war.
He won the Battle at the Colline Gate (and defeated his last remaining opponents),
and was appointed as the first dictator for life.
He doubled the size of the senate and increased the number of praetors (judges).
During his proscriptions, he promised rewards for the deaths of his enemies.
He made peace with Mithridates VI, king of Pontus.
Machiavelli refers to Sulla as ``half fox and half man''.

\textbf{Pyrrhus} of Epirus had a conflict with Rome as well.
He aided the city of Tarentum, in Magna Graecia (southern Italy).
He took a bunch of elephants to stop the Romans and defeated the Romans at Heraclea.
Pyrrhus won a costly victory at Asculum (279 BC) resulting in the term ``Pyrrhic Victory''.
He then became ruler of Sicily for a time, before retreating from Italy.

During the Republic, Rome got into a series of three Punic Wars against Carthage.
During these wars, Cato the Elder repeatedly said ``Carthage must be destroyed''.

The \textbf{First Punic War} primarily took place on Sicily, with quite a bit of fighting around Syracuse.
It was a result of Hiero~II stopping a Mamertine uprising in Messina, Sicily.
As a result of the First Punic war, Sicily became the first conquered Roman territory.

The \textbf{Second Punic War} was a much bigger deal.
It started over a conflict of control of Saguntum in Hispania (Spain).
\textbf{Hannibal} Barca, his father Hamilcar, and his brother Hasdrubal, were the leading Carthaginian commanders.
Hasdrubal would die at the Battle of Metaurus during the war.
They would be opposed by \textbf{Scipio Africanus}, a highly competent Roman general.
Hannibal took his elephants and his army through Spain and across the Alps,
and he attacked the Romans when they weren't expecting it.
He won battles at Trebia, Lake Trasimene, and Cannae.
At this point, the Romans were deploying the ``Fabian Strategy'', which was them just trying to delay dying.
The Romans pushed back, and eventually the final showdown at Zama was a win for Scipio.
Hannibal eventually killed himself using poison he kept stored in a ring he wore.

% TODO Gracchi

The next most important events in the Roman Republic were largely the work of the two triumvirates.

\textbf{First Triumvirate}:

\textbf{Julius Caesar} was perhaps the most famous of the figures of the Roman Republic.
As a young man, he was kidnapped by pirates, and he was insulted by the low ransom they asked for.
When he eventually became older, he hunted down the pirates and killed them.
In 63 BC, he was chosen to be Pontifex Maximus.
His commentary on the Gallic Wars includes his victory over \textbf{Vercingetorix} at Alesia.
Caesar's generals at Alesia included Titus Labienus and Mark Antony.
He was co-consul with Bibulus, and won at Thapsus.

\textbf{Pompey} was a general. He was granted a navy by the Lex Gabinia, and he defeated
the Mediterranean pirates quickly and efficiently.
He also helped Metellus fight Sertorius, in Spain.
The Lex Manilia gave him command of the war against Mithridates VI of Pontus, taking command from Lucullus.
He quickly drove Mithridates back, and Mithridates proceeded to commit suicide.
His wife was Julia.
His was the first permanent theater in Rome --- where Caesar would later be killed.

\textbf{Marcus Crassus} was a very rich person.
Along with Sulla, he fought at the Colline Gate, where he commanded the right wing.
When \textbf{Spartacus}, a gladiator trained in Capua, aided by Crixus, revolted, Crassus fought against him.
Spartacus would be killed at the Siler River, and
Crassus crucified those who participated in Spartacus's slave revolt along the Appian Way,
and he was mad when Pompey ended up claiming credit for quelling Spartacus.
He lost to the Parthians at Carrhae, and was supposedly killed when they poured gold down his throat.

When returning from Gaul, Caesar crossed the Rubicon with his army, saying ``The die is cast''.
This triggered a civil war with Pompey.
Caesar narrowly avoided being destroyed at Dyrrachium, but came back and decisively defeated Pompey at Pharsalus.
Mark Antony notably helped command the left wing at Pharsalus.
As the new dictator for life, he lived well, increased the size of the senate from 600 to 900.
In 44 BC, his wife Calpurnia warned him to beware the Ides of March,
but he went to the Theater of Pompey and got stabbed many times by Brutus and Cassius,
resulting in the end of his rule and the rise of the Second Triumvirate.
\textbf{Brutus} was notable because he was actually the descendant of the Brutus that had ousted Tarquin.
He was also the only person to coin money with his own face on it,
Brutus had been saved at Pharsalus by Servillia, and he had been raised by Cato the Younger.

\textbf{Second Triumvirate}:

\textbf{Octavian} was the adopted son of Julius Caesar, and he brought together a new triumvirate.
His best friend was \textbf{Marcus Agrippa}, who commanded his armies.
He married Scribonia, and had a daughter with her.

\textbf{Mark Antony} was a friend of Octavian, who eventually fell out with him.
He divorced Fulvia in order to marry Octavia the Younger.
He delivered the eulogy at Caesar's funeral, as noted by Shakespeare:
``Friends, Romans, Countrymen \ldots''.
He defeated Caesar's murderers Brutus and Cassius at Philippi along with Octavian.

\textbf{Lepidus} is relatively lesser known, but he had a lot of money.

Later, when Octavian and Antony became enemies, Antony was in Egypt with Cleopatra.
They battled, and in 31 BC, at the Battle of Actium, Agrippa was sent to crush Antony's forces.
They were successful, and Octavian became Augustus, first emperor of Rome.

\subsection*{Empire}

Rome had quite a few emperors, so I'll go over the ones that come up often.

The first five emperors were the Julio-Claudian emperors.

\textbf{Augustus}, formerly Octavian, ruled over the Pax Romana.
He failed to uphold the Treaty of Tarentum with the Parthians, and won at Cape Naulochus.

Augustus was succeeded by \textbf{Tiberius}, who wasn't his first choice.
Tiberius eventually exiled himself to Capri and died there.

\textbf{Caligula}, the third emperor, was quite crazy.
His name means ``little boot'', a name he was given as a kid in the military.
In 39 AD, he performed a stunt where he ordered a floating bridge to be built on the Bay of Baiae.
Caligula then rode his horse across the bay, in defiance of an astrologer who told him that he
``had no more chance of being emperor than riding a horse across the bay of Baiae''.
Reportedly, he wanted to make his horse Incitatus a senator.
His sister was Agrippina the Younger, and he was killed by the Praetorian Guard in AD 41 addressing some actors.

\textbf{Claudius} came to power when his nephew Caligula died.
He wasn't the most confident of people, and the guard who came to tell
him that he was going to be emperor supposedly found him hiding behind a curtain, shaking.
Claudius was also a stutterer.
In his letter to the Alexandrians, he ordered toleration of Judaism.
During his reign, Rome conquered Britain, and Claudius spared a Briton named Caractacus.
Eventually, he was murdered on the orders of Agrippina the Younger, so that \ldots

\textbf{Nero}, Agrippina's son, could take the throne.
Nero was almost as insane as Caligula, and he apocryphally fiddled while Rome burned.
He had a huge statue of himself built in front of his golden house, the \textit{domus aurea}.
He sent Paulinus to suppress the revolt of \textbf{Boudicca} in Britain.
The Pisonian conspiracy and a revolt by Vindex were targeted against him,
and the death of this insane matricidal emperor led to \ldots

\textbf{The Year of the Four Emperors} in AD 69. During this year, the four emperors were, in order,
Galba, Otho, Vittelius, and finally, Vespasian.

\textbf{Vespasian} was the founder of the new Flavian dynasty of Roman emperors.
An accomplished general, he'd commanded troops with his son Titus in Judaea during the Great Jewish Revolt,
during which the mass suicide of the rebels at Masada took place.
He had also been proconsul in Africa, and had faced a riot in Hadrumetum.
He'd allegedly thrown people into the Dead Sea to test its buoyancy, according to \textbf{Josephus}.
He began construction on the Colosseum, and construction was completed in AD 80 under Titus.
He sent Agricola to conquer the rest of Britain and finish Claudius's work,
conquering the Isle of Wight in the process.

After Vespasian died, his son \textbf{Titus} became emperor.
He completed the construction of the Colosseum, and he has an arch in Rome.
He is known for his disaster recovery following the eruption of Vesuvius,
an event chronicled by \textbf{Pliny the Elder}.

Following Titus, \textbf{Domitian}, his brother, ruled the empire.
He recalled Agricola from Britain, and purged the Senate after a revolt by Saturninus.
He defeated the Chati, and called himself Germanicus.
He was assassinated in AD 96, and succeeded by Nerva.

Following the death of Domitian, the dynasties changed again.
The new Nerva-Antonine dynasty began with the \textbf{Five Good Emperors}:
Nerva, Trajan, Hadrian, Antoninus Pius, and Marcus Aurelius.

\textbf{Trajan}, adopted son of Nerva, was a military man.
He expanded Rome to its greatest extent by conquering Dacia (Romania),
having defeated Decebalus at Sarmisegetusa and earning the title of Dacicus,
and his efforts were commemorated with a big column in the Forum ---
he had designed a new forum with Apollodorus of Damascus.
His bridge was built east of the Iron Gates of Transylvania,
Lucius Quietus was sent to suppress a Jewish uprising called the Kitos War.
Trajan died while returning from a campaign in Parthia.

\textbf{Hadrian} built a well known wall in Britain (Hadrian's Wall).
He had Lucius Quietus killed after Quietus claimed that Hadrian had made up stories of his adoption.
Hadrian also stopped the Bar Kokhba Jewish revolt, which had been supported by Rabbi Akiva,
and he withdrew Roman troops from Mesopotamia.
He deified his lover Antinous after he drowned in the Nile,
and he stipulated that Antoninus Pius adopt Marcus Aurelius and Lucius Verus.

\textbf{Marcus Aurelius} was a Stoic philosopher, who wrote \textit{Meditations}.
His tutor was Fronto, and his wife was Faustina the Younger.
The Antonine plague was brought to Rome by his legions,
and he fought the Marcomannic Wars.
Notably, Aurelius is the subject of the only equestrian statue that survived antiquity.
He co-ruled with Lucius Verus, and he was succeeded by Commodus.

\textbf{Commodus} (the emperor in \textit{Gladiator}) wasn't a very nice person.
He renamed Rome, and renamed the months after himself.
He ended the Marcomannic Wars that Aurelius began, and he was strangled in his bathtub
as part of a conspiracy created by his mistress Marcia.
He was succeeded by Pertinax.

\textbf{Valerian} was interesting largely because of how he died.
He had been taken captive by \textbf{Shapur~I} of the \textbf{Sassanid Empire}.
The Sassanid Empire was notably the last one to use Zoroastrianism as its state religion.
Shapur allegedly made Valerian swallow molten gold in captivity.

\textbf{Diocletian} was a much later ruler, from 284--305.
He formed the tetrarchy, splitting the Roman empire into four parts.
After he resigned, his spot was contested.

The two people fighting for his spot were \textbf{Constantine I, the Great} and \textbf{Maxentius}.
Before the decisive battle at the Milvian Bridge (312), Constantine had a dream.
He saw a chi-rho symbol in the sky, with the words ``in this sign, you will conquer''.
He had his soldiers paint it on their shields, and they won.
This resulted in Constantine converting to Christianity and being the first Christian Roman emperor.

Constantine was victorious against Abantus at the Hellespont,
and later at Chrysopolis he finally defeated his rival Licinius and gained full control of his empire.
He convened the First Council of Nicaea, and he issued the Edict of Milan,
which allowed for Christian toleration.
A forgery during his reign claimed that Constantine was apparently moving east,
and giving all his lands to Sylvester~I, the pope and the Catholic Church,
but this ``Donation of Constantine'' was disputed by Nicolas of Cusa and Lorenzo Valla.
Constantine killed his son Crispus and his wife Fausta, somewhat controversially.
His capital was moved to Byzantium, which would be renamed Constantinople.
