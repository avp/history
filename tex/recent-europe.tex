\chapter{Recent European History}

\section{Republics of France}

\subsection*{Fourth Republic}

On October 13, 1946, a new \textbf{Fourth Republic} established itself in France.
The French Colonial Empire proceeded to fall apart.
French Indochina was lost at the Battle of \textbf{Dien Bien Phu},
falling into Viet Minh hands, led by Ho Chi Minh (more on that later).

In 1956, France faced the \textbf{Suez Crisis}.
France had built the Suez Canal, and therefore owned the Canal.
When Egyptian President \textbf{Gamal Abdel Nasser} (more in the later chapters)
decided to nationalize the Canal, Britain and France attacked.
Isreal had also allied with Britain and France in the secret Protocol of Sevres.
It started when Israel, aiding the British and French, began Operation Kadesh.
They invaded Port Said as part of Operation Musketeer.
Eisenhower would put a stop to the fighting.

\subsection*{President de Gaulle}

Shortly after the Suez Crisis, Guy Mollet was forced out of the post of Prime Minister.
By 1958, the Fourth Republic was falling apart under Rene Coty.
French army units took power in Algiers in May and the whole thing fell apart.

Into this crisis rose \textbf{Charles de Gaulle}, who had led the Free French in World War~II\@.
The National Assembly put de Gaulle in power and he founded the new \textbf{Fifth Republic}.
He introduced his ``Politics of Grandeur'' and demanded that France be given complete autonomy in its affairs.
He withdrew France from NATO's military command, because he thought that the US had too much control.
In 1965, de Gaulle helped start the Empty Chair Crisis,
which involved financing the Common Agricultural Policy of the European Economic Community (EEC).
The crisis was resolved when the Luxembourg compromise was reached in January 1966.

On a visit to Canada in 1967, de Gaulle voiced support for a free Quebec,
declaring, ``vive le Quebec libre'' (long live free Quebec).

The new president also had to deal with the war in Algeria.
After visiting Africa, he decided that he supported independence for Algeria,
and neutralized the army that was stationed there.
In 1959, he gave the country self-determination, and there followed a revolt by the French settlers.
In 1962, he signed the \textbf{Evian Accords},
creating a ceasefire in Algeria and giving victory to the FLN (National Liberation Front in Algeria).
This resulted in several assassination attempts against him by the OAS (the settlers' resistance group),
including one in which he narrowly escaped machine gun fire in a limousine.

In May 1968, protests broke out against de Gaulle's government.
His OTRF broadcasting organization had a monopoly on TV and radio.
On May 29, de Gaulle disappeared without telling anyone.
He went to Baden-Baden, from whence he returned with the military's support.
During the revolts, the Grenelle Agreements were signed, increasing the minimum wage,
but failing to resolve the conflict.
Eventually, de Gaulle's time as a leader was up and \textbf{George Pompidou} took over running the country.

\section{United Kingdom: The Commonwealth}

Important things happened between the end of the Second World War and the late 1960s,
but I'm not going to talk about them.

\textbf{The Troubles} started in 1968 as a conflict between republicans and unionists in Northern Ireland.
They started when a series of riots broke out in Londonderry.
During the Troubles, ten inmates at Maze Prison, including Bobby Sands,
starved themselves to death.
The Troubles were ended by the \textbf{Good Friday Agreement} in 1998,
which came about as the result of peace talks chaired by George Mitchell.

In the 1970s, the UK went through some rough economic times.
Labour had returned to power in 1974 under Harold Wilson.
The economy got worse until Wilson was replaced by James Callaghan.
But, he presided over the Winter of Discontent and he was voted out with no confidence.

In 1979, Conservative \textbf{Margaret Thatcher}, the Iron Lady, was elected prime minister.
Philosophically, Thatcher held views similar to Ronald Reagan and Brian Mulroney (elected in Canada in 1984).
Her Conservative Party was made up of ``wets'' and ``dries''.
After the Maze Prison hunger strike, Sinn Fein started to come back into power.
In 1981, the royal family was brought back into the limelight when
\textbf{Prince Charles} married \textbf{Princess Diana}, and everyone cared.

In Argentina, an unstable military junta under Leopoldo Galtieri had taken power.
On April 2, 1982, Argentina invaded the British-controlled Falkland Islands, starting the \textbf{Falkland War}.
In response, the British sent a naval task force to the islands.
They opened with the Black Buck Raids and the Raid on Pebble Island,
and proceeded to try Operation Mikado, which was aborted.
After an amphibious landing at Port San Carlos, they won the Battle of Goose Green.
A big chunk of Argentine casualties occurred when the British sunk ARA \textit{General Belgrano}.
The British used Ascension Island as a staging ground, where they used US technology.
The whole war lasted over seventy days, and resulted in British victory and restoration of the status quo.

In other foreign affairs, Thatcher visited Deng Xiaoping in China
and negotiated the peaceful transfer of Hong Kong back to China in 15 years.
She also set up the creation of Zimbabwe from the former state of Rhodesia.

At home, Thatcher faced a National Union of Mineworkers strike led by Arthur Scargill.
During the Westland affair,
Thatcher and her Defence Minister Michael Heseltine dealt with the rescuing of Westland Helicopters.
The incident embarrassed the government, hurt Thatcher's public reputation, and led to Heseltine's resignation.
At one point, she survived an IRA assassination attempt by Patrick Magee and his men
in which they tried to blow her up in the Grand Hotel in Brighton.
When she tried to institute the Community Charge (essentially a poll tax)
her popularity vanished and she was replaced by \textbf{John Major}.

\section{Eastern Bloc}

\subsection*{Czechoslovakia}

After World War~II, Czechoslovakia expelled a bunch of Germans who'd been living in its borders.
The Third Republic began, but the country became part of Stalin's sphere of influence.
In February 1948, Communists took over the country, and Edvard Benes came to power.
In June 1953, strikes in Plzen were broken up without too much blood,
disappointing Allen Dulles, who'd wanted an excuse for the CIA to help the people resist Soviet control.
The 1960 Constitution proclaimed the victory of socialism, and created the Czechoslovak Socialist Republic.

In January 1968, \textbf{Alexander Dubcek}, a Slovak reformer,
was elected first secretary of the KSC (Communist Party of Czechoslovakia).
He replaced Antonin Novotny as Novotny also ceded the presidency to Ludvik Svoboda.
Dubcek started to enact a new reform movement in the spring of 1968, which would be known as \textbf{Prague Spring}.
On April 5, he set up the Action Programme, which guaranteed freedom of religion, press, assembly, and speech,
calling it ``socialism with a human face''.
During Prague Spring, Ludvik Vaculik published his ``Two Thousand Words'' manifesto.
Other Warsaw Pact countries weren't particularly happy about what was happening in Prague Spring,
so they invaded Czechoslovakia in August.
Dubcek was arrested and taken to Moscow, where he negotiated the \textbf{Brezhnev Doctrine},
allowing for Czechoslovakia to become partially sovereign, with the KSC gaining power.
Notably, in January 1969,
Jan Palach set himself on fire in Prague to protest the invasion of the country and the end of Prague Spring.
By April 1969, Dubcek was replaced by \textbf{Gustav Husak}.

Skip forward a bit to 1989, when the anti-Communist revolution started.
Between November and December of that year, the non-violent \textbf{Velvet Revolution} took place.
The main anti-communist party was Charter 77, a civic initiative that managed to overthrow the communist regime.
Former playwright \textbf{Vaclav Havel} was elected the first president of Czechoslovakia in 41 years.
Havel was reelected and became the first president of the new Czech Republic,
after it split from Slovakia following the Velvet Divorce in 1993.

\subsection*{Hungarian Revolution}

\subsection*{Polish Solidarity}

\section{The Balkans}

\subsection*{Yugoslavia}

\subsection*{Bosnian War}
% Sniper Alley
% Dayton Accords
% Siege of Sarajevo

