\chapter{Recent European History}

\section{Republics of France}

\subsection*{Fourth Republic}

On October 13, 1946, a new \textbf{Fourth Republic} established itself in France.
The French Colonial Empire proceeded to fall apart.
French Indochina was lost at the Battle of \textbf{Dien Bien Phu},
falling into Viet Minh hands, led by Ho Chi Minh (more on that later).

In 1956, France faced the \textbf{Suez Crisis}.
France had built the Suez Canal, and therefore owned the Canal.
When Egyptian President \textbf{Gamal Abdel Nasser} (more in the later chapters)
decided to nationalize the Canal, Britain and France attacked.
Isreal had also allied with Britain and France in the secret Protocol of Sevres.
It started when Israel, aiding the British and French, began Operation Kadesh.
They invaded Port Said as part of Operation Musketeer.
Eisenhower would put a stop to the fighting.

\subsection*{President de Gaulle}

Shortly after the Suez Crisis, Guy Mollet was forced out of the post of Prime Minister.
By 1958, the Fourth Republic was falling apart under Rene Coty.
French army units took power in Algiers in May and the whole thing fell apart.

Into this crisis rose \textbf{Charles de Gaulle}, who had led the Free French in World War~II\@.
The National Assembly put de Gaulle in power and he founded the new \textbf{Fifth Republic}.
He introduced his ``Politics of Grandeur'' and demanded that France be given complete autonomy in its affairs.
He withdrew France from NATO's military command, because he thought that the US had too much control.
In 1965, de Gaulle helped start the Empty Chair Crisis,
which involved financing the Common Agricultural Policy of the European Economic Community (EEC).
The crisis was resolved when the Luxembourg compromise was reached in January 1966.

On a visit to Canada in 1967, de Gaulle voiced support for a free Quebec,
declaring, ``vive le Quebec libre'' (long live free Quebec).

The new president also had to deal with the war in Algeria.
After visiting Africa, he decided that he supported independence for Algeria,
and neutralized the army that was stationed there.
In 1959, he gave the country self-determination, and there followed a revolt by the French settlers.
In 1962, he signed the \textbf{Evian Accords},
creating a ceasefire in Algeria and giving victory to the FLN (National Liberation Front in Algeria).
This resulted in several assassination attempts against him by the OAS (the settlers' resistance group),
including one in which he narrowly escaped machine gun fire in a limousine.

In May 1968, protests broke out against de Gaulle's government.
His OTRF broadcasting organization had a monopoly on TV and radio.
On May 29, de Gaulle disappeared without telling anyone.
He went to Baden-Baden, from whence he returned with the military's support.
During the revolts, the Grenelle Agreements were signed, increasing the minimum wage,
but failing to resolve the conflict.
Eventually, de Gaulle's time as a leader was up and \textbf{George Pompidou} took over running the country.


\section{Cold War Britain}

\section{Eastern Bloc}

\subsection*{Czechoslovakia}
% Prague Spring

\subsection*{Hungarian Revolution}

\subsection*{Polish Solidarity}

\section{The Balkans}

\subsection*{Yugoslavia}

\subsection*{Bosnian War}
% Sniper Alley
% Dayton Accords
% Siege of Sarajevo

