\chapter{Recent European History}

\section{Republics of France}

\subsection*{Fourth Republic}

On October 13, 1946, a new \textbf{Fourth Republic} established itself in France.
The French Colonial Empire proceeded to fall apart.
French Indochina was lost at the Battle of \textbf{Dien Bien Phu},
falling into Viet Minh hands, led by Ho Chi Minh (more on that later).

In 1956, France faced the \textbf{Suez Crisis}.
France had built the Suez Canal, and therefore owned the Canal.
When Egyptian President \textbf{Gamal Abdel Nasser} (more in the later chapters)
decided to nationalize the Canal, Britain and France attacked.
Isreal had also allied with Britain and France in the secret Protocol of Sevres.
It started when Israel, aiding the British and French, began Operation Kadesh.
They invaded Port Said as part of Operation Musketeer.
Eisenhower would put a stop to the fighting.

\subsection*{President de Gaulle}

Shortly after the Suez Crisis, Guy Mollet was forced out of the post of Prime Minister.
By 1958, the Fourth Republic was falling apart under Rene Coty.
French army units took power in Algiers in May and the whole thing fell apart.

Into this crisis rose \textbf{Charles de Gaulle}, who had led the Free French in World War~II\@.
The National Assembly put de Gaulle in power and he founded the new \textbf{Fifth Republic}.
He introduced his ``Politics of Grandeur'' and demanded that France be given complete autonomy in its affairs.
He withdrew France from NATO's military command, because he thought that the US had too much control.
In 1965, de Gaulle helped start the Empty Chair Crisis,
which involved financing the Common Agricultural Policy of the European Economic Community (EEC).
The crisis was resolved when the Luxembourg compromise was reached in January 1966.

On a visit to Canada in 1967, de Gaulle voiced support for a free Quebec,
declaring, ``vive le Quebec libre'' (long live free Quebec).

The new president also had to deal with the war in Algeria.
After visiting Africa, he decided that he supported independence for Algeria,
and neutralized the army that was stationed there.
In 1959, he gave the country self-determination, and there followed a revolt by the French settlers.
In 1962, he signed the \textbf{Evian Accords},
creating a ceasefire in Algeria and giving victory to the FLN (National Liberation Front in Algeria).
This resulted in several assassination attempts against him by the OAS (the settlers' resistance group),
including one in which he narrowly escaped machine gun fire in a limousine.

In May 1968, protests broke out against de Gaulle's government.
His OTRF broadcasting organization had a monopoly on TV and radio.
On May 29, de Gaulle disappeared without telling anyone.
He went to Baden-Baden, from whence he returned with the military's support.
During the revolts, the Grenelle Agreements were signed, increasing the minimum wage,
but failing to resolve the conflict.
Eventually, de Gaulle's time as a leader was up and \textbf{George Pompidou} took over running the country.

\section{United Kingdom: The Commonwealth}

Important things happened between the end of the Second World War and the late 1960s,
but I'm not going to talk about them.

\textbf{The Troubles} started in 1968 as a conflict between republicans and unionists in Northern Ireland.
They started when a series of riots broke out in Londonderry.
During the Troubles, ten inmates at Maze Prison, including Bobby Sands,
starved themselves to death.
The Troubles were ended by the \textbf{Good Friday Agreement} in 1998,
which came about as the result of peace talks chaired by George Mitchell.

In the 1970s, the UK went through some rough economic times.
Labour had returned to power in 1974 under Harold Wilson.
The economy got worse until Wilson was replaced by James Callaghan.
But, he presided over the Winter of Discontent and he was voted out with no confidence.

In 1979, Conservative \textbf{Margaret Thatcher}, the Iron Lady, was elected prime minister.
Philosophically, Thatcher held views similar to Ronald Reagan and Brian Mulroney (elected in Canada in 1984).
Her Conservative Party was made up of ``wets'' and ``dries''.
After the Maze Prison hunger strike, Sinn Fein started to come back into power.
In 1981, the royal family was brought back into the limelight when
\textbf{Prince Charles} married \textbf{Princess Diana}, and everyone cared.

In Argentina, an unstable military junta under Leopoldo Galtieri had taken power.
On April 2, 1982, Argentina invaded the British-controlled Falkland Islands, starting the \textbf{Falkland War}.
In response, the British sent a naval task force to the islands.
They opened with the Black Buck Raids and the Raid on Pebble Island,
and proceeded to try Operation Mikado, which was aborted.
After an amphibious landing at Port San Carlos, they won the Battle of Goose Green.
A big chunk of Argentine casualties occurred when the British sunk ARA \textit{General Belgrano}.
The British used Ascension Island as a staging ground, where they used US technology.
The whole war lasted over seventy days, and resulted in British victory and restoration of the status quo.

In other foreign affairs, Thatcher visited Deng Xiaoping in China
and negotiated the peaceful transfer of Hong Kong back to China in 15 years.
She also set up the creation of Zimbabwe from the former state of Rhodesia.

At home, Thatcher faced a National Union of Mineworkers strike led by Arthur Scargill.
During the Westland affair,
Thatcher and her Defence Minister Michael Heseltine dealt with the rescuing of Westland Helicopters.
The incident embarrassed the government, hurt Thatcher's public reputation, and led to Heseltine's resignation.
At one point, she survived an IRA assassination attempt by Patrick Magee and his men
in which they tried to blow her up in the Grand Hotel in Brighton.
When she tried to institute the Community Charge (essentially a poll tax)
her popularity vanished and she was replaced by \textbf{John Major}.

\section{Eastern Bloc}

\subsection*{Czechoslovakia}

After World War~II, Czechoslovakia expelled a bunch of Germans who'd been living in its borders.
The Third Republic began, but the country became part of Stalin's sphere of influence.
In February 1948, Communists took over the country, and Edvard Benes came to power.
In June 1953, strikes in Plzen were broken up without too much blood,
disappointing Allen Dulles, who'd wanted an excuse for the CIA to help the people resist Soviet control.
The 1960 Constitution proclaimed the victory of socialism, and created the Czechoslovak Socialist Republic.

In January 1968, \textbf{Alexander Dubcek}, a Slovak reformer,
was elected first secretary of the KSC (Communist Party of Czechoslovakia).
He replaced Antonin Novotny as Novotny also ceded the presidency to Ludvik Svoboda.
Dubcek started to enact a new reform movement in the spring of 1968, which would be known as \textbf{Prague Spring}.
On April 5, he set up the Action Programme, which guaranteed freedom of religion, press, assembly, and speech,
calling it ``socialism with a human face''.
During Prague Spring, Ludvik Vaculik published his ``Two Thousand Words'' manifesto.
Other Warsaw Pact countries weren't particularly happy about what was happening in Prague Spring,
so they invaded Czechoslovakia in August.
Dubcek was arrested and taken to Moscow, where he negotiated the \textbf{Brezhnev Doctrine},
allowing for Czechoslovakia to become partially sovereign, with the KSC gaining power.
Notably, in January 1969,
Jan Palach set himself on fire in Prague to protest the invasion of the country and the end of Prague Spring.
By April 1969, Dubcek was replaced by \textbf{Gustav Husak}.

Skip forward a bit to 1989, when the anti-Communist revolution started.
Between November and December of that year, the non-violent \textbf{Velvet Revolution} took place.
The main anti-communist party was Charter 77, a civic initiative that managed to overthrow the communist regime.
Former playwright \textbf{Vaclav Havel} was elected the first president of Czechoslovakia in 41 years.
Havel was reelected and became the first president of the new Czech Republic,
after it split from Slovakia following the Velvet Divorce in 1993.

\subsection*{Communism in Hungary}

After the war, Hungary had passed a new constitution, and socialism was set up as the main governmental goal.
The People's Front, led by Matyas Rakosi, took power and declared all other parties illegal.
In May 1949, Laszlo Rajk of the ``Hungarian'' Communists, was arrested and put on trial.
He was found guilty and executed, and Rakosi proceeded to impose full totalitarian rule, executing thousands.
His secret police, led by Gabor Peter, persecuted anyone that Rakosi called an enemy.
Rakosi fell in 1956.

On October 23, 1956, a student demonstration in Budapest began.
The protests got attention from the police, who opened fire on the students,
sparking the \textbf{Hungarian Revolution of 1956}.
Stalin's statue was toppled, and people wanted Erno Gero to step down from power.
At two in the morning the following day, Soviet tanks entered Budapest.
On the 25th, they opened fire on protesters,
and the Central Committee forced Gero to resign, replacing him with \textbf{Janos Kadar}.
At this point, \textbf{Imre Nagy} and a group of supporters took over the Hungarian Working People's Party.
Nagy freed Cardinal Josef Mindszenty, among other political prisoners.
On November 1, Nagy announced that he was going to withdraw from the Warsaw Pact.
Then, Khrushchev intervened.
Soviet tanks took airfields and subdued the Hungarian forces.
Nagy was arrested and replaced by Janos Kadar, a Soviet loyalist.

Kadar proceeded to lead an attack on the revolutionaries, imprisoning or killing them.
He then declared general amnesty.
By the late 1980's, Kadar's government was poised to become a Western-style democracy.
Kadar was replaced in 1988 by Imre Pozsgay.

\subsection*{Polish Republic}

The Polish People's Republic was created under the communist Polish United Workers' Party,
and the new name was adopted in 1952.
Under the Stalinist regime, thousands were tried and executed.
The Catholic Church was persecuted after Stalin died,
Cardinal Stefan Wyszynski was detained, and there was a show trial of the Krakow Curia in January 1953.
When the Warsaw Pact formed in 1955, the Polish People's Republic had the second largest army.

As de-Stalinization started in 1956, riots broke out in Poznan.
During the \textbf{Polish October},
the new Polish Party's First Secretary \textbf{Wladyslaw Gomulka} liberalized life.
This Polish Thaw started the country in the ``way to socialism''.
During the 1968 political crisis, inspired in part by Prague Spring,
Polish opposition set up protests, prompting a crackdown by the authorities.

In the early 1970's, Poland's economy was in a good place, but the 1973 oil crisis caused a recession.
Edward Gierek, who had replaced Gomulka, tried to help the economy, but he wasn't successful,
and 1976 led to protests against him.
In 1978, Cardinal Karol Jozef Wojtyla became \textbf{Pope John Paul~II}, improving Catholic morale.
By 1979, economic growth was tanking and foreign debt was going straight up.

In 1980, the government tried to increase meat prices, but that resulted in general strikes in Lublin.
Then, protests at the \textbf{Gdansk Shipyard} started a reaction of strikes that stopped a lot of progress.
Workers led by \textbf{Lech Walesa}, a former electrician who had led the strike at the Lenin Shipyard,
signed the Gdansk Agreement that ended the strike.
Now, a bunch of unionizing movements swept over the country.
On September 17, they got together in Gdansk, led by Walesa,
and created a new national union organization called \textbf{Solidarity}.
In 1981, at the first Solidarity National Congress, Walesa was elected the national chairman of the Union.

In December, the regime began martial law in the country, and ZOMO riot police were deployed to stop Solidarity.
The Military Council of National Salvation was put in charge,
as a basic appearance of political stability had been attained.
When Gorbachev came to power, his reforms helped Poland.
Then, the Round Table Negotiations and the election of 1989 led to the fall of communism.

The Round Table Agreement set up local self government,
and the legislature (the Sejm) was split based on the new agreement.
After some work, Lech Walesa was elected president of the Third Polish Republic in November 1990.

\section{The Balkans}

\subsection*{Yugoslavia}

Yugoslavia had been created following the First World War,
when it was called the Kingdom of Serbs, Croats and Slovenes.
But they decided that name was too long, and in 1929 it was renamed to the Kingdom of Yugoslavia.
It was invaded during World War~II, and after the war, it was known as the Federal People's Republic of Yugoslavia.
A communist regime was put in place.

At the head of the country was the Partisan \textbf{Josip Broz Tito}.
Tito took down the Chetnik (opposing party) leader Draa Mihailovic, and he set up the UDBA secret police.
Yugoslavia was expelled from Cominform in 1948 following Tito's notable \textbf{split with Stalin},
and he became the First Secretary General of the Non-Aligned Movement,
which he helped created with Gamal Abdel Nasser and Jawaharlal Nehru (more on that later).
Tito signed the Treaty of Vis with Ivan Ubaic, and his British liaison was Fitzroy Maclean.
After imprisoning Aloysius Stepinac for helping the Ustashi movement against him,
Tito was excommunicated by the pope.
In 1974, Tito called himself ``President for Life'', promoting ``brotherhood and unity'' in Yugoslavia.

Tito died in 1980.
This led to some internal strife in Yugoslavia, and the country dissolved by 1991, leading to the Yugoslav Wars.

\subsection*{Bosnian War}

I'm going to go a little further than we have in other sections.
The \textbf{Bosnian War} took place between 1992--1995, in Bosnia and Herzegovina.
It came about as a result of the dissolution of Yugoslavia.
When Bosnia and Herzegovina declared independence,
the Bosnian Serbs, supported by \textbf{Slobodan Milosovic} and his Serb forces,
mobilized inside the country and started a war that raged across all of Bosnia and Herzegovina.

On one side was the Serb Army of the Republika Srpska (VSR),
and on the other was the Army of the Republic of Bosnia and Herzegovina (ARBiH).
Also in the mix were the Croats, who wanted to make parts of Bosnia and Herzegovina part of Croatia.

The war was full of fighting, unchecked shelling of populated areas, ethnic cleansing, and mass rape.
These atrocities were especially exemplified by the Siege of Sarajevo
and the Srebrenica Massacre, during which Ratko Mladic's troops killed over 8,000 men and boys.
During the war, Sarajevo was home to Sniper Alley, a major street in the city,
where signs reading ``PAZI'' warned about the dangers inherent in using the road.
After the massacres, NATO forces intervened in 1995.
The war included Operation Deliberate Force and Operation Deny Flight,
and the latter resulted in the Banja Luka incident,
in which six Serb attack jets were shot down by American F-16s.

Peace negotiations in Ohio resulted in the \textbf{Dayton Agreement}, which was finalized in November 1995.
The treaty that was signed by Alija Izetbegović, Franjo Tuđman, and Bill Clinton.
