\chapter{From B.C. to A.D.: The East}

\section{India}

Recall that there was some burgeoning civilization in India.
Around the 4th century BC, more centralized dynasties with influential impact started taking power.
Here we go over some of the more important ones.

\subsection*{Maurya Dynasty (324 -- 184 BC)}

The dynasty was founded by \textbf{Chandragupta Maurya}.
He defeated the Nanda Empire and the Seleucid Empire (Macedonian),
and having defeated Alexander the Great, he was able to unify India.

Chandragupta's grandson,
\textbf{Ashoka} (273--232 BC) was considered perhaps the greatest ruler of ancient India.
He took the throne after killing some relatives (but spared his brother, Tissa).
He conquered the Kalinga empire, an action that changed him forever,
since he then adopted Buddhism as his new religion.
After this conversion, he put up edicts on columns all over the place
promoted non-violence and animal rights,
and became generally a pacifist ruler.
One of those columns is at Sarnath, and it has four lions on it,
as well a 24 spoke wheel of righteousness, which would come to be placed on the Indian flag.

\subsection*{Gupta Dynasty (240 -- 550)}

\textbf{Chandra Gupta II} was the most important ruler during this time,
and the Gupta Empire was biggest under him.
Besides that, not a lot of hugely notable things happened around this time.

There would follow periods of India being invaded repeatedly by people from
all over the place, including Persia, Scythia, Parthia, and more.

So, until we get to a point in history where India starts to interact with the
rest of the world more, here's a brief overview of some Chinese dynasties, starting in ancient times.

\section{China}

The history of China until modern times is largely one of dynasties.
So, here are the most important dynasties.

\subsection*{Shang Dynasty (1600 -- 1046 BC)}

The Shang is the first real Chinese dynasty with written records.
The writings are found on ``oracle bones'',
which were bones that were heated and used to tell the future (pyromancy).
The capital was near Anyang, and the Shang dynasty succeeded the lesser known Xia dynasty.
The emperor Wu Yi was killed by a bolt of lightning,
which is funny because he was quite vocal about not liking the god of lightning and thunder very much.
Other notable emperors include Wu Ding and Fu Hao.
The emperor Xi Din lost the battle of Muye and killed himself,
leading to the Zhou dynasty ruling.

\subsection*{Zhou Dynasty (1046 -- 256 BC)}

The Zhou was the longest lasting period of Chinese history.
During the Zhou, the use of iron began.
When the Zhou capital was sacked, the Eastern Zhou and the Spring and Autumn period began.
Intellectualism flourished at this point:
Sun Tzu wrote \textit{The Art of War},
and the Hundred Schools of thought (one of which was Confucianism) were introduced.

However, during the Zhou dynasty, real power rested with the feudal lords, so
by the end of this really long (nearly 800 years) dynasty,
everything fell apart into seven separate warring states.
When the fighting had settled, the state of Qin was victorious.

\subsection*{Qin Dynasty (221 -- 206 BC)}

A short, but important dynasty, consisting mostly of one emperor who did anything: \textbf{Qin Shi Huangdi}
He burned a lot of books, so people tend to not think too highly of him.
He also standardized weights, and unified Chinese writing,
and begin building the Great Wall (you may have heard of it).
Along with his minister Li Si, he implemented Legalism.
When he died, he was buried near Xi'an with a massive and famous terra cotta army.
Qin had created the first true unified China.

With Qin Shi Huangdi's death, the Qin dynasty was mismanaged into failure, and it collapsed quickly.

\subsection*{Han Dynasty (206 BC -- AD 220)}

A golden age of China that had a lasting impact, the Han dynasty names China's predominant ethnic group today.
The emperor \textbf{Gaozu} (aka Liu Bang) founded the dynasty, even though he was born as a peasant.
He reunited China and made a new capital at Chang'an (now Xi'an).

One of the greatest rulers of China was \textbf{Wudi}, who expanded China greatly.
He reformed government and he made Confucianism the new official doctrine.
An official named Wang Mang temporarily overthrew the Han dynasty
to try and make the Xin dynasty, but it didn't take.
Wang Mang died at Chang'an, and Emperor Guangwu took back the throne

By the end, it had faced the \textbf{Yellow Turban Rebellion},
a Taoist peasant revolt during the reign of Emperor Ling.
The Yellow Turban Rebellion was the opening to the Romance of the Three Kingdoms.

\subsection*{Period of the Three Kingdoms (184 -- 280)}

A short and violent period in history.
The classic novel \textit{Romance of the Three Kingdoms} ensured that it had a big impact.
The three kingdoms were Wei, Shu, and Wu:
Cao Wei north of the Yangtze,
Eastern Wu in the lower Yangtze,
and Shu Han in Sichuan.

The Battle of Red Cliffs (208) was fought between Liu Bei and Sun Quan,
(although admittedly it overlapped a bit with the Han dynasty).
It set up the basis for the states of Shu and Wu.

Eventually, Cao Wei managed to defeat the other kingdoms, but failed to reunify China.
China ended up going through a short period of Southern and Northern dynasties,
but not a lot of hugely important things happened during that time, so let's skip to the Tang dynasty.

\subsection*{Tang Dynasty (618 -- 907)}

The Tang dynasty succeeded in reuniting China after the Sui dynasty (very short dynasty) fell.

Another golden age of China: important people such as \textbf{Li Po} and Du Fu worked,
and the printing press was invented.
The Tang dynasty moved the capital back to Chang'an and was also ruled by \textbf{Gaozu}.
His son, emperor Taizong (aka Li Shimin), deposed him and became a great ruler,
consolidating power fairly effectively and conquering a lot more of western China.
He was succeeded by \textbf{Wu}, the only woman to be emperor of China,
presiding over the ``second Zhou dynasty''.

Later, during the reign of Xuanzong, the \textbf{An Lushan} rebellion tore apart the Tang.
Even though it was quelled, the revolt had lasting effects that ended up beginning the
periods of the five dynasties and the Ten Kingdoms.

\subsection*{Song Dynasty (960 -- 1279)}

Soon afterwards, the Song dynasty began.
This was a time dedicated to culture, not warfare;
to engineering, not killing.
Gunpowder was discovered, the first compass was made,
a standing navy was established, and paper money was circulated.

The first ruler was Taizu, and he realized that he could save his own neck if he
asked all the generals and military people around China to retire.
This resulted in scholars being far more dominant during the time.

The capital was moved to Kaifeng, which was captured by the Jin dynasty from the north.
So, the Song fled down south and made a new Southern Song dynasty, capital at Hangzhou,
and the Jin didn't bother them any more.
But, then the Mongols came down and destroyed the Jin, with the help of the Song.
The Song didn't realize that the Mongols wanted their territory too,
so even though they lasted a while under Mongol attacks, they eventually were defeated.

\subsection*{Yuan Dynasty (1271 -- 1368)}

The Mongols established the Yuan dynasty, most notable for the emperor \textbf{Kublai Khan}.
Kublai tried to conquer Japan, but was deterred by typhoons called the kamikaze, ``divine wind''.

In the 1350's, the White Lotus Society (Buddhists) created an army to go against the Yuan dynasty.
This \textbf{Red Turban Rebellion} would bring about the demise of the Yuan dynasty within a few years.

\subsection*{Ming Dynasty (1368 -- 1644)}

The rulers of the Ming dynasty were the Zhu family.
Everyone knows about their porcelain work, and their vases.
The word ``china'' being used to describe high quality porcelain originated in this dynasty.

The emperor \textbf{Hongwu} founded the dynasty, and he had led the Red Turbans.
The eunuch Zheng He led fleets on treasure voyages, to show off how rich they were.

This is also when China's capital was moved to Beijing.
The Ming dynasty collapsed as a result of a flailing economy,
coupled with the invasion of the Manchu people from the north.

\subsection*{Qing Dynasty (1644 -- 1911)}

The Manchurian Qing dynasty was the last dynasty of China.
They created the banner system, and the emperor Kangxi quelled the Revolt of the Three Feudatories.
The dowager empress Cixi weakened the dynasty, and at one point,
they were also threatened by a White Lotus Rebellion.
Other notable emperors include Guangxu.
As a result of globalization, we'll deal with a lot of this era of Chinese history in later chapters.
