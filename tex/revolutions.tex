\chapter{Revolutions in Europe}

The turn of the 19th century was full of revolution in Europe.
After France's monarchy got all turned upside-down, Napoleon took power,
and rebellions would sweep through the continent.

\section{French Revolution}

The French Revolution was a particularly important event for Europe,
and it was influenced partly by the American Revolution.

\subsection*{Louis~XVI}

\textbf{Louis~XVI} was the grandson of Louis~XV and he came to power in 1774,
when his father, the Dauphin, died.
His wife was \textbf{Marie Antoinette} of Austria.
Louis supported the American Revolution starting in 1776 until their independence.
In the beginning of his reign, he wanted to make France more Enlightened,
and so he tried to abolish serfdom, stop the \textit{taille}, and increase religious tolerance.
His foreign minister Charles Gravier was an important part in Louis's support of the American Revolutionaries.

Louis's financial reforms were carried out by Turgot and Malesherbes, but the nobles weren't happy with that.
Turgot had been appointed by Louis's advisor Maurepas.
So, Louis fired Turgot and replaced him with finance minister \textbf{Jaques Necker}.
But, Louis wasn't happy with his failure and would replace Necker with Charles Alexandre du Calonne.
When Calonne wasn't able to get France out of debt, Louis called the Assembly of Notables.

\subsection*{Tennis Court Oath}

The king was losing power, and people were calling for him to convene the \textbf{Estates General}
(an assembly of the nobility, clergy, and common folk),
which had not been called since 1614 and the reign of Louis~XIII\@.
In 1788, Louis called the Estates General.

The Third Estate wasn't pleased by the fact that they weren't as well off as the nobility and clergy.
So, they split off in June 1789 and declared themselves the \textbf{National Assembly}.
When the Menus Plaisirs at Versailles was locked and the 577 assembly members were left without a place to meet,
they went to an indoor tennis court and took the \textbf{Tennis Court Oath}.
The agreement, written by Jean-Sylvan Bailly and Jean-Joseph Mounier,
promised that they would stay together until a new constitution was created.
Other notables at the Tennis Court Oath were Abbe Sieyes and Mirabeau.
Jacques-Louis David made a sketch of the Oath.

\subsection*{Outbreak of Revolution}

On July 11, 1789, when Jacques Necker was dismissed, people were mad.
Necker had been doing good things for the people.
So, on July 14, some insurgents decided to storm the \textbf{Bastille},
a fortress prison that had become a symbol of the power of the king,
despite only holding seven prisoners at the time.
They went for ammo caches inside the building.
The commander of the Bastille, Bernard de Launay, was killed and his head was put on a pike.
After the storming, Jean-Sylvan Bailly became mayor because Jacques de Flesselles had been killed.
The wave of revolution resulted in the Great Fear.

In August, the \textbf{Declaration of the Rights of Man} was passed,
and it was heavily influenced by the Declaration of Independence in the USA,
where the \textbf{Marquis de Lafayette} had worked with Jefferson.
The National Constituent Assembly then worked to make a new constitution.

In October, there was a \textbf{Women's March} on Versailles,
protesting the high price of bread and bringing many weapons with them.
The mob stormed the palace, killing some guards.
Louis gave in to their demands and moved his family from Versailles to the \textbf{Tuileries} in Paris.

\subsection*{Constitutional Monarchy}

At this point, the \textbf{Jacobin} Club (Society of Friends of the Constitution) was forming.
The Jacobin's would become a more radical group of revolutionaries, in contrast to the \textbf{Girondists}.
Members included \textbf{Maximilien Robespierre} and \textbf{Georges Danton},
who would become quite important in the coming years.
Robespierre wrote the ``Defender of the Constitution'' to oppose war with Austria.

Issues were raised regarding how the new government would govern,
at the meetings that tried to define a new constitution.
It was decided to continue the monarchy, checked by a constitution.
Louis would remain in power, with some weaker power.
These new ideas weren't very popular with the radicals.

\subsection*{Terror and War}

In 1791, Louis~XVI wasn't very happy with the state of the revolution.
So, he decided to take his family and run for the border.
They fled towards Varennes, trying to get to Austria and safety.
But, they were captured, and placed under house arrest in the Tuileries.

In the Brunswick Manifesto, Austria and Prussia demanded that Louis be put back in power.
The Legislative Assembly went to war with Austria and Prussia.
Because they thought that he was conspiring with enemies of France, the people condemned Louis to death.
He was executed by \textbf{guillotine} on the Place de la Concorde in January 1793, stripped of his title.
His final words were drowned out by a National Guard drum roll ordered by Antoine-Joseph Santerre.
Marie Antoinette would be killed in the same way nine months later.

Lots of violence broke out at this point.
The wars weren't going very well, and \textbf{sans-culottes} (without culottes),
poor workers and some radicals, were rioting.

This was the point that the \textbf{Committee of Public Safety} rose to power,
under the leadership of the Jacobins.
Robespierre was the leader of the Committee,
and he ordered executions in large numbers during the ensuing \textbf{Reign of Terror}.
At least sixteen thousand people lost their lives to the guillotine.
Many Girondins were arrested, including Jacques Pierre Brissot.

\textbf{Jean-Paul Marat} was a journalist who was known for his inflammatory rhetoric.
Marat had been a scientist and doctor, but he became the editor of \textit{Friend of the People}.
He wrote \textit{The Chains of Slavery} and \textit{A Philosophical Essay on Man}.
Marat spent a lot of time attacking Girondins, and when he was in his bath,
\textbf{Charlotte Corday}, a Girondin assassin, assassinated him.
His death was painted by Jacques-Louis David.

The \textbf{Vendee Revolt} was a 1792 uprising against the government.
The rebels were people who were loyal to the church and to the king.
This counter-rebellion was suppressed by General Lazare Hoche at battles such as Savenay and Cholet.

Members of the Committee of Public Safety such as Georges Danton were being removed all the time.
Robespierre, ``the Incorruptible'', became more and more powerful, and more and more radical.
A new Constitution of 1793 was adopted; it granted universal male suffrage.
Robespierre, in 1794, created a new religion, known as the \textbf{Cult of the Supreme Being},
which replaced the Cult of Reason.
The French calendar was changed --- the new months were completely different.

\subsection*{The Directory}

In the new month of Thermidor, people stopped loving Robespierre, who had gone off the deep end.
The moderate factions, led by Paul Barras, went and arrested Robespierre at the Hotel de Ville.
As a result of this \textbf{Thermidorian Reaction}, Robespierre was guillotined,
and \textbf{The Directory} rose to power.

The Directory was against the pointless death of the Reign of Terror, but it wasn't effective either.
So, the Army, led by \textbf{Napoleon Bonaparte},
took control of France after deposing them in the Coup of 18 Brumaire.
This established a Consulate, and Napoleon would declare himself Emperor in 1804.

\section{Napoleonic Era}

\section{Europe in Turmoil}

