\chapter{Recent U.S.\ History}

\epigraph{%
  The arc of the moral universe is long, but it bends toward justice.
}{Martin Luther King, Jr.}

\section{Postwar America}

\subsection*{Harry Truman}

Let's talk about Harry S.\ Truman, Roosevelt's final Vice President.
Truman was born in Missouri, and he joined Tom Pendergast's political machine in Kansas City.
As discussed before, Truman assumed the presidency just before the Nazis surrendered,
and he ordered dropping the bombs on Hiroshima and Nagasaki.

Truman ran for reelection in 1948.
His opponent was \textbf{Thomas Dewey}, a New York Republican.
Also running was \textbf{Strom Thurmond}, who had shown up from South Carolina with the support of Dixiecrats,
and formed the States' Rights Democratic Party with his running mate Fielding Wright.
During the election, Truman criticized the ``Do-Nothing Congress'' for opposing his Fair Deal.
On election day, the Chicago Tribune preemptively announced election results: ``Dewey Defeats Truman!''
However, they were wrong because some polls weren't available, and Truman ended up winning reelection.
Truman's new cabinet included Secretary of State \textbf{Dean Acheson}.

\subsection*{I Like Ike}

\textbf{Dwight David Eisenhower} was a five-star general in the United States Army,
serving as Supreme Commander of Allied Forces in Europe during World War~II\@.
In 1952, he entered the Presidential race as a Republican.
His running mate was Richard M. Nixon, and he defeated the Adlai Stevenson/John Sparkman ticket in a landslide.

Eisenhower authorized Operation Ajax,
which deposed \textbf{Mohammad Mosaddegh}, prime minister of Iran, in a 1953 coup.
He also gave the go-ahead on Operation PBSUCCESS, targeting Jacobo Arbenz in Guatemala,
ending the democratic government in Guatemala, ending the revolution,
and putting Carlos Castillo Armas in charge of a new military dictatorship.
His New Look policy was an attempt at nuclear deterrence in the face of the burgeoning Cold War.

In 1947, a young Wisconsin Senator named \textbf{Joseph McCarthy} took office,
having defeated Robert La Follette, Jr in 1946.
McCarthy had flown combat missions for the Marines during the war,
and he turned his nickname ``Tail-Gunner Joe'' into a campaign slogan: ``Congress needs a tail-gunner!''
In 1950, McCarthy came to prominence by giving a speech to the Women's Club of Wheeling, West Virginia:
\begin{quote}
  The bright young men who are born with silver spoons in their mouths are the ones who have been the worst.
  In my opinion, the State Department, which is one of the most important government departments,
  is thoroughly infested with Communists.
  I have in my hand 57 cases of individuals who would appear to be either card carrying members
  or certainly loyal to the Communist Party,
  but who nevertheless are still helping to shape our foreign policy.
\end{quote}
McCarthy quickly became the most well-known face in an anti-Communist witch hunt.
His modern Red Scare went until the Army-McCarthy hearings of 1954,
until the Senate eventually voted to censure McCarthy.
Senator Joseph Welch said:
\begin{quote}
  Senator, may we not drop this?
  We know he belonged to the Lawyers Guild.
  Let us not assassinate this lad further, Senator.
  You've done enough.
  Have you no sense of decency, sir?
  \textbf{At long last, have you left no sense of decency?}
\end{quote}

Eisenhower covertly opposed McCarthy while he was doing his whole crusade.
He was a mostly moderate conservative president, keeping around most of the New Deal programs.
He also launched the \textbf{Interstate Highway System} and DARPA\@.
In 1957, Orval Faubus opposed the enrollment of nine black students into Central High School in Little Rock.
Eisenhower sent in the 101st Airborne to intervene and make sure that they got into school.

In January 1961, Eisenhower gave his farewell address,
written by himself, his brother Milton, and his speechwriter Malcolm Moos:
\begin{quote}
  As we peer into society's future,
  we --- you and I, and our government --- must avoid the impulse to live only for today,
  plundering for, for our own ease and convenience, the precious resources of tomorrow.
  We cannot mortgage the material assets of our grandchildren
  without asking the loss also of their political and spiritual heritage.
  We want democracy to survive for all generations to come,
  not to become \textbf{the insolvent phantom of tomorrow}.
\end{quote}

\section{The Sixties}

\subsection*{Civil Rights}
% TODO Malcolm X
% TODO Booker T Washington

After the Civil War, the Reconstruction amendments fixed some big problems, such as slavery.
However, into the 1950s, segregation was still rampant and racism was common, especially in the South.
Segregation was sanctioned by the government, as shown by the \textbf{Jim Crow system}.
The Supreme Court upheld it in the ``separate but equal'' doctrine
established in \textit{Plessy v.\ Ferguson} (1896).
But, real crux of the story of the African American Civil Rights Movement takes place between 1954 and 1968\ldots{}

In May 1954, the Supreme Court rules on \textit{Brown v.\ Board of Education of Topeka, Kansas}.
You can see overviews of important cases in the appendix, but the decision overturns \textit{Plessy}.

On March 2, 1955, a black woman in Alabama is arrested for not giving up her seat on the bus to a white man.
The ACLU and civil rights leaders quickly mobilize, but then they stop.
The fifteen year old is pregnant and her name is \textbf{Claudette Colvin}.
A pending bus boycott is called off, and leaders decide not to publicize Colvin's story,
even though her court case will eventually end segregation on the buses.

On December 1, \textbf{Rosa Parks}, a seamstress and secretary for the NAACP,
doesn't listen to James Edwards's instructions to leave her seat on a bus and gets arrested.
This finally triggers the \textbf{Montgomery Bus Boycott}.
Leading this boycott is a Baptist minister named \textbf{Martin Luther King, Jr.}
He is elected president of the Montgomery Improvement Association.
People start to listen to Dr.\ King, who rises to prominence at the head of a burgeoning movement for civil rights.

In September 1957, Orval Faubus, governor of Arkansas,
decides to block integration of Little Rock Central High School.
Within two weeks, President Eisenhower federalizes the National Guard and orders the Army into Little Rock.
The \textbf{Little Rock Nine}, a set of students who were to attend the school, get a military escort.
By the end of the month, Eisenhower signs the \textbf{Civil Rights Act of 1957},
in spite of segregationist \textbf{Strom Thurmond},
who tried to stop it by giving the longest one-person filibuster of all time, lasting over 24 hours.

Meanwhile, King founds the \textbf{South Christian Leadership Conference} (SCLC)
and is chosen as its first president.
The SCLC leads the Albany Movement in Georgia against segregation in 1962.
In 1963, King organizes nonviolent protests in Birmingham, garnering national attention.
Alabama is governed by \textbf{George Wallace},
who had called for ``segregation now, segregation tomorrow, segregation forever'' at his inauguration.
The protests result in a massive police response they land King in jail,
where he writes his ``Letter from a Birmingham Jail'':
\begin{quote}
  Moreover, I am cognizant of the interrelatedness of all communities and states.
  I cannot sit idly by in Atlanta and not be concerned about what happens in Birmingham.
  \textbf{Injustice anywhere is a threat to justice everywhere.}
  We are caught in an inescapable network of mutuality, tied in a single garment of destiny.
  Whatever affects one directly, affects all indirectly.
\end{quote}

In August 1963, King organizes the \textbf{March on Washington} for Jobs and Freedom,
with logistical help from Bayard Rustin.
Speakers at the huge event include Walter Reuther, Josephine Baker,
and John Lewis, who criticizes Kennedy in his speech, causing some controversy.
Of course, the most famous speaker is King himself, who gives his ``Normalcy, Never Again'' speech,
saying that he had come to ``cash a check'' for inalienable rights,
and he refused to believe that the bank was bankrupt.
The speech quickly becomes known as the \textbf{``I Have a Dream''} speech.

In 1968, King travels to Memphis, Tennessee, to help striking African American sanitation workers,
who have walked out in protest of Mayor Henry Loeb.
On April 3, he delivers his \textbf{``I've Been to the Mountaintop''} speech:
\begin{quote}
  Well, I don't know what will happen now.
  We've got some difficult days ahead.
  But it doesn't matter with me now.
  Because I've been to the mountaintop.
  And I don't mind.
  Like anybody, I would like to live a long life.
  Longevity has its place.
  But I'm not concerned about that now.
  I just want to do God's will.
  And He's allowed me to go up to the mountain.
  And I've looked over.
  And I've seen the promised land.
  I may not get there with you.
  But I want you to know tonight, that we, as a people, will get to the promised land!
  And so I'm happy, tonight.
  I'm not worried about anything.
  I'm not fearing any man.
  My eyes have seen the glory of the coming of the Lord!
\end{quote}
While staying at the Lorraine Motel in Memphis,
King walks out onto his balcony at 6:01 PM on April 4.
He is struck by a bullet fired into his jaw, and he dies an hour later at St.\ Joseph's Hospital.
The FBI investigation finds the fingerprints of \textbf{James Earl Ray} at the origin of the gunfire.
Two months later, Ray is captured at Heathrow Airport and confesses to having killed King.

\subsection*{JFK}

The 1960 Democratic primary resulted in \textbf{John Fitzgerald Kennedy}
defeating Hubert Humphrey and Lyndon B.\ Johnson.
At the convention, Kennedy gave his ``New Frontier'' speech:
\begin{quote}
  For the problems are not all solved and the battles are not all won ---
  and we stand today on the edge of a New Frontier\ldots{}
  But the New Frontier of which I speak is not a set of promises --- it is a set of challenges.
  It sums up not what I intend to offer the American people, but what I intend to ask of them.
\end{quote}
Upon winning, Kennedy asked Johnson to be his running mate.

The general election was shaping up to be the closest in many years.
Kennedy would be fighting a battle against \textbf{Richard M.\ Nixon},
centering around Cuba, Kennedy's Roman Catholicism, and the space race.
The election brought the first televised presidential debates in history.
Nixon looked uncomfortable with his five o'clock shadow and injured leg,
while Kennedy looked far better to a television audience.

Kennedy ended up being the youngest man elected president.
During his inauguration, he said, ``Ask not what your country can do for you; ask what you can do for your country.''
Kennedy would go on to write \textit{Profiles in Courage}.

Kennedy's Secretary of Defense was \textbf{Robert McNamara}, who ramped up the Vietnam War.
When he was Secretary, McNamara surrounded himself with a set of experts from RAND Corporation,
who were called the \textbf{Whiz Kids}, hearkening back to a group McNamara had been a part of at Ford.
This was all part of McNamara's attempt to try and run the Pentagon like a business.

We've discussed some of what Kennedy did in relation to the Cold War already.
He signed the Nuclear Test Ban Treaty, set up the Peace Corps, and increased US involvement in Vietnam.
The botched Bay of Pigs invasion occurred soon after he took office,
and he was forced to resolve the Cuban Missile Crisis.
In West Germany, he gave his ``Ich bin Ein Berliner'' speech.

On November 22, 1963, Kennedy was traveling through Dallas in a presidential motorcade.
Accompanying him were his wife \textbf{Jacqueline Kennedy}, Texas Governor John Connally, and Connally's wife Nellie.
While driving through Dealey Plaza, shots were fired from the Texas School Book Depository.
The assassination was recorded by \textbf{Abraham Zapruder} on his namesake film.
\textbf{Lee Harvey Oswald} was arrested and charged for killing Kennedy,
but he claimed he was just a patsy.
Two days later, Oswald was killed by \textbf{Jack Ruby}.
Chief Justice Earl Warren chaired the \textbf{Warren Commission} to investigate the assassination.
The commission, which included Gerald Ford, concluded that Oswald was the only assassin,
and the nearby ``grassy knoll'' that some claimed to have significance wasn't important at all.

\subsection*{Lyndon B.\ Johnson}

\textbf{Lyndon Baines Johnson} had been elected as a Texan Senator in 1948.
The election was fraught with controversy,
and Johnson defeated incumbent governor Coke Stevenson in court with help from his friend Abe Fortas.
The situation resulted in Johnson being called ``Landslide Lyndon'' permanently thereafter amidst fraud allegations.
Johnson proceeded to work his way up to majority whip and Democratic leader in the Senate.
The senator was also quite the imposing personality;
he would stand uncomfortably close to people while talking to them, in a strategy called ``The Treatment''.
In 1960, Kennedy chose Johnson as his running mate;
upon Kennedy's assassination, Johnson took the oath of office on board Air Force One.

The 1964 campaign for reelection made Johnson look for a new catchy slogan to encompass his social agenda.
He settled on ``The Great Society''.
The plan discussed urban improvements, a war on poverty, educational reform, and crime control.
During election season, Johnson aired the ``Daisy Ad'' against his opponent \textbf{Barry Goldwater},
claiming that a vote for Goldwater is a vote for the nuclear annihilation of children:
``the stakes are too high for you to stay home [and not vote]''.

Johnson passed the Economic Opportunity Act, creating the Job Corps and Head Start.
He appointed Abe Fortas and Thurgood Marshall to the Supreme Court.
Robert Weaver became the first African-American cabinet member in the new Housing and Urban Development Department.

\section{Nixon and Carter: The 1970s}

\subsection*{Rise of Nixon}

A front runner in the Democratic primary campaign in 1968,
\textbf{Robert Francis Kennedy} had previously served on the Senate Labor Rackets Committee.
He had challenged the Teamsters under \textbf{Jimmy Hoffa} and written \textit{The Enemy Within}.
Robert Kennedy had served on his brother's cabinet as Attorney General,
advising the president during the Cuban Missile Crisis.
He'd delivered a speech in Indianapolis following the assassination of Martin Luther King.
In June 1968, RFK won the California primary, defeating Senator Eugene McCarthy.
He was celebrating in the Ambassador Hotel,
where Palestinian \textbf{Sirhan Sirhan} shot him.
Sirhan was subdued by bodyguards including Rafer Johnson and Rosey Grier in the Embassy Room.

\textbf{Richard Milhous Nixon} had served as Vice President under Eisenhower.
In response to slush fund allegations,
Nixon had delivered the ``Checkers Speech'', discussing a dog he'd gotten as a present.
Previously, he'd also defeated Helen Douglas for his Senate seat,
claiming she was ``pink right down to hr underwear'', calling her ``the Pink Lady''.
He'd lost to Kennedy back in 1960.
He proceeded to fail to win the California gubernatorial election in 1962,
and he decided to try for the presidency again in 1968.
His opponent was \textbf{Hubert Humphrey} running with Edmund Muskie.
Humphrey was the incumbent Vice President, hailing from Minnesota.
Nixon crushed Humphrey in the election, getting himself the presidency.

Nixon signed the Anti-Ballistic Missile Treaty.
He visited China with his First Lady Pat, and he ended the draft and proceeded to pull troops out of Vietnam.
In 1972, he was reelected in a huge landslide over opponent \textbf{George McGovern}
and his running mate \textbf{Sargent Shriver} (who had replaced Thomas Eagleton).

\subsection*{The Watergate Affair}

In 1967, Robert McNamara had prepared a set of documents on the military presence in South Vietnam,
and had sent them to RAND Corporation.
These documents would eventually come to be known as the \textbf{Pentagon Papers}.
A military analyst named \textbf{Daniel Ellsberg} opposed the war,
and when he discovered the papers, decided that the American public should see them,
and gave them in March 1971 to the New York Times.
Attorney General John Mitchell cited the Espionage Act and tried to arrest Ellsberg for the release,
but the papers ended up staying public.

Nixon's top advisors at this time included Chief of Staff H.R. Haldeman,
John Ehrlichman, and John Mitchell, and John Dean.
They decided to set up a secret organization to fix leaks, the White House Plumbers.
One of their first jobs was to break into the office of Lewis Fielding, Ellsberg's psychiatrist.
Also on the Plumbers were Howard Hunt and \textbf{G.\ Gordon Liddy}.

At this point, the Republicans wanted to run against McGovern,
because it seemed like he'd be easier to beat than Muskie.
The Republicans had set up the \textbf{Committee to Re-Elect the President} (CREEP).

Liddy came up with an idea to burglarize the \textbf{Watergate complex}.
They were to break into DNC headquarters and bug the telephones.
People recruited into the operation included James McCord and Bernard Barker.
But, the DC Police caught the burglars, and they were hauled in front of Judge John Sirica.
Now, the FBI started investigating,
and it was suggested to Nixon to put an end to this because it was a security risk.

At this point, McGovern became the Democratic candidate, and he botched his run.
He ended up winning only Massachusetts and DC, and got trounced by Nixon.

Now, the White House was denying any connection to the Watergate break-in.
The people who had actually broken in weren't particularly happy about this.
Then, John Dean decided to go and talk about how Haldeman, Ehrlichman,
and the President had been part of the cover up.
Now, CREEP, Nixon, and everyone was being investigated by the grand jury, the FBI, and by the media.

Chief among the reporters looking into Watergate were two \textit{Washington Post} reporters
named \textbf{Bob Woodward and Carl Bernstein}.
They were helped by a mysterious informant named \textbf{Deepthroat}
(now revealed to be \textbf{Mark Felt}, the \#2 man at the FBI).
There was also a special prosecutor named Archibald Cox, who was looking into CREEP and friends.
CREEP money was supposedly being channeled into covert operations.

When it was revealed that Nixon had been recording all the conversations he'd had,
everyone wanted the tapes.
Of course, Nixon didn't want to give the tapes away.

Nixon tried to get Attorney General Elliot Richardson to get rid of Cox,
but Richardson didn't want to, and he resigned.
Deputy Attorney General William Ruckelshaus did the same,
and then Nixon turned to Solicitor General \textbf{Robert Bork}.
Bork determined that it was legal to do what Nixon was asking, Cox was sent away, and his office was sealed off.
This series of events came to be known as the \textbf{Saturday Night Massacre}.
Nixon was forced to put Leon Jaworski into the Cox's old position.

Then, Nixon decided to give some transcripts of the tapes over.
This didn't work for very long,
and the Supreme Court decided in United States v. Nixon that he had to release them all.
When people heard the tapes, it was fairly obvious that the Nixon administration was wholly corrupt.
Pressure against Nixon grew until he was about to be impeached, and he resigned.

\subsection*{Gerald Ford}

When Spiro Agnew resigned as Nixon's Vice President,
Nixon appointed \textbf{Gerald Ford} (born Leslie Lynch King Jr.) to the post.
On Nixon's resignation, Ford became president,
making him the only person to have served in both the vice presidency and the presidency
without having been elected to either post.
A month after Nixon's resignation, Ford pardoned him, and he also pardoned Vietnam war draft dodgers.

In 1975, in the span of three weeks, Ford had two attempts on his life.
Sara Jane Moore and Lynette ``Squeaky'' Fromme,
the only two women to attempt presidential assassination, tried to kill Ford.

Ford's economic policy was called ``whip inflation now''.
Donald Rumsfeld served as both his Chief of Staff and Secretary of Defense.
He also retained \textbf{Henry Kissinger} as Secretary of State from the Nixon administration.
Ford signed the \textbf{Helsinki Accords} in 1975, attempting to improve relations with the Communist bloc.
During the Mayaguez incident (the last official battle of the Vietnam War),
Ford sent Marines to retake the \textit{Mayaguez}, a merchant ship near Cambodia.

\subsection*{Jimmy Carter}

At the start of the 1976 campaign, Ford had to overcome opposition in his own party.
He defeated Ronald Reagan, a former California governor, and got the nomination at the convention,
selecting Bob Dole as his running mate.
His opponent was \textbf{Jimmy Carter}, a peanut farmer from rural Georgia who served as governor until 1975.
Apparently when Carter told his mother he was running for president, his mother asked, ``President of what?''
Carter's running mate was Walter Mondale,
and he beat Carl Sanders (``Cufflinks Carl'') in the primary.
During the campaign, Ford famously claimed,
``There is no Soviet domination of Eastern Europe and there never will be under a Ford Administration''.
The election ended up being fairly close but Carter won.

On his second day in office, Carter pardoned Vietnam War draft dodgers.
He created the Department of Energy.
He established the Department of Education, appointing Shirley Hufstedler as its first secretary.
He signed the Torrijos-Carter Treaties, guaranteeing that Panama would own its Canal by 1999.
Carter's Secretary of State was Cyrus Vance, but when Vance resigned,
Carter replaced him with Edmund Muskie.
At the urging of his National Security Advisor Zbigniew Brzezinski, Carter established his ``Carter Doctrine'',
stating that the US would use military force to defend the Persian Gulf.

In March 1979, a partial meltdown at \textbf{Three Mile Island} almost caused an evacuation of Harrisburg, PA\@.
The meltdown was partially blamed on a maintenance bag that covered up information from Met Ed employees.
While the energy crisis was setting in, Carter gave his ``malaise speech'':
\begin{quote}
 The threat is nearly invisible in ordinary ways.
 It is a \textbf{crisis of confidence}.
 It is a crisis that strikes at the very heart and soul and spirit of our national will.
 We can see this crisis in the growing doubt about the meaning of our own lives
 and in the loss of a unity of purpose for our nation.
\end{quote}

\section{Republicans in Power}

\subsection*{Ronald Reagan}

The incumbent Jimmy Carter went up against \textbf{Ronald Reagan}, former actor and governor of California,
and independent John B.\ Anderson in the election of 1980.
As a result of the failure to deal with the Iran Hostage Crisis and a failing economy,
Reagan won the election in a landslide, with Carter only winning 6 states and the District of Columbia.
Simultaneously, Republicans took control of the Senate for the first time in 28 years.

Reagan's ``voodoo economics'' had been attacked by his Vice President,
\textbf{George H.W. Bush}, during the primaries.
Reagan signed the Kemp-Roth tax cut, and his director of the OMB was David Stockman,
who helped implement his supply-side economics.
While Reagan was in office, he dealt with the air traffic controller strike.

In March 1981, \textbf{John Hinckley, Jr.} tried to kill Reagan at the Hilton Hotel in Washington, D.C.
Hinckley was obsessed with Jodie Foster, who'd starred in \textit{Taxi Driver} just a few years prior, writing:
\begin{quote}
Over the past seven months I've left you dozens of poems, letters and love messages
in the faint hope that you could develop an interest in me.
Although we talked on the phone a couple of times
I never had the nerve to simply approach you and introduce myself\ldots{}
The reason I'm going ahead with this attempt now is because I cannot wait any longer to impress you.
\end{quote}
All targets survived.
Hinckley was found not guilty by reason of insanity.

In 1983, Reagan ordered US forces to invade Grenada.
Codenamed \textbf{Operation Urgent Fury},
the goal was to intervene with the Marxist-Leninist government under Maurice Bishop that had taken control in 1979.
Bishop had been killed six days prior and hidden in a dumpster by Bernard Coard's new JEWEL faction.
Reagan also cited a threat to a few hundred medical students at St.\ George's University.
The invasion was a success, resulting in a new government and Bishop's body

In 1984, Reagan and Bush utterly demolished Mondale and Geraldine Ferraro.
Mondale won only Minnesota and DC, and Reagan was reelected.

Of course, in Reagan's second term, he dealt with media exposure of the \textbf{Iran-Contra Affair}.
Here's how it happened.

At this point in time, The \textbf{Contras} in Nicaragua were fighting against the \textbf{Sandinistas}.
The Sandinistas, under Daniel Ortega, had taken power a few years ago,
and since they leaned very far to the left,
Reagan wasn't overly fond of them, and he wanted to help the Contras.
Congress didn't want to fund the Contras,
so they passed the \textbf{Boland Amendment} preventing sending government money to them.

Seven Americans were taken hostage in Lebanon by Hezbollah.
Iran was currently involved in the Iran-Iraq War (more on that later).
In order to free them, Israel was meant to send weapons to Iran, with the US resupplying Israel.
So, this was a fairly straightforward arms-for-hostages situation, with the Americans using Israel as a proxy.

Then, National Security Council military political liaison named \textbf{Oliver North}
diverted funds from the weapon sales to fund the \textbf{Contras}.
So, the money was being funneled in a way that tried to bypass the Boland Amendment.

In November 1986, the story was leaked by the Lebanese magazine Ash-Shiraa.
This turned into a Congressional investigation, and Reagan claimed no knowledge of the affair.
Defense Secretary \textbf{Caspar Weinberger} wrote that Reagan knew that the hostage-for-arms deal was happening,
In March 1987 Reagan took full responsibility for the bad stuff that happened during the scandal.

A few investigations into the affair were conducted, notably one by Congress,
and one by the Reagan appointed \textbf{Tower Commission} (John Tower, Brent Snowcroft, and Edmund Muskie).
Reagan wasn't implicated in any criminal acts in the end,
but fourteen officials including Weinberger were indicted, and eleven were convicted,
including National Security Advisor John Poindexter.
During the trials, Fawn Hall, North's secretary, was granted immunity for her testimony.
During his state of the union, Reagan said of the affair: ``Mistakes were made.''

\subsection*{H.W.}

Incumbent Vice President \textbf{George Herbert Walker Bush} and \textbf{Dan Quayle} of Indiana
won the 1988 election in a landslide,
defeating Democratic candidate \textbf{Michael Dukakis}.
On accepting the nomination, Bush had given his ``thousand points of light'' speech.
Bush campaigned largely on his economic policy: ``Read my lips --- no new taxes.''
This made Bush the first sitting vice president to be elected to the presidency since Van Buren back in 1836.

Before he'd been vice president, Bush had been the RNC chairman during Watergate and he'd directed the CIA\@.
As president, he signed the Americans with Disabilities Act into law.
He launched Operation Restore Hope to give aid to Africa.
Bush also launched \textbf{Operation Just Cause},
which ousted Manuel Noriega from his position in Panama (more on that later).
While in office, he appointed David Souter and Clarence Thomas to the Supreme Court.
When Saddam Hussein invaded Kuwait in 1991, a coalition was put together to free the country,
and we'll talk the resulting Operation Desert Storm when we get to recent Middle Eastern history.
