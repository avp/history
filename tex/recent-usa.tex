\chapter{Recent U.S.\ History}

\section{Postwar America}

\subsection*{Harry Truman}

Let's talk about Harry S.\ Truman, Roosevelt's final Vice President.
Truman was born in Missouri, and he joined Tom Pendergast's political machine in Kansas City.
As discussed before, Truman assumed the presidency just before the Nazis surrendered,
and he ordered dropping the bombs on Hiroshima and Nagasaki.

Truman ran for reelection in 1948.
His opponent was \textbf{Thomas Dewey}, a New York Republican.
Also running was \textbf{Strom Thurmond}, who had shown up from South Carolina with the support of Dixiecrats,
and formed the States' Rights Democratic Party with his running mate Fielding Wright.
During the election, Truman criticized the ``Do-Nothing Congress'' for opposing his Fair Deal.
On election day, the Chicago Tribune preemptively announced election results: ``Dewey Defeats Truman!''
However, they were wrong because some polls weren't available, and Truman ended up winning reelection.
Truman's new cabinet included Secretary of State \textbf{Dean Acheson}.

\subsection*{I Like Ike}

\textbf{Dwight David Eisenhower} was a five-star general in the United States Army,
serving as Supreme Commander of Allied Forces in Europe during World War~II\@.
In 1952, he entered the Presidential race as a Republican.
His running mate was Richard M. Nixon, and he defeated the Adlai Stevenson/John Sparkman ticket in a landslide.

Eisenhower authorized Operation Ajax,
which deposed \textbf{Mohammad Mosaddegh}, prime minister of Iran, in a 1953 coup.
He also gave the go-ahead on Operation PBSUCCESS, targeting Jacobo Arbenz in Guatemala,
ending the democratic government in Guatemala, ending the revolution,
and putting Carlos Castillo Armas in charge of a new military dictatorship.
His New Look policy was an attempt at nuclear deterrence in the face of the burgeoning Cold War.

In 1947, a young Wisconsin Senator named \textbf{Joseph McCarthy} took office,
having defeated Robert La Follette, Jr in 1946.
McCarthy had flown combat missions for the Marines during the war,
and he turned his nickname ``Tail-Gunner Joe'' into a campaign slogan: ``Congress needs a tail-gunner!''
In 1950, McCarthy came to prominence by giving a speech to the Women's Club of Wheeling, West Virginia:
\begin{quote}
  The bright young men who are born with silver spoons in their mouths are the ones who have been the worst.
  In my opinion, the State Department, which is one of the most important government departments,
  is thoroughly infested with Communists.
  I have in my hand 57 cases of individuals who would appear to be either card carrying members
  or certainly loyal to the Communist Party,
  but who nevertheless are still helping to shape our foreign policy.
\end{quote}
McCarthy quickly became the most well-known face in an anti-Communist witch hunt.
His modern Red Scare went until the Army-McCarthy hearings of 1954,
until the Senate eventually voted to censure McCarthy.
Senator Joseph Welch said:
\begin{quote}
  Senator, may we not drop this?
  We know he belonged to the Lawyers Guild.
  Let us not assassinate this lad further, Senator.
  You've done enough.
  Have you no sense of decency, sir?
  \textbf{At long last, have you left no sense of decency?}
\end{quote}

Eisenhower covertly opposed McCarthy while he was doing his whole crusade.
He was a mostly moderate conservative president, keeping around most of the New Deal programs.
He also launched the \textbf{Interstate Highway System} and DARPA\@.
In 1957, Orval Faubus opposed the enrollment of nine black students into Central High School in Little Rock.
Eisenhower sent in the 101st Airborne to intervene and make sure that they got into school.

In January 1961, Eisenhower gave his farewell address,
written by himself, his brother Milton, and his speechwriter Malcolm Moos:
\begin{quote}
  As we peer into society's future,
  we --- you and I, and our government --- must avoid the impulse to live only for today,
  plundering for, for our own ease and convenience, the precious resources of tomorrow.
  We cannot mortgage the material assets of our grandchildren
  without asking the loss also of their political and spiritual heritage.
  We want democracy to survive for all generations to come,
  not to become \textbf{the insolvent phantom of tomorrow}.
\end{quote}

\section{The Sixties}

\subsection*{Civil Rights}
% TODO Malcolm X
% Booker T Washington

After the Civil War, the Reconstruction amendments fixed some big problems, such as slavery.
However, into the 1950s, segregation was still rampant and racism was common, especially in the South.
Segregation was sanctioned by the government, as shown by the \textbf{Jim Crow system}.
The Supreme Court upheld it in the ``separate but equal'' doctrine
established in \textit{Plessy v.\ Ferguson} (1896).
But, real crux of the story of the African American Civil Rights Movement takes place between 1954 and 1968\ldots{}

In May 1954, the Supreme Court rules on \textit{Brown v.\ Board of Education of Topeka, Kansas}.
You can see overviews of important cases in the appendix, but the decision overturns \textit{Plessy}.

On March 2, 1955, a black woman in Alabama is arrested for not giving up her seat on the bus to a white man.
The ACLU and civil rights leaders quickly rush to help her out, but then they stop.
Her name is \textbf{Claudette Colvin}, she's fifteen years old, unmarried, and pregnant.
A pending bus boycott is called off, and leaders don't publicize Colvin's story,
even though her court case is what will eventually end segregation on the buses.

Nine months later, on December 1, \textbf{Rosa Parks}, a seamstress and secretary for the NAACP,
doesn't listen to James Edwards's instructions to leave her seat on the bus and gets arrested.
This triggers the \textbf{Montgomery Bus Boycott}.
Leading this boycott is a Baptist minister named \textbf{Martin Luther King, Jr.}
He is elected president of the Montgomery Improvement Association, and people start to listen to him.

In September 1957, Orval Faubus, governor of Arkansas,
decides to block integration of Little Rock Central High School.
Within two weeks, President Eisenhower federalizes the National Guard and orders the Army into Little Rock.
The \textbf{Little Rock Nine} get a military escort to school.
By the end of the month, Eisenhower signs the \textbf{Civil Rights Act of 1957}.
This happens in spite of segregationist \textbf{Strom Thurmond},
who tried to stop it by giving the longest one-person filibuster of all time, lasting over 24 hours.

Meanwhile, King founds the \textbf{South Christian Leadership Conference} (SCLC), and is its first president.
The SCLC leads the Albany Movement in Georgia against segregation in 1962.
In 1963, he organizes nonviolent protests in Birmingham, garnering national attention.
Alabama is governed by \textbf{George Wallace},
who called for ``segregation now, segregation tomorrow, segregation forever'' when he was inaugurated.
The protests result in a massive police response they land King in jail,
where he writes his ``Letter from a Birmingham Jail'':
\begin{quote}
  Moreover, I am cognizant of the interrelatedness of all communities and states.
  I cannot sit idly by in Atlanta and not be concerned about what happens in Birmingham.
  \textbf{Injustice anywhere is a threat to justice everywhere.}
  We are caught in an inescapable network of mutuality, tied in a single garment of destiny.
  Whatever affects one directly, affects all indirectly.
\end{quote}

In August 1963, King organizes the \textbf{March on Washington} for Jobs and Freedom,
with logistical help from Bayard Rustin.
Speakers at the huge event include Walter Reuther, Josephine Baker,
and John Lewis, who criticizes Kennedy in his speech, causing some controversy.
Of course, the most famous speaker is King himself, who gives his ``Normalcy, Never Again'' speech,
saying that he had come to ``cash a check'', and refused to believe the bank was bankrupt.
The speech quickly becomes known as his \textbf{``I Have a Dream''} speech.

In 1968, King travels to Memphis, Tennessee, to help striking African American sanitation workers,
who have walked out in protest of Mayor Henry Loeb.
On April 3, he delivers his \textbf{``I've Been to the Mountaintop''} speech:
\begin{quote}
  Well, I don't know what will happen now.
  We've got some difficult days ahead.
  But it doesn't matter with me now.
  Because I've been to the mountaintop.
  And I don't mind.
  Like anybody, I would like to live a long life.
  Longevity has its place.
  But I'm not concerned about that now.
  I just want to do God's will.
  And He's allowed me to go up to the mountain.
  And I've looked over.
  And I've seen the promised land.
  I may not get there with you.
  But I want you to know tonight, that we, as a people, will get to the promised land!
  And so I'm happy, tonight.
  I'm not worried about anything.
  I'm not fearing any man.
  My eyes have seen the glory of the coming of the Lord!
\end{quote}
While staying at the Lorraine Motel in Memphis,
King walks out onto his balcony at 6:01 PM on April 4.
He is struck by a bullet fired into his jaw.
The FBI investigation finds the fingerprints of \textbf{James Earl Ray} at the origin of the gunfire.
Two months later, Ray is captured at Heathrow Airport, and he confesses to having killed King.

\subsection*{On the Brink}
% JFK
% Assassination
% Jacqueline

\subsection*{Lyndon B.\ Johnson}

\section{Nixon and Carter: The 1970s}

\subsection*{Rise and Fall of Richard M.\ Nixon}

\subsection*{Carter and Foreign Policy}

\section{Republicans in Power}

\subsection*{Ronald Reagan}

\subsection*{Gulf Wars}
