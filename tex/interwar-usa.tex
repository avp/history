\chapter{Boom and Bust}

\epigraph{%
  The worst is over without a doubt.
}{Secretary of Labor James Davis \\ June 29, 1930}

\section{Roaring Twenties}

\subsection*{1920's Republicans}

\subsubsection*{Warren G.\ Harding}

In 1920, \textbf{Warren G. Harding} was elected to the Presidency,
promising a ``return to normalcy'' following the First World War.
Harding had defeated Leonard Wood to get the Republican nomination,
which he'd won in a ``smoke-filled room'' thanks to his campaign manager Harry Daugherty;
he proceeded to use a front-porch campaign to win the general election.
Notably, Harding was the first Senator to be elected to president.
His Vice President was Calvin Coolidge, and his closest advisors were called the ``Ohio Gang''.
While Harding was president, 10,000 West Virginia mine workers went on strike,
and they fought with the Logan Defenders in the \textbf{Battle of Blair Mountain}.

Harding established the General Accounting Office,
and he appointed William Howard Taft and Edward Sanford to the Supreme Court.
He also signed the Budget and Accounting Act and Sheppard-Towney Act.

Harding's administration was also part of the \textbf{Teapot Dome} scandal.
The scandal centered around Harding's Interior Secretary \textbf{Albert Fall}.
Fall was favored by companies such as Pan American Petroleum to give rights to oil fields such as Elk Hills.
Robert La Follette ordered investigation of the scandal,
and Thomas Walsh headed the investigation, asking for the resignation of Secretary of the Navy Edwin Denby.
The Supreme Court heard the case McGrain v. Daugherty,
and Edward Doheny was one of the others implicated in the investigation.

\subsubsection*{Calvin Coolidge}

When Harding died in office, \textbf{Calvin Coolidge} became president.
While governor of Massachusetts, Coolidge had put down a police trike in Boston, calling on the militia.
He was called ``Silent Cal'', known for his laconic wit.
Coolidge signed the Indian Citizenship Act into law,
and his Secretary of State Frank Kellogg signed the \textbf{Kellogg-Briand Pact},
prohibiting the use of war to resolve disputes (it didn't work very well).
Coolidge was reelected to office over Robert La Follette and John W. Davis, and his running mate was Charles Dawes.
He appointed J. Edgar Hoover head of the FBI\@.
He then declined renomination, saying ``I do not choose to run''.

\subsection*{Prohibition}

In the early 20th century, a good number of people were following the \textbf{temperance movement}.
People like Carrie Nation had been opposing alcohol use for some time, enforcing beliefs with a hatchet.
The Women's Christian Temperance Union (WCTU) was founded by Frances Willard to promote prohibition of alcohol.
The Anti-Saloon League also supported banning alcohol.

In 1920, the ``Noble Experiment'' of Prohibition was put into effect by the \textbf{18th Amendment}.
The \textbf{Volstead Act} enforced Prohibition, setting the law that actually banned alcohol itself.
Prohibition would remain in effect until 1933 until it was repealed by the \textbf{21st Amendment}.

\subsection*{Gangs and Gangsters}

The start of Prohibition led to the rise of all sorts of unsavory people.
Chief among these was \textbf{Al Capone}, a gangster on the south side of Chicago.
At one point, Capone had been slashed in the face while working for Frankie Yale in the Harvard Inn,
leading to his nickname: ``Scarface''.
When William Emmett Dever forced Capone out of the city, Capone made a deal with mayor Big Bill Thompson,
and got himself a new headquarters at the Lexington Hotel.

At the \textbf{St.\ Valentine's Day Massacre} in 1929,
Capone's agents attacked North Side Gang members who worked for Bugs Moran.
The victims had been lured to the SMC Cartage Co.\ building for a fake shipment of whiskey.
Six of Moran's men were killed, but Moran lived.
After the massacre, there was an attack on the Barker compound.

Capone was targeted by the Department of Justice's Prohibition Bureau.
The bureau was led by \textbf{Eliot Ness}, who formed the \textbf{Untouchables} in Chicago to go after gangsters.
Capone was eventually sent to prison for tax evasion,
based on precedent set in United States v. Sullivan.

\section{Great Depression}

\subsection*{Hoover's Crash}

\textbf{Herbert Hoover} was the last president of the roaring twenties.
He'd made his fortune directing Australia's Consolidated Zinc Inc.
Hoover won the election of 1928 by defeating Al Smith,
promising ``a chicken in every pot and a car in every garage''.
Hoover signed the Norris-LaGuardia act,
ending ``yellow-dog'' labor contracts that prevented employees from unionizing.

On October 29, 1929, \textbf{Black Tuesday}, the New York Stock Exchange crashed, starting the Great Depression.
Hoover's worst mistake was signing the \textbf{Smoot-Hawley Tariff} in 1930,
claiming that ``Prosperity is just around the corner''.
He also created the Reconstruction Finance Corporation to try to stabilize the economy.

The \textbf{Bonus Army} was a group of World War~I veterans, led by Walter Waters,
that traveled to Washington to force payment of their insurance certificates.
Others that supported the army included Evelyn McLean and General Pelham Glassford.
They camped out in Anacostia Flats, in a collection of tents some described as an ``immense hobo jungle''.
Hoover sent Douglas MacArthur to break up the army.
MacArthur, of course, did this using artillery and tear gas, but Hoover never reprimanded him for it.

Of course, Hoover's failure to deal with the situation correctly led to people not to like him very much.
Small shantytowns known as ``Hoovervilles'' sprung up all over the place.

\subsection*{A New Deal}

The election of 1932 wasn't a great year for Herbert Hoover.
He lost all but six states to the Democratic candidate, \textbf{Franklin Delano Roosevelt}.
FDR's running mate was John Nance Garner.

Shortly after the election, Giuseppe Zangara shot at Roosevelt while he was meeting with Chicago mayor Anton Cermak.
Zangara missed because he was standing on a wobbly chair, killing Cermak but leaving Roosevelt unscathed.

Roosevelt proposed a \textbf{New Deal} to help get the country through the Depression.
An important piece of New Deal legislation was the \textbf{National Industrial Recovery Act} (NIRA),
which would set up the PWA and the NRA (see below).
Schechter v. US ruled the NIRA unconstitutional.
The New Deal was composed of many different projects, most of which have a 3 letter acronym associated with them.

\begin{itemize}
  \item
    \textbf{NRA}:
    The NRA was symbolized by a Blue Eagle with the words ``We Do Our Part''.
    It allowed a ``blanket code'' and was originally directed by General Hugh S. Johnson.
    The NRA also regulated business practices and set up ``codes of fair competition''.
    Clarence Darrow and the FTC would investigate the NRA for monopolistic practices.

  \item
    \textbf{WPA}:
    The Works Progress Administration, or Works Project Administration,
    was a federal agency that employed people to carry out public projects.
    It was run by Harry Hopkins, and employed over 8.5 million people.
    It was preceded by the FERA\@.

  \item
    \textbf{CCC}:
    The Civilian Conservation Corps was a program that employed and trained younger men.
    The CCC planted millions of trees and did outdoor work projects.

  \item
    \textbf{TVA}:
    The Tennessee Valley Authority was a government entity that provided power to the Tennessee Valley.
    The first chairman of the TVA was \textbf{David Lilienthal},
    who also served on the Atomic Energy Commission.
    It used the Wilson Dam and the Tellico Dam, and was ruled to be legal in Ashwander v. TVA\@.
    Reagan would eventually criticize the TVA in ``A Time for Choosing'',
    and Barry Goldwater said that he would sell it ``for a dollar''.
\end{itemize}

In an attempt to improve the chances of New Deal legislation winning in the Supreme Court,
FDR tried \textbf{court packing}.
He wanted to add extra justices and fundamentally change the makeup of the court.
The court packing plan also called for a ten year term limit and mandatory retirement of justices.
The conservative justices were called the ``Four Horsemen'',
and the liberal justices were the ``Three Musketeers''.
On ``White Monday'', Owen Roberts changed his mind on a Washington state minimum wage case,
tipping the court to a 5--4 ruling.
Now that Roberts was considered to be on the Democratic side,
the change of opinion is called ``the switch in time that saved nine''.

Another piece of important New Deal era legislation is the \textbf{National Labor Relations Act}, or Wagner Act.
It set up the National Labor Relations Board and was quite pro-union.
However, the act eventually was gutted by the later Taft-Hartley Act (more on that later).

