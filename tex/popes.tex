\chapter{Popes}

\newcommand{\papacy}[2]{\noindent\textit{Pontificate: #1--#2}}

The Catholic Church crops up fairly often in our survey of history.
Almost every time they've been mentioned, we've been acquainted with new popes,
but we haven't really taken the time to describe the exploits of all of them,
like we have with U.S.\ presidents or monarchs of the British isles,
because a lot of them simply weren't that notable.
Here, we go over the ones that you should care about,
as well as some that you probably shouldn't, but are interesting regardless.

\subsection*{Saint Peter}
\papacy{c. 30}{64}

The first pope was likely born in AD 1.
An Apostle of Jesus, the Bible claims he was given the keys to the Kingdom of Heaven by Christ,
and was crucified upside-down.
The Vatican's \textit{Basilica di San Pietro} is of course named after him.

\subsection*{Gregory I}
\papacy{590}{604}

Saint Gregory the Great ordered Augustine of Canterbury to undertake the \textbf{Gregorian Mission},
the first large-scale mission out of Rome,
to convert the Anglo-Saxon pagans to Christianity.
By the time the final missionary died in 635, much of southern Britain had been converted,
and Augustine was made the first Archbishop of Canterbury.
He notably had to deal with the Italian incursions of the Lombards,
but despite his ability to secure a short truce
(due to a massive payment to Lombard King Agilulf as he marched on Rome)
Gregory wasn't able to create a lasting peace with them.

Gregory is considered the founder of the medieval notion of the papacy,
largely due to his correspondence and other notable writings.
His \textit{Commentary on Job}, known as the \textit{Magna Moralia},
is perhaps his most famous work, taking up 6 volumes of discussion of morality.

\subsection*{Gregory VII}
\papacy{1073}{1085}

Born Hildebrand of Sovana, Gregory was elected by acclamation.
The cry of ``Let Hildebrand be pope!'' went up at the funeral of his predecessor, Pope Alexander~II\@.
He was elected legally later the same day.

Gregory is probably best known for launching the Investiture Controversy,
which you can read more about in the subsection* on Henry~IV and the early Holy Roman Empire.
The Controversy wouldn't be resolved until the Concordat of Worms by Callixtus~II\@.

Gregory was well known for his largely successful attempts to reform the church.
He was the first in centuries to actually enforce internal simony rules,
as well as the first to vigorously push celibacy policy on the clergy.

\subsection*{Urban II}
\papacy{1088}{1099}

The Frenchman Odo of Ch\^atillon went on to become one of the most famous popes of all time.
He originally studied under Bruno of Cologne (founder of the Carthusian Order of monks),
and Gregory~VII named him Bishop of Ostia in 1080;
Odo was one of the most prominent supporters of Gregory's extensive reforms.

At the start of his papacy in 1088,
Urban was kept away from the Vatican because of the continuing Investiture Controversy.
Henry~IV had taken Rome and installed the antipope Clement~III in 1084.

Urban continued his predecessor's reforms while touring Europe.
He notably supported Conrad~II of Italy (then Prince Conrad)
in his rebellion against his father Henry~IV\@;
in particular, he facilitated Conrad's marriage to Maximilia of Sicily,
whose father Count Roger financed Conrad's military ambitions.
Urban also notably supported \textbf{Anselm}, Archbishop of Canterbury,
when he fled William~II and England.
He excommunicated Philip of France, due to the latter's bigamy in marrying Bertrade de Montfort
(eventually, a public penance resulted in Bertrade's children succeeding to the French throne).

Of course, Urban is best known for his involvement in the First Crusade.
Having been asked for help by Alexios~I Komnenos at the Council of Piacenza,
Urban called for the First Crusade at the Council of Clermont
(more on this in the section on the Crusades).

\subsection*{Alexander VI}
\papacy{1492}{1503}

The Spaniard.
Born Rodrigo Llan\c{c}ol,
he took the name \textbf{Rodrigo Borgia} on the ascension of his uncle Callixtus~III to the papacy.
He took full advantage of the nepotism characteristic of the age,
and was appointed Cardinal-Bishop of Albano by the time he was 40.
Rodrigo was elected to the papacy in 1492 following the death of Innocent~VIII\@.
Giovanni Lorenzo de Medici (the future Pope Leo~X) reportedly was made uneasy by this, warning:

\begin{quote}
  Now we are in the power of a wolf, the most rapacious perhaps that this world has ever seen.
  And if we do not flee, he will inevitably devour us all.
\end{quote}

Alexander unabashedly used his newfound position of power to advance the interests of his (illegitimate) children;
in particular his son Cesare (an ambitious and violent man, recall Machiavelli's \textit{The Prince})
and his daughter Lucrezia (who nobody has ever proven poisoned many people).
Also never proven were allegations of incest between Lucrezia and Cesare.

In the process of acquiring wealth for his progeny,
Alexander started numerous wars with France and Spain for land in Northern Italy.
When this didn't work out so well for him and Charles~VIII of France came knocking on the gates of Rome,
he formed the Holy League of 1495 to push him back out of the peninsula.

During the Jubilee in 1500,
Alexander created the tradition of opening a holy door on Christmas Eve and closing it the following day.
He set up opening/closing these doors in the four major basilicas and formalized the process.

Alexander was such a terrible pope that the priests of St.\ Peter's simply refused to bury him on his death.
In addition, he did establish the despicable behavior which would become a standard of Renaissance papacy.

\subsection*{Julius II}
\papacy{1503}{1513}

Giuliano della Rovere, one of Alexander~VI's greatest rivals,
spent the entirety of Alexander's pontificate in exile after losing the election in 1492 to him.
He'd been promoted to cardinal initially when his uncle Pope Sixtus~IV had begun his papacy.

Following Alexander's death, Della Rovere supported the election of Pius~III, who died within a month.
Through some impressive diplomacy,
Della Rovere then managed to secure the support of Cesare Borgia for the ensuing election
and was elected almost unanimously and took the name of Pope Julius~II\@.

Julius proceeded to systematically dismantle everything the Borgias had built while Alexander was pope.
He formed the \textbf{League of Cambrai}
with Louis~XII of France, Maximilan~I Holy Roman Emperor, and Ferdinand~II of Aragon,
as a means to defeat Venice.
Following the conflict, Julius placed France under interdict.
As the resultant \textbf{Italian Wars} drew on, Venice and France switched places,
and Julius then entered into the new Holy League of 1511 to deal with France.

Julius amended the Treaty of Tordesillas to accommodate Portuguese claims on Brazil,
and he allowed Henry~VIII of England to marry Catherine of Aragon (which of course we know didn't last).

While in office, Julius had the Sistine Chapel torn down and rebuilt,
and he commissioned Michelangelo to paint its ceiling.
Upon his death, Michelangelo created the Tomb of Julius in San Pietro in Vincoli,
but since it wasn't ready soon enough, Julius is actually buried in Rome instead.

Julius's numerous military accomplishments have led to his being called ``The Warrior Pope'',
and while they were indeed impressive, he is often considered a failure as a religious leader,
leaving a situation in which the Protestant Reformation was able to grow.

\subsection*{Leo X}
\papacy{1513}{1521}

Giovanni di Lorenzo de Medici was made a cardinal at only 13 by Innocent~VIII,
whom G.J.\ Meyer refers to as ``ludicrously misnamed'' due to his extreme penchant for nepotism.

Upon his election to the papacy, Leo decided to enact some reform.
He improved the quality of the college of cardinals,
and he closed the Fifth Lateran Council (started by Julius);
however, he failed to implement the reforms prescribed by the council.

While Leo wasn't a despicable human being like most of his predecessors,
his pursuits were slightly eccentric or perhaps buffoon-like.
After Manuel~I gifted Leo a white elephant named Hanno,
Leo used him to embarrass an abbot named Giacomo Baraballo by making him ride Hanno until thrown off.

Of course, much of what Leo did was overshadowed by Martin Luther's actions
in response to Leo's issuance of indulgences.
The Reformation caused Leo to issue the bulls \textit{Exsurge Domine} and \textit{Decet Romanum Pontificem}.

He died of illness in 1521.

\subsection*{Clement VII}
\papacy{1523}{1534}

After Leo came Adrian~VI, the only Dutchman to be Pop.e
He also failed to reform the Renaissance papacy much,
and was succeeded by Guilio de' Medici,
the nephew of Lorenzo the Magnificent.

Upon taking the papacy, Guilio named himself Clement~VII,
and he sent Archbishop Nikolaus von Schonberg of Capua to end the Italian Wars, but that failed.
He allied himself with the Italian princes and Francis~I of France,
but when Francis lost badly at Pavia, Clement had to let go of ties.
A few years after Pavia, Clement resumed the alliance with France
by forming the \textbf{League of Cognac} with France, Venice, and Milan.

In 1527, Charles~III, Duke of Bourbon, marched on Rome and besieged the city.
Shortly thereafter, he died while trying to climb a ladder,
and his troops quickly began to sack Rome.
These events ended the grand Renaissance in Rome.
Terrified, Clement took refuge in the Castel Sant'Angelo,
where he was captured and imprisoned for a few months;
during this time, Niccolo Capponi was elected as Gonfaloniere in Florence.
Eventually, he was let free, and he spent some time in exile before returning in late 1528.

Clement is known now for opposing the annulment of Henry~VIII of England's marriage to Catherine of Aragon.
His excommunication of Henry led to the eventual 1534 Act of Supremacy that created the Church of England,
and began the English Reformation.
Read more about that in the section on Henry~VIII.

\subsection*{Paul III}
\papacy{1534}{1549}

Following the sack of Rome, Clement's Catholic Church was in turmoil.
Paul~III was made pope into this era in the wake of the Protestant Reformation.
Born Alessandro Farnese in the Papal States,
Paul was the first of the Renaissance popes to actively take action to improve the Catholic Church,
in response to the Reformation.

In an attempt to fix the problems that Martin Luther and Charles~V, Holy Roman Emperor,
had with the Catholic Church, Paul started taxing his own subjects more,
relieved certain important nobles from positions of power, and caused strife in his own domain.
Cities like \textbf{Perugia} attempted to renounce the pope, but were forcibly suppressed by Paul's son Pier Luigi.

Paul also notably recognized multiple religious societies and orders,
including the Jesuits, the Barnabites, et al.

\subsection*{Leo XIII}
\papacy{1878}{1903}

Born Vincenzo Gioacchino Raffaele Luigi Pecci,
had been Camerlengo to Pius~IX, whose papacy was one of only two which lasted longer than Leo's own.
Following Pius's death, Pecci was elected pope and chose the name Leo~XIII.

Known for his intellectualism, he became an advocate for social welfare,
writing the encyclical \textit{Rerum novarum},
which explained that workers needed a safe workplace, fair wages, and the right to unionize.

He issued many other encyclicals, and became known as the ``Rosary Pope''.
Leo also established Mary as the Mediatrix (the person through whom Christ bestows graces).

\subsection*{Pius XII}
\papacy{1939}{1958}

The successor of Pius~XI,
Pius~XII was born Eugenio Maria Giuseppe Giovanni Pacelli in Rome.
Prior to his appointment, he was papal nuncio and Cardinal Secretary of State,
during which time he secured treaties with Latin America
and signed the \textit{Reichskonkordat} with Hitler's Germany
(the treaty kept the Church in Germany but forced bishops to swear loyalty to the Reich).

Pius was made pope just months before the outbreak of World War~II\@.
He denounced the Nazis and tried to keep the Catholic Church in Germany,
denouncing totalitarianism.
Pius also defined the Assumption of Mary in his \textit{Munificentissimus Deus},
during which he invoked papal infallibility.

He was succeeded by John~XXIII\@.
