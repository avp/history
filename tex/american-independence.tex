\chapter{American Independence}

\section{18th Century Colonies}

We start in the years following the French and Indian War in the colonies of America.
Various people would begin to dislike British rule in the area.
In 1763, the \textbf{Proclamation of 1763} was issued, forbidding settlement west of the Appalachians.

In the same year, \textbf{Pontiac's Rebellion} broke out against British policies in the Great Lakes area.
Pontiac, an Ottawa chieftain, started the war near Fort Detroit.
An attack on the weapons caches was stopped because the mistress of Henry Gladwin, the commander of the fort,
warned the defenders of the fort of the attack.
During the war, Jeffrey Amherst tried to win by infesting blankets with smallpox.
A massacre occurred at Parents' Creek, after which the river was called Bloody Run.
Other battles included the Devil's Hole Massacre and the Battle of Bloody Run.
The Paxton Boys, a vigilante group, went on a rampage in western Pennsylvania,
resulting in the governor John Penn ordering their arrest.
The rebellion ended in a stalemate, and it was stopped partially by the Fort Niagara treaty.

\subsection*{Taxing the Colonies}

In 1764, Parliament passed the Currency Act, which stopped use of some paper money,
and they also passed the \textbf{Sugar Act} which placed duties on multiple goods.
The prime minister at this time was \textbf{George Grenville}, who wanted to impose more taxes,
but he waited for a year before doing so.

When the colonists couldn't put up any more money, the \textbf{Stamp Act} (1765) was passed,
which was the first direct tax only on the colonies, and it taxed paper products.
Also in 1765, the \textbf{Sons of Liberty} were formed.
They tried to show that the British tax laws were pointless,
and in Boston, they looted the home of Thomas Hutchinson, chief justice.
The Stamp Act Congress was convened at Federal Hall in New York City,
and a Declaration of Rights and Grievances was made.
The Congress was convened by John Otis, and participants included John Dickenson and Caesar Rodney.

In 1767, the \textbf{Townshend Acts} (named after the Chancellor of the Exchequer)
were imposed on paper, glass, tea, etc.
In 1768, colonists started putting together resistance ---
riots broke out when \textit{Liberty}, John Hancock's sloop, was seized.

On March 5, 1770, a mob gathered around a bunch of British soldiers, throwing snowballs,
supposedly because of some accusations made by a wigmaker's apprentice.
The soldiers, under Thomas Preston, started firing into the crowd and hit 11 people ---
five people died in total in the \textbf{Boston Massacre} on King Street,
which ended when governor \textbf{Thomas Hutchinson} cleared the crowd.
Among the people who died were the African American merchant Crispus Attucks,
as well as Samuel Gray, James Caldwell, Samuel Maverick, and Patrick Carr.
\textbf{Paul Revere}, a local silversmith, made and circulated Henry Pelham's print of the massacre.
John Adams defended the soldiers involved against the prosecution of Samuel Quincy and Robert Treat Paine,
and they were acquitted.

In 1770, the new Prime Minister was \textbf{Lord North},
and Parliament withdrew a bunch of taxes except for tea.
In the \textit{Gaspee} Affair, some patriots under John Brown burned a warship.
The ``Committees of Correspondence'' were a setup of communications between the colonies,
and they allowed for a rebel government.

Then, Parliament passed the \textbf{Tea Act}, to help the East India Company sell Dutch tea.
A meeting in Boston decided that the tea wouldn't be allowed to land,
in opposition to Hutchinson's stance.
Francis Roth tried to negotiate with Hutchinson, but Hutchinson wouldn't give in.
So, the \textbf{Boston Tea Party} took place.
Patriots met at Faneuil Hall and the Old South Meeting Hall and dressed like Mohawks,
and on a signal from their leader \textbf{Sam Adams}, they went out to Boston Harbor.
They then went on ships in the harbor and dumped tea off into the water.
Similar events took place in Edenton, North Carolina (organized completely by women),
and in Annapolis, Maryland (organized by Thomas Charles Williams).

In response to the Boston Tea Party, Parliament passed the \textbf{Intolerable Acts} (Coercive Acts).
These consisted of:
the Massachusetts Government Act,
the Administration of Justice Act,
the Boston Port Act,
and the Quartering Act.
So, patriots issued the Suffolk Resolves, which formed a new ``Provincial Congress''.
In 1774, the \textbf{First Continental Congress} convened.
During the Congress, it was decided that Americans would obey Parliament,
but continue to resist taxes and boycott goods.

\section{Revolution}

\subsection*{Opening Battles}

In February 1775, Massachusetts was declared to be in a state of rebellion.
This resulted in the \textbf{Battles of Lexington and Concord},
which started with the ``Shot heard `round the world''
(not to be confused with Bobby Thomson's 1951 home run).

The \textbf{Battle of Bunker Hill} near Boston followed a siege on the city,
after Thomas Gage had tried to take Dorchester Heights.
The Americans were led by \textbf{Israel Putnam}, who told the soldiers to
not fire until they could see the whites of the British soldiers' eyes.
The American doctor Joseph Warren died at the battle,
William Prescott led the expedition to occupy Breed's Hill,
and the British admiral Samuel Graves provided artillery support from the HMS Somerset.
Other hills included Moulton and Copp Hills, which were assaulted by Robert Pigot.
The British, under William Howe, won a Pyrrhic victory at Bunker Hill.

After Bunker Hill, the colonists issued the \textbf{Olive Branch Petition} to try and make peace with George.
The king rejected it after hearing about Bunker Hill,
and he issued the Proclamation for Suppressing Rebellion and Sedition.

\subsection*{Declaring Independence}

By June 1776, colonies were getting on board with the idea of independence,
after the Halifax Resolves allowed voting for independence.
The Committee of Five, consisting of
John Adams,
Roger Sherman,
\textbf{Benjamin Franklin},
Robert Livingston,
and \textbf{Thomas Jefferson},
drafted the \textbf{Declaration of Independence}.
On signing it on July 4, 1776,
John Hancock said that he wanted to make sure the king could read his huge signature.

The United States was a new nation, but they still had to defend their independence.
The British returned to America, and they swept down from Canada with the help of their German Hessian mercenaries,
and they took New York City after the Battle of Brooklyn.
At the Staten Island Peace Conference, John Adams and Benjamin Franklin met \textbf{William Howe},
commander of the British forces, but it didn't end very well.
The British also took New Jersey.

\subsection*{Turning Points}

Recall that \textbf{George Washington} was the commander of the Continental Army.
On December 26, 1776, the Hessian forces under Johann Rall were quartered in Trenton
(people say they were drunk because of Christmas, but that's probably not true).
Washington crossed the Delaware River and launched a surprise attack at the First \textbf{Battle of Trenton}.
In order to help the attack,
the American spy John Honeyman had fed the Hessians bad intelligence after an interrogation.
The local militia, led by John Cadwalader, launched an attack on Bordentown to try to stop enemy supplies,
but weather conditions didn't really let them pull that off very well.
John Sullivan led his column around,
and stopped the Hessians from escaping down an abandoned road near Assunpink Creek.
Washington's forces were victorious at Trenton,
and soon afterwards, at the Battle of Princeton, the Americans defeated \textbf{Charles Cornwallis},
and took back the rest of New Jersey.
These events really helped boost American morale.

In 1777, the British sent an invasion under \textbf{John Burgoyne} to move down from Canada.
They met the Americans, led by Horatio Gates, at the \textbf{Battle of Saratoga},
which followed the British capture of Fort Ticonderoga.
Near the start of the battle, Daniel Morgan's sharpshooters killed many British officers.
Soldiers under \textbf{Benedict Arnold} charged the Hessians,
and Arnold was wounded in the leg, but he was a hero of the battle.
Notable engagement locations in the battle included Freeman's Farm and Bemis Heights.

The American victory at Saratoga signaled a turning point,
because France (and Spain) decided they wanted to help the Americans out,
having seen them decisively defeat a British force.
The British were stuck fighting a global war they never really wanted,
against powerful countries, without any real allies of their own.

\subsection*{Conclusion}

In 1781, the British army, now led by Cornwallis, marched on Yorktown, Virginia.
They hoped to rendezvous with the British fleet on the coast,
but that didn't end up happening.
A fleet from Saint Domingue, led against the British by the Comte de Grasse, blocked escape for the
British fleet led by Thomas Graves at the \textbf{Battle of the Chesapeake}.
The joint command of Comte de Rochambeau and George Washington defeated Cornwallis
on land, during the \textbf{Battle of Yorktown}.
During the battle, Robert Abercrombie tried to disable enemy cannons.
After the battle, the British band played ``The World Turned Upside Down''.
This was Cornwallis's final surrender to Washington.

In 1783, the \textbf{Treaty of Paris} was signed, ending the Revolution.
It was signed by John Jay, Ben Franklin, and John Adams.
One of the negotiators, Henry Laurens, had just been released from the Tower of London.
Article 2 used the Mitchell Map, which wasn't very accurate.
Article 6 required forts on the Great Lakes to be vacated.
It set the borders of the USA as the Mississippi River and Florida.

\section{A New Nation}

The new USA was under the rule of the \textbf{Articles of Confederation}.
They established a ``league of friendship'' when the colonies declared independence,
but they didn't afford a lot of power to a federal government.
The government passed the Land Ordinance and Northwest Ordinance,
and approved American expansion into Canada.

In 1786, \textbf{Shays' Rebellion} broke out in Massachusetts,
led by Daniel Shays and other farmers who had helped win the Revolutionary War.
The farmers were mad about confiscation of property that resulted from the fact that they were in debt.
It had to be suppressed by Governor James Bowdoin, who asked William Shephard to lead the military.
Shephard and Benjamin Lincoln led forces that held the Springfield Armory and killed some rebels.
Other rebels include Job Shattuck, Luke Day, John Bly, Charles Rose, and Moses Sash.
At one point, rebels stormed the Northampton Courthouse.
John Hancock became governor of Massachusetts following the rebellion,
which showed that the Articles of Confederation weren't good enough.
The Annapolis Convention was called to try and make them better,
which called for a better constitutional convention.

So, the \textbf{Constitutional Convention} was called in Philadelphia in 1787,
by Federalists who wanted a stronger national governmental presence.
It was there that the new Constitution, drafted largely by \textbf{James Madison},
was submitted for ratification.
