\chapter{American Independence}

\epigraph{%
  Once vigorous measures appear to be the only means left
  of bringing the Americans to a due submission to the mother country,
  the colonies will submit.
}{King George~III}

\section{Colonies in the 18th Century}

\subsection*{Conflicts}

We start in the years following the French and Indian War in the colonies of America.
Various people came to dislike British rule in the area.
In 1763, the \textbf{Proclamation of 1763} was issued, forbidding settlement west of the Appalachians.

In that same year,
\textbf{Pontiac's Rebellion} broke out against British policies in the Great Lakes area.
Pontiac, an Ottawa chieftain, started the war near Fort Detroit.
An attack on the weapons caches was stopped because the mistress of Henry Gladwin, the commander of the fort,
warned the defenders of the fort of the attack.
Jeffrey Amherst tried to win the war by infesting blankets with smallpox.
A massacre occurred at Parents' Creek, after which the river was called Bloody Run.
Other battles included the Devil's Hole Massacre.
A vigilante group called the Paxton Boys went on a rampage in western Pennsylvania,
resulting in governor John Penn ordering their arrest.
The rebellion ended in a stalemate and was partially concluded by the Fort Niagara treaty.

\subsection*{Taxing the Colonies}

In 1764, Parliament passed the Currency Act, which stopped use of some paper money.
They also passed the \textbf{Sugar Act}, placing duties on multiple goods.
The prime minister at the time was \textbf{George Grenville}, who wanted to levy even more taxes,
but he waited for a year before doing so.

In 1765, the \textbf{Sons of Liberty} were formed.
They tried to show that the British tax laws were pointless,
and in Boston, they looted the home of Thomas Hutchinson, chief justice.
The Stamp Act Congress was convened at Federal Hall in New York City,
and a Declaration of Rights and Grievances was made.
The Congress was convened by John Otis, and participants included John Dickenson and Caesar Rodney.

When the colonists couldn't put up any more money,
Parliament passed the \textbf{Stamp Act} (1765),
the first direct tax only on the colonies, which taxed paper products.
The ``Committees of Correspondence'', a setup of communications between the colonies, were created.
As a result of the Stamp Act, Ebenezer MacIntosh led an assault on Thomas Hutchinson's house.

In 1767, the \textbf{Townshend Acts} (named after the current Chancellor of the Exchequer)
were imposed on paper, glass, tea, etc.
Within a year, colonists started putting together resistance ---
riots broke out when \textit{Liberty}, John Hancock's sloop, was seized.

\subsection*{Massacre and Party}

On March 5, 1770, a mob gathered around a bunch of British soldiers, throwing snowballs,
supposedly because of some accusations made by a wigmaker's apprentice.
The soldiers, under Thomas Preston, started firing into the crowd and hit 11 people ---
five people died in total in the \textbf{Boston Massacre} on King Street,
which ended when governor \textbf{Thomas Hutchinson} cleared the crowd.
Among the people who died were the African American merchant \textbf{Crispus Attucks},
as well as Samuel Gray, James Caldwell, Samuel Maverick, and Patrick Carr.
\textbf{Paul Revere}, a local silversmith you may have heard of,
made and circulated Henry Pelham's print of the massacre.
John Adams defended the soldiers involved against the prosecution of Samuel Quincy and Robert Treat Paine,
resulting in their acquittal.
The press compared the event to a similar one at St.\ George's Field a few years before.

In 1770, the new Prime Minister was the Tory \textbf{Lord North},
and Parliament withdrew a bunch of taxes except for tea.
In the \textit{Gaspee} Affair, some patriots under John Brown burned a warship.

Parliament then passed the \textbf{Tea Act} to help the East India Company sell Dutch tea.
A meeting in Boston decided that the tea shouldn't be allowed to land,
a position in stark opposition to Hutchinson's stance.
Francis Roth tried to negotiate with Hutchinson, but Hutchinson wouldn't give in.
So, the Patriots organized the \textbf{Boston Tea Party}.
Participants met at Faneuil Hall and the Old South Meeting Hall, dressed like Mohawks,
and on a signal from their leader \textbf{Samuel Adams}, they went out to Boston Harbor.
They then went on ships in the harbor and dumped tea off into the water.
Similar events took place in Edenton, North Carolina (organized completely by women),
and in Annapolis, Maryland (organized by Thomas Charles Williams).

In response to the Boston Tea Party, Parliament passed the \textbf{Intolerable Acts} (Coercive Acts).
These consisted of:
the Massachusetts Government Act,
the Administration of Justice Act,
the Boston Port Act,
and the Quartering Act.
The patriots issued the Suffolk Resolves, forming a new ``Provincial Congress''.
In 1774, the \textbf{First Continental Congress} convened.
During the Congress,
it was decided that Americans would obey Parliament but continue to resist taxes and boycott goods.

\section{Revolution}

\subsection*{Opening Battles}

In February 1775, Massachusetts was declared to be in a state of rebellion.
Thus began \textbf{Battles of Lexington and Concord},
which started with the ``Shot heard `round the world''
(not to be confused with Bobby Thomson's 1951 home run).
Revere, Prescott, and Dawes went around warning the Americans that the British were on their way.

The \textbf{Battle of Bunker Hill} near Boston followed a siege on the city
after Thomas Gage tried to take Dorchester Heights.
The Americans were led by \textbf{Israel Putnam},
and William Prescott ordered soldiers to not fire until they could see the whites of the British soldiers' eyes.
The American doctor Joseph Warren died at the battle
and Prescott led the expedition to occupy Breed's Hill.
British admiral Samuel Graves provided artillery support from the HMS Somerset.
Other hills included Moulton and Copp Hills, which were assaulted by Robert Pigot.
The British under William Howe won a Pyrrhic victory at Bunker Hill
in what would be the bloodiest battle of the war.

After Bunker Hill, the colonists issued the \textbf{Olive Branch Petition} to try to make peace with George.
The document was requested by John Dickinson at the \textbf{Second Continental Congress}
and carried by Robert Penn.
George~III never actually saw the letter
and he issued the Proclamation for Suppressing Rebellion and Sedition.

\subsection*{Declaring Independence}

By June 1776, colonies were getting on board with the idea of independence
after the Halifax Resolves allowed voting for independence.
The Committee of Five, consisting of
John Adams,
\textbf{Benjamin Franklin},
Roger Sherman,
Robert Livingston,
and \textbf{Thomas Jefferson},
drafted the \textbf{Declaration of Independence}.
On signing it on July 4, 1776,
John Hancock said that he wanted to make sure the king could read his huge signature.

The United States was a new nation, but they still had to defend their independence.
The British returned to America,
sweeping down from Canada with the help of their German Hessian mercenaries,
and taking New York City after the Battle of Brooklyn.
At the Staten Island Peace Conference, John Adams and Benjamin Franklin met with Howe
but failed to come to an agreement.
The British also took New Jersey at some point.

\subsection*{Turning Points}

Recall that \textbf{George Washington} was the commander of the Continental Army.
On December 26, 1776, the Hessian forces under Johann Rall were quartered in Trenton
(people say they were drunk because of Christmas, but that might not be true at all).
Washington crossed the Delaware River and launched a surprise attack at the First \textbf{Battle of Trenton}.
In order to help the attack,
the American spy John Honeyman had fed the Hessians bad intelligence after an interrogation.
The local militia, led by John Cadwalader, launched an attack on Bordentown to try to stop enemy supplies,
but weather conditions didn't really let them pull that off very well.
John Sullivan led his column around
and stopped the Hessians from escaping down an abandoned road near Assunpink Creek.
Washington's forces were victorious at Trenton,
and soon afterwards,
the Americans defeated \textbf{Charles Cornwallis} at the Battle of Princeton to take back the rest of New Jersey.

In 1777, the British sent an invasion under \textbf{John Burgoyne} to move down from Canada.
Following the British capture of Fort Ticonderoga,
they met the Americans, led by Horatio Gates, at the \textbf{Battle of Saratoga}.
Near the start of the battle, Daniel Morgan's sharpshooters killed many British officers.
Soldiers under \textbf{Benedict Arnold} and Ebenezer Learned charged the Hessians.
Arnold was wounded in the leg and removed from combat, but was considered a hero of the battle.
Later, Simon Fraser was mortally wounded by a sniper.
Notable engagement locations in the battle included Freeman's Farm and Bemis Heights.

Later, Gates would be the leader that some Continental officers called the \textbf{Conway Cabal}
that tried to replace George Washington as command of the army.
The Cabal obviously failed, and Gates apologized for his role.

The American victory at Saratoga signaled a turning point,
because France (and Spain) decided they wanted to help the Americans out.
The British were stuck fighting a global war they never really wanted
against powerful countries without any real allies of their own.

\subsection*{Conclusion}

In 1781, the British army, now led by Cornwallis, marched on Yorktown, Virginia.
They hoped to rendezvous with the British fleet on the coast,
but that didn't end up happening.
A fleet from Saint Domingue, led against the British by the Comte de Grasse,
blocked escape for the British fleet led by Thomas Graves at the \textbf{Battle of the Chesapeake}.
Joint forces under the Comte de Rochambeau and Washington defeated Cornwallis on land
during the \textbf{Battle of Yorktown}.
In the midst the battle, Robert Abercrombie tried to disable enemy cannons.
The British plan to evacuate to Gloucester Point was stopped by a storm and they lost.
After the battle, the British band played ``The World Turned Upside Down''.
The battle signaled Cornwallis's final surrender to Washington,
and North proceeded to resign from his post as Prime Minister.

In 1783, the \textbf{Treaty of Paris} was signed, ending the Revolution.
Notable signers include John Jay, Ben Franklin, and John Adams.
One of the negotiators, Henry Laurens, had just been released from the Tower of London.
Article 2 used the Mitchell Map, which wasn't very accurate,
while Article 6 required forts on the Great Lakes to be vacated.
It set the borders of the USA as the Mississippi River and Florida.

\section{A New Nation}

The new United States were under the rule of the \textbf{Articles of Confederation}.
The Articles established a ``league of friendship'' when the colonies declared independence,
but they didn't give a lot of power to a federal government.
The government passed the Land Ordinance and Northwest Ordinance,
approving American expansion into Canada.

In 1786, \textbf{Shays' Rebellion} broke out in Massachusetts,
led by Daniel Shays and other farmers who had helped win the Revolutionary War.
Other rebels included Job Shattuck, Luke Day, John Bly, Charles Rose, and Moses Sash.
The farmers were mad about confiscation of property because they were in debt.
It had to be suppressed by Governor James Bowdoin, who asked William Shephard to lead the military.
Shephard and Benjamin Lincoln led forces that held the Springfield Armory and killed some rebels.
At one point, rebels stormed the Northampton Courthouse.
John Hancock became governor of Massachusetts following the rebellion,
which was effective in showing that the Articles of Confederation weren't good enough.
The Annapolis Convention was called to try and make them better,
and it called for a better constitutional convention.

The \textbf{Constitutional Convention} was called in Philadelphia in 1787
by Federalists who wanted a stronger national governmental presence.
It was there that the new Constitution, drafted largely by \textbf{James Madison},
was submitted for ratification.

In an attempt to ensure ratification,
John Jay, James Madison, and Alexander Hamilton wrote the \textbf{Federalist Papers}.
Signed ``Publius'', the papers said it'd be easier to prevent factions from being formed in a strong central union.
The 51st section of the papers claimed that ``if men were angels, no government would be necessary'',
and the paper opposed the Bill of Rights.
