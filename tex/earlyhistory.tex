\chapter{Dawn of Civilization}

\section{Mesopotamia}

First, it's worth noting that \textbf{Sumer} existed.
It had a city called Ur, and it also had a period of history called Uruk.

One of the first notable rulers of the beginning was \textbf{Sargon I of Akkad}.
Sargon conquered much of Sumer.
He fought a war against Ur-Zababa prior to becoming one of the greatest conquerors the world had ever seen.
His capital was never found.
Later, Ur became very important in the changes that the world would undergo.

\textbf{Hammurabi of Babylon} ruled around 1800 BC\@.
Notably, he made a code of law can be summarized with ``an eye for an eye'' --- it was quite harsh and painful.
The code was carved on a column in Susa,
and it discusses the consequences for slaves that disobey their masters,
and the epilogue of the code thanks the gods Zamana and Ishtar, important gods around the time.
Now, the code itself sits in the Louvre, having been discovered.

% TODO Nebuchadnezzar

% TODO Ashurbanipal

\section{Ancient Egypt}

In Egypt, pharaohs were building pyramids and other big things. Here are some notables:

\textbf{Djoser} was a king of the 3rd dynasty (Old Kingdom),
and he was buried in a notable namesake pyramid.
He worked with his vizier \textbf{Imhotep} (the guy from \textit{The Mummy})
to make it, and Imhotep was one of the most important people during that time.

\textbf{Akhenaten} (1353--1336 BC), formerly Amenhotep IV, completely reorganized all the religion in Egypt.
He worshiped Aten (a solar disc) instead of the old religion.
Obviously, this didn't ingratiate him with the priests of the time, and he fell out with them 5 years into ruling.
He was married to Nefertiti, and his son was Tutenkhamen (who was a very young king).
He constructed Amarna, and was succeeded by Smenkhare.

\textbf{Rameses~II, The Great} (1279--1213 BC) fought the Hittites at the Battle of Kadesh,
near the Orontes River, where over 6,000 chariots were used;
the battle was fought to a draw.
Rameses also suppressed the Shardana pirates in a naval battle.
His father was Seti I, who did some military things.
Rameses built the Abu Simbel temples and a temple at Luxor that houses the House of Life.
His wife was Nefertari, and the Greeks called him Ozymandias.
He built the Karnak complex and he was succeeded by Merneptah I when he was buried in the Valley of the Kings.

\section{Indus River Valley}

Further East, some small civilizations were cropping up in the Indus River Valley.

Notably, these include Harappa and Mohenjo-Daro, which date to the 27th century BC\@.
Mohenjo-Daro had a famous Great Bath and a College of Priests.
These were the centers of the civilization.
