\chapter{Expansionism in Europe}

\epigraph{%
  I have conquered an empire but I have not been able to conquer myself.
}{Peter the Great}

In the 18th century, European states were truly expanding.
The population was increasing, the Enlightenment was fostering intellectualism, and war was rampant.
But to see what happened in the 18th century, first we need some background in the 17th century,
in Russia and Prussia.

\section{Ascension of the Romanovs}

Last time we looked at Russia,
it had descended into a Time of Troubles following the death of Boris Godunov.

\subsection*{Michael Romanov}

In 1613, \textbf{Michael~I Romanov} (Mikhail~I Romanov), son of the patriarch Filaret,
was elected tsar by the Zemsky Sobor,
ending the Time of Troubles and founding the House of Romanov.
The early part of his reign, while he was a child,
was dominated by the Saltykovs from his mother's side of the family.
The Romanovs would stay in power until 1917.
The Russians moved to peace with their enemies,
and Michael signed the Peace of Stolbovo with Sweden.
Michael died in 1645.

After Michael died, there were multiple peasant disorders in Russia.
In the mid to late 17th century, rebellions included the
Salt Riot, Copper Riot, and the Moscow Uprising.
In 1667, the Cossacks, free settlers of South Russia, led by Stenka Razin,
went up the Volga River, but eventually Stenka was caught and killed.
Later, Stepan Razin would be supported by the streltsy in a 1670s uprising.
It's probably useful to note that of the four great Russian rebellions at this time,
all were led by Cossacks --- we mostly care about Razin and Pugachev (we'll see him later)

\subsection*{Peter the Great}

\textbf{Peter the Great} (1672--1725) was an important part of modernizing Russia.
When he was a child, he got military experience from playing with his ``toy army'' in war games.
He originally shared power with his brother Ivan~V, who had severe mental and physical disabilities.
By the time he became tsar, Russia was the biggest country in the world.

In 1689, the streltsy, musketeers that Ivan the Terrible had organized when he was tsar,
revolted to put Peter's half sister Sophia on the throne.
Peter put down the rebellion with the help of his advisor Patrick Gordon.
Peter's first attacks were against the Ottoman Turks,
and he wanted to take Azov, near the Black Sea.

When Peter wanted a better ``window to the West'',
he made an alliance with the Polish-Lithuanian commonwealth
and started the \textbf{Great Northern War} against Charles~XII of Sweden.
Sweden was able to stop attacks at Travendal and the \textbf{Battle of Narva} (1700),
and pushed back through to Saxony where, after the Battle of Klisz\'ow,
it forced Augustus~II of Poland-Lithuania to accept the Treaty of Altranstadt.
The treaty ensured the execution of Johann Patkul, who'd made the alliance in the first place.

Peter pushed west and in 1703, he built \textbf{St.\ Petersburg} on the Baltic Sea.
Charles moved into Russia and met Peter's forces at the \textbf{Battle of Poltava} (1709).
At the battle, Ivan Mazepa, a Cossack commander in the Russian army,
deserted and went to the Swedish side
after learning that he was going to be replaced by Alexander Menshikov.
The Russian army under Menshikov crushed the Swedish forces at Poltava
and Charles was exiled to the Ottoman Empire (Moldova, to be precise).
Charles would later be killed at the Siege of Fredriksten.
In 1714, at the Battle of Gangut, Russia won its first ever naval victory.
The war ended with Sweden defeated and Russia as a dominant power in the world.
The war concluded in 1721 with the Treaties of Stockholm, Treaty of Frederiksborg,
and the \textbf{Treaty of Nystad}, in which Russia got Estonia and East Ingria.

Peter modernized and reorganized his governmental structure.
He turned it into a more centralized state, and he replaced the \textbf{Duma},
the old council of \textit{boyars} (nobles), with his new nine-man senate.
In 1722, he set up the Table of Ranks, which formalized positions in the military and public office.
He created the Holy Synod, a collective body led by a government official.
Peter imposed a beard tax as part of his war on facial hair.
He took a Grand Embassy through Europe, working as a carpenter for Lynst Rogge in a Dutch shipyard.

Peter launched the Russo-Persian war in 1722 to take down the Safavid Empire in Persia.
Even though the Safavid lost a lot of territory,
they got it back when Nader Shah, the new leader,
signed the Treaty of Resht to create an alliance against the Ottomans.

\section{Rise of Prussia}

\subsection*{Background}

The \textbf{Hohenzollerns} were an important part of the history of Prussia.
Around the time of the Thirty Years' War, the Mar\-grave George William fled to Konigsberg.
His successor was \textbf{Frederick William~I, The Great Elector}.
Frederick William~I reformed the army and paid homage to Wladyslaw~IV (House of Vasa) of Poland,
in return for the Duchy of Prussia.
With the duchy in hand, the Hohenzollern dynasty had a territory all their own.
Frederick William~I organized an absolutist monarchy in Brandenburg-Prussia
and issued the Edict of Potsdam,
which opened the borders for Protestants (including French Huguenots).

The Great Elector's son, Elector Frederick~III,
crowned himself King \textbf{Frederick~I} of Prussia,
and the Holy Roman Emperor \textbf{Leopold~I}
allowed him to do so even though Prussia was largely in the empire.
Frederick~I sponsored the arts, and Prussia grew.
Frederick~I was succeeded by King \textbf{Frederick William~I}, the Soldier King.
He put together a better standing army, and they saw a little action in the Great Northern War.
In the Treaty of Stockholm, he got half of Swedish Pomerania.

\subsection*{Frederick the Great}

When his father Frederick William died in 1740, \textbf{Frederick~II, the Great} took the throne.
He built the Sans Souci Palace, near Berlin.
He also created the \textit{Furstenbund}, a Protestant alliance of German princes.

At the time, Prussia was still mostly a scattered mess of territories.
When Frederick took the throne, he wanted to take the Austrian province of Silesia.
Holy Roman Emperor \textbf{Charles~VI} had issued a \textbf{Pragmatic Sanction} in 1713,
stating that Habsburg domains in Austria would be inherited by his daughter, \textbf{Maria Theresa}.

Frederick disputed the claim because he wanted the money in Silesia.
He won the First Silesian War, which ended with the Treaty of Breslau.
The three Silesian Wars were part of the \textbf{War of the Austrian Succession} (1740--1748),
known in the Americas as King George's War.
Frederick was worried that Augustus~III of Poland would take Silesia.
As is often the case with wars of succession, a lot of other countries were drawn into the conflict.
George~II of Britain led troops at the Battle of Dettingen
(the last time a British monarch would ever do so).
One part of the war was the First Carnatic War.

In another sub-conflict,
some Spanish coast guards boarded a British merchant ship and cut off the captain's ear,
resulting in the \textbf{War of Jenkins's Ear}.
Other notable battles in the war included Fontenoy and Mollwitz.
Frederick's forces also won at Hohenfriedberg.

The war was ended by the \textbf{Treaty of Aix-la-Chapelle}.
It confirmed Maria Theresa as ruler of Austria and Hungary, while Prussia kept Silesia.
The treaty also reasserted Britain's right to the \textit{asiento} (slave trade),
which was a big part of their motivation for war.

Later, Frederick was involved in the 1778--79 \textbf{War of the Bavarian Succession},
the ``Potato War'', against the Austrians.
Frederick wanted to prevent the Habsburgs from taking control of Bavaria.
The Treaty of Teschen ended the war without any major battles;
however, about 10,000 soldiers on each side died of starvation and disease.

\section{Seven Years' War}

The \textbf{Seven Years' War} (1756--1763)
was one of the most important conflicts of modern history.
It involved much of the world, and it shaped history in all of its theaters.
It was composed of:
French and Indian War (US and Canada),
Pomeranian War (Sweden and Prussia),
Third Carnatic War (India),
and Third Silesian War (Prussia and Austria)

\subsection*{Origins}

Great Britain and France started fighting in 1754 because of tensions in the New World.
At the same time,
Prussia was in a conflict with Austria (again) over lands and dominance in the Holy Roman Empire.
In 1756, Prussia allied with Britain, and France and Austria allied against them.
The Anglo-Prussian alliance was joined by Portugal and some German states,
and the French and Austrians were joined by Sweden, Saxony, and Spain.
Russia joined Austria at the outset, but then switched sides when Peter~III came to power there.

As far as notable battles in Europe go: at the Battle of Leuthen, Charles of Lorraine was defeated.
At the Battle of Wandiwash, Sir Eyre Coote was defeated by Count de Lally.
And, at the Battle of Rossbach,
Charles de Rohan and Prince Joseph of Saxe-Hildburghausen were defeated.
Baron von Seydlitz was promoted after his victory at the Battle of Kolin.

\subsection*{French and Indian War}

The American theater of the war was the \textbf{French and Indian War}
because the Native Americans allied with the French against British control in America.

During the French and Indian War,
the \textbf{Braddock Expedition} (1755) tried to capture Fort Duquesne,
but it failed at the Battle of Monongahela, where Edward Braddock was killed.
Notably, \textbf{George Washington} was the American aide to Braddock during the expedition.
After the disastrous expedition,
Governor Dinwiddie of Virginia commissioned Washington as Commander of the Virginia Regiment.

The \textbf{Battle of the Plains of Abraham}
in Quebec was the major victory in the American theater.
At the battle, the Marquis de Montcalm and James Wolfe both died;
the latter inspired a famous painting by Benjamin West.
The French had regrouped near Bougainville across the St.\ Charles River,
and the battle itself lasted less than an hour.

\subsection*{Resolution}

The war was a great success for William Pitt the Elder, who strategized for the British.
Britain took possession of Canada in the \textbf{Treaty of Paris}.
One of the notable negotiators for the treaty was John Stuart.
The \textbf{Treaty of Huburtusburg} allowed Frederick the Great to keep Silesia.

\section{Enlightenment}

The Enlightenment was very influential in the mid 18th century.
It brought a new way of thinking about philosophy and social science to the world.
Thinkers of the era included Locke, Rousseau, and Montesquieu,
but we'll leave discussion of them to someone who cares.
Let's look at how the Enlightenment affected history.

\subsection*{Catherine the Great}

\textbf{Catherine~II, the Great} came to power in Russia in 1762,
having engineered a coup that killed her husband Peter~III\@.
Her accession to the throne was aided by her favorites:
Count Grigory Orlov and \textbf{Grigory Potemkin}.
Potemkin notably made fake villages to impress the empress
after ousting Alexander Vassilchikov and helping win the Russo-Turkish Wars.
Catherine's generals included Alexander Suvorov and Pyotr Rumyantsev,
and one of her admirals was Fyodor Ushakov.

Catherine crushed the Ottomans in the Russo-Turkish Wars,
and after the Battle of Chesma, she was able to take Azov and the Crimea.
She helped Stanislaw Poniatowski acquire the throne in Poland,
but then she, along with Frederick the Great and Maria Theresa,
partitioned the Polish-Lithuanian commonwealth amongst their three countries,
with Russia getting the biggest chunk.
Under Catherine, Russia started the colonization of Alaska.

She continued to modernize Russia in the style of Peter the Great,
but serfdom was still an important part of the economy.
Catherine faced \textbf{Pugachev's Rebellion} (1773--1775),
a large scale uprising of Cossacks and peasants
which she had to have her general Suvorov put down.
The rebellion was led by Yemelyan Pugachev, a Cossack pretender to the throne.
At the Battle of Kazan,
the rebels took the city but were then beaten by the government led by Peter Panin.
Alexander Pushkin wrote about Pugachev in his
\textit{The History of Pugachev} and \textit{The Captain's Daughter}.

Partially because of the Enlightenment, Catherine's Russia was in a golden age.
She defined her rule in much the same way Elizabeth~I had done in England.
Catherine is a notable example of an \textbf{enlightened despot} ---
she spent time with the ideas of the Enlightenment,
frequently corresponding with Diderot and Voltaire,
who called her the ``Star of the North'' and the ``Semiramis of Russia''.
Catherine wrote the \textit{Nakaz},
a set of legal principles influenced by the French Enlightenment.
She also established the Smolny Institute, the first place of higher learning for women in Europe.
Her Free Economic Society for the Encouragement of Agriculture and Husbandry
was Russia's first learned society and became an important part of liberalism in Russia.

Catherine died of a stroke in 1796.
She was succeeded by her son, \textbf{Paul~I} of Russia.

\section{Hanoverian Succession}

\subsection*{Anne}

\textbf{Queen Anne} (House of Stuart) was Queen starting in 1702.
Under the \textbf{Acts of Union} which united England and Scotland,
she became the monarch of the new united state called Great Britain.
She favored the more moderate Tory politicians because she was Anglican,
and during the War of the Spanish Succession, she dismissed many of their opponents, the Whigs.
Anne died childless in in 1714.

She was succeeded by her second cousin \textbf{George~I} of the \textbf{House of Hanover}.
He was descended from James~I, through the Stuarts,
because the Act of Settlement of 1701 prevented Catholics from taking the throne.
He was succeeded by \textbf{George~II}, who in turn was followed by George~III\@.

In 1721, \textbf{Robert Walpole} became the first Prime Minister of Great Britain.
He'd previously been Chancellor of the Exchequer,
during which time he'd created a ``sinking fund'' to try and reduce national debt.
He managed to keep the position of prime minister for 20 years straight ---
some people call the period the ``Robinocracy''.
He covered up a bunch of scandals including the South Sea Bubble (during which he came to power),
leading to his sometimes being called the ``Screenmaster General''.
Walpole started using 10 Downing Street as the primary residence for the Prime Minister.
His government collapsed partly as a result of Britain's entrance into the War of Jenkins' Ear,
which Walpole was somewhat forced into joining against his will.

\subsection*{George~III}

\textbf{George~III}, King of Great Britain and Ireland, grandson of George~II,
son of Frederick Prince of Wales,
reigned from 1760 to 1820.
During his reign, Great Britain and Ireland formed the United Kingdom of Great Britain and Ireland
under the Acts of Union (1800).
George was the longest reigning monarch up to that point,
and his wife was Charlotte of Mecklenburg-Strelitz.

The early parts of his rule were marked by the Seven Years' War.
The Whigs didn't really like him because they thought he favored the Tories.
During George's reign, the American Revolution took place, but we'll look at that later.

Later in his reign, George appointed Lord Shelburne to the Prime Minister's position.
Charles James Fox, however, didn't like Shelburne and feuded with George constantly.
Nevertheless, George approved Fox's appointment to the ``ministry of all the talents'',
which was eventually disbanded under Baron Grenville.
George developed porphyria later in life and died in 1820.
