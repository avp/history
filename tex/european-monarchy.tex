\chapter{European Monarchy}

The 17th century was full of war and crisis.
The monarchs were powerful, and there's no question history from now on gets a bit more complicated.

\section{Thirty Years' War}

Remember that France had just been through a lot, with its wars of religion.
Protestantism was still making waves, and even though the Peace of Augsburg helped a little bit,
there was still quite a bit of unease running through Europe.
Calvinism still wasn't really recognized in the Peace.

At this point,
Spain wanted the German states because of territory in the Netherlands,
France was uneasy about its proximity to the Habsburg controlled Spain and Holy Roman Empire,
and Sweden and Denmark felt like they wanted some more land.
The Holy Roman Empire was all over the place, but it was heavily fragmented.

The \textbf{Thirty Years' War} is generally thought of as having four phases:
the Bohemian Revolt,
the Danish intervention,
the Swedish intervention,
and the French intervention.

\subsection*{Bohemian Revolt}

Archduke Ferdinand~II of Austria was the Crown Prince in Bohemia.
He was Catholic, and the Protestants in Bohemia weren't particularly pleased with that.
In 1618, they tossed some of his representatives out of a window in the \textbf{Defenestration of Prague}.

The Defenestration of Prague began the Bohemian revolt,
and the conflict would move across the entirety of Europe.
The Protestant leaders moved the war to western Germany,
and Philip~IV of Spain was called to assist.

The Ottoman Empire decided to help out the Bohemian Protestants after they elected Frederick~V as a king,
because the Ottomans were never particularly fond of Catholics.
This triggered the minor Polish-Ottoman War (1620--1621),
and the Ottomans defeated the Poles at the Battle of Cecora.
But, the Ottomans didn't help the Bohemians at the \textbf{Battle of White Mountain} (near Prague)
where Christian of Anhalt was defeated by Ferdinand~II, Holy Roman Emperor and the Count of Tilly.

\subsection*{Danish Intervention}

In 1625, Christian~IV of Denmark decided to help the Lutherans, resulting in the Low Saxon War.
In response, Ferdinand~II hired Albrecht von Wallenstein, a rich Bohemian,
who let Ferdinand use his army.
Wallenstein and Tilly pushed Christian back.

Wallenstein moved north and took parts of Denmark, but wasn't able to take Copenhagen.
So after a bit more fighting, a small Treaty of Lubeck put the Danish Intervention mostly to rest.

After a War of Mantuan Succession between France and the Habsburgs in Italy,
a new black death swept across northern Italy.

\subsection*{Swedish Intervention}

In 1630, \textbf{Gustavus Adolphus}, king of Sweden, invaded Europe.
The Swedish armies pushed the Catholics back, and Gustavus Adolphus took much of the land for himself.
France and Bavaria allied with the 1631 Treaty of Fontainebleau.
At the Battle of Breitenfeld, the Swedes defeated Tilly, who would end up dead soon.
Other notable Swedish victories around this time included the Battle of the Rain.

Now that Tilly was dead, Ferdinand~II asked Wallenstein for help again.
The Swedes met Wallenstein at the \textbf{Battle of L\"utzen},
but even though they won, Gustavus Adolphus died.
Near the end of this phase, Ferdinand~II signed the Peace of Prague,
ending the civil war-like aspects of the war.
Later, because of treachery against the Holy Roman Emperor,
Ferdinand~II would have Wallenstein assassinated.

\subsection*{French Intervention}

France was a Catholic state; recall their difficulties with the Protestant Huguenots.
However, their hatred of the Holy Roman Empire and the Habsburgs who controlled it and Spain
transcended religious boundaries.
Cardinal Richelieu, Chief Minister, had already subsidized Gustavus Adolphus's invasion.

Emperor Ferdinand~II died in 1637, and Ferdinand~III succeeded him.
The new emperor wanted to end the war using negotiation, and he convened the Imperial Diet.
In Flanders, the French defeated the Spanish at the Battle of Rocroi
(soon after the death of Louis~XIII), and proceeded to take all of Flanders.

In 1642, Richelieu died, but his war continued.
They continued to push through, winning at battles such as Nordlingen.
By 1648, battles were coming all the way to Prague, where the final battle was fought.

\subsection*{Peace of Westphalia}

It took four years for the war to be resolved.
Much of the work was done at Osnabruck and Munster, in Westphalia.
Other treaties that were part of the final \textbf{Peace of Westphalia} (1648)
included the Treaty of Hamburg.
The war was decided in favor of letting countries deal with religion for themselves.

\section{Stuart England and the Protectorate}

\subsection*{James~I}

When Elizabeth~I died in 1603, her closest Protestant relative was
James~VI of Scotland, who became crowned \textbf{James~I} of England in the Union of the Crowns.
He quickly made peace with the Spanish
and the English didn't really deal with the rest of Europe for a good part of the 17th century.
In the Spanish Match, he tried to find a Habsburg Infanta wife for his son Charles, but that failed.
He was called the ``wisest fool in Christendom'', because he was educated but foolish.
He wrote the The True Law of Free Monarchies and the Basilikon Doron.
James also started the North Berwick witch trials.
There were people who weren't particularly happy with the Protestant James as king,
and they tried to kill him every once in a while, notably with the Main Plot and the Bye Plot in 1603.

On the fifth of November 1605, a group of Catholics led by Sir Robert Catesby schemed to kill James~I.
This \textbf{Gunpowder Plot} aimed to blow up Parliament.
\textbf{Guy Fawkes}, a conspirator, rented a room in the Whynniard House
under the name ``John Johnson'' from Thomas Percy in preparation for the event.
When Robert Cecil found a warning letter from Francis Tresham to his brother-in-law William Parker, Lord Monteagle,
they looked and found Fawkes with 36 barrels of gunpowder under the Houses of Parliament.
In the aftermath, the principal Jesuit in England, Henry Garnet was convicted of treason and killed.
Now, Guy Fawkes Day is commemorated with fireworks and bells and such.

\subsection*{Charles~I}

James~I was succeeded by \textbf{Charles~I} in 1625.
At this point, the Parliament didn't really have a lot of power,
and it was summoned whenever the king felt like it.
However, it was particularly useful in that it could raise taxes.
In 1628, a new Parliament drafted the Petition of Right, which further restricted Charles's power.

For the next ten years, Charles didn't call Parliament in a period called the ``Personal Rule''.
He believed in High Anglicanism,
which he supported through his appointment of William Laud as Archbishop of Canterbury.
His religious beliefs resulted in the Bishops' War, an uprising in Scotland,
which he stopped with the Pacification of Berwick.

He eventually did need more money so he called a new Parliament in 1640, led by John Pym.
They weren't nice enough to Charles and he disbanded the Parliament after a few weeks,
which is now called the \textbf{Short Parliament}.
Then, Charles went back into Scotland but didn't do very well.

In November 1640, he called another Parliament under pressure (the \textbf{Long Parliament}).
They started to air their grievances
and made it so that a Parliament must convene at least every 3 years (Triennial Act).
They accused Thomas Wentworth, Earl of Strafford, of treason.
Charles tried to capture five members of Parliament, but he failed.

\subsection*{English Civil War}

In 1642, after the failed capture by Charles,
an \textbf{English Civil War} broke out between Parliament and the crown.
The Parliament was known as the \textbf{Roundheads}, and Charles's forces were the \textbf{Cavaliers}.
The first battle of the war was fought at Edgehill in 1642, which ended inconclusively.
After the Battle of Turnham Green, Charles was pushed back to Oxford.

At the \textbf{Battle of Marston Moor} in 1644, the Parliament won with the help of the Scots
and the strong leadership of \textbf{Oliver Cromwell}.
Parliament proceeded to create a \textbf{New Model Army} under Sir Thomas Fairfax and Cromwell.
At the \textbf{Battle of Naseby} in 1645, the Parliament destroyed Charles's forces.
The First English Civil War ended with the imprisonment of Charles.
In 1647, Cromwell and others debated the Putney Debates against the Levellers,
who wanted more political equality.

Charles escaped, and the short Second English Civil War was a short conflict,
which ended with the New Model Army securing the country.
The Long Parliament was disbanded in a coup known as \textbf{Pride's Purge},
and Charles was tried at the \textbf{Rump Parliament},
after which his head was separated from him in 1649 at Whitehall Gate in London.

\subsection*{Oliver Cromwell, Lord Protector}

After Charles was killed, Cromwell proceeded to subjugate Scotland, Ireland, and the rest of England.
In Ireland, he won the Siege of Drogheda (DROY-eh-da) in 1649, killing 3,500 people.
In 1653, John Lambert wrote the Institution of Government
and Cromwell was given the title of Lord Protector for life,
making him ``king in all but name''.
He disbanded the Rump Parliament, and he established the Barebone's Parliament.

Cromwell ended the First Anglo-Dutch War, and he died in 1658,
and power passed to his son Richard Cromwell.
But, Richard was ineffective and the Protectorate of his father crumbled around him.

\subsection*{Restoration: Charles~II}

In 1659, Richard Cromwell was forced to abdicate.
The newly elected Parliament let the exiled king back from the Netherlands,
and he sailed from Scheveningen in 1660.
The monarchy was restored to England, and \textbf{Charles~II} (The ``Merry Monarch'') became king.
He moved back from the Puritans who had been in charge.
He had many mistresses, including the ``pretty, witty'' Nell Gwyn;
he had at least twelve children, but none by his wife, Catherine of Braganza.

Parliament passed the Clarendon Code, which was comprised of
the Conventicle Act,
the Corporation Act
and the Act of Uniformity.
It aimed to make the Church of England the dominant faith in Britain.
In 1665, the Great Plague of London swept through the city,
and Charles and his family fled to Salisbury.

In 1666, the \textbf{Great Fire of London} started in a bakery in Pudding Lane.
It burned down much of the city, including St.\ Paul's Cathedral,
which would have to be restored by \textbf{Christopher Wren}.
It was chronicled, along with much of Charles's reign, in the journal of \textbf{Samuel Pepys} [peeps].

Charles's brother James was the heir because of Catherine's inability to have kids.
In 1678, Titus Oates, a priest, warned Charles of the ``Popish Plot'' which aimed to kill him.
He was making it all up, but Charles had his minister Lord Danby investigate, causing much panic.
Charles would later have Lord Danby imprisoned for treason and other crimes.

\subsection*{Glorious Revolution}

Charles died in 1685 and his Catholic brother \textbf{James~II} took the crown.
James's wife was Mary of Modena.
After James Scott tried to overthrow James in the Monmouth Rebellion,
James started a set of trials called the Bloody Assizes following the Battle of Sedgmoor.
He set out the Declaration of Indulgence, a step toward religious freedom,
but it was opposed by seven Anglican bishops, who were upset by it.

People weren't happy with James,
so a group called the Immortal Seven asked his Protestant daughter
\textbf{Mary} and her husband \textbf{William~III of Orange} to take power in the country.
In the \textbf{Glorious Revolution}, beginning in 1688, William landed in England and was crowned.
James tried to fight back in the Williamite War,
but William defeated James at the \textbf{Battle of the Boyne} in Ireland in 1690.

In 1689, the \textbf{English Bill of Rights} was passed.
It stated that monarchs could not be Catholic, and it gave the Parliament more power.
But, Catholics loyal to James were still trying to take the throne back.
This resulted in 1692 at the Massacre of Glencoe.
Jacobite rebellions continued until the forces of
\textbf{Bonnie Prince Charlie} were defeated at the Battle of Culloden, in 1746.

\section{France: Louis and Louis}

\subsection*{Louis~XIII and Cardinal Richelieu}

When Henry~IV was killed in 1610, \textbf{Louis~XIII} succeeded him.
He was only nine, so his mother \textbf{Marie de' Medici} was his regent.
But, she wasn't very competent so she was exiled by the king in 1617,
and he proceeded to kill a bunch of her followers, such as Concino Concini.
He then worked closely with chief minister \textbf{Cardinal Richelieu},
who succeeded the Duke of Luynes in the position.
Richelieu had previously implemented the reforms of the Council of Trent in France.

At one point, Richelieu was exiled to Avignon,
and in another event, he had to run to Blois after Concino Concini was killed.
He created the Company of One Hundred Associates,
and he started the ``reformation of the third order of the realm''.

In 1627, in an attempt to defeat the Huguenots, Richelieu ordered the \textbf{Siege of La Rochelle}.
The city withstood a whole year before surrendering in 1628.
In 1630, Marie and enemies of Richelieu tried to get the king to dismiss Richelieu.
They thought they were successful, but Henry kept Richelieu;
this event came to be called the Day of the Dupes.
This shows how reliant Richelieu was on Henry for his power.

Louis died in 1543, just a few days before the battle of Rocroi.
His son succeeded him.

\subsection*{Louis~XIV}

\textbf{Louis~XIV}, known as the ``Sun King'', was one of the most important French kings.
He reigned from 1643 to 1715, in the longest reign of any monarch so far in European history.
When his father first died,
Louis's mother \textbf{Anne of Austria} was in charge with Richelieu's successor, \textbf{Cardinal Mazarin}.

Louis's minority included the \textbf{Fronde} immediately after the Peace of Westphalia,
in which a bunch of nobles rebelled against Mazarin.
It was called the Fronde after the slings people used to smash windows during it.
There were two phases to the Fronde: the phase of the Parlements and the phase of the Princes.
The first phase was put to rest by the Peace of Rueil,
and the Battle of Rethel was the decisive battle of the Princes' phase.
The 1652 Battle of Faubourg St.\ Antoine took place next to the Bastille.

Louis~XIV truly came to the crown in 1661 when Mazarin died.
He started with fiscal reform by appointing \textbf{Jean-Baptiste Colbert} as his finance minister.
In order to do this, he first neutralized the Superintendent of Finances, Nicolas Fouquet,
by convicting him of embezzlement after a feast at the Chateau of Vaux-le-Vicomte.
Colbert proceeded to lessen national debt and improve taxation.
Louis's other advisors included Michel Le Tellier and Hugues de Lionne.

Under Louis~XIV, the Midi Canal was built while he was at Languedoc,
and he also built the Royal Mirror-Glass Factory.
In 1685, Louis issued the \textbf{Edict of Fontainebleau},
revoking the Edict of Nantes and revoking privileges for Protestants that it gave.

Early in his rule, he participated in the War of Devolution,
in which he fought with Habsburgs in the Spanish Netherlands.
The war was ended by the Treaty of Aix-la-Chapelle.

Louis was also involved in the \textbf{War of the League of Augsburg},
also called the Nine Years' War (1688--1697).
In 1685, the Elector Palatine Charles~II died; Maximilian, an ally of France, died in 1688.
In order to stop Louis from becoming too powerful, the Holy Roman Emperor organized a League of Augsburg.
When William and Mary took the throne in England,
he took up arms against the French and the League of Augsburg became known as the \textbf{Grand Alliance}.
The war ended with the \textbf{Treaty of Ryswick} in 1697 when France was mostly exhausted by war.

\textbf{Charles~II of Spain} ruled a large empire, encompassing Spain, Milan, parts of the Netherlands, etc.
But, he had no children, and when he died, he reneged on an agreed will,
giving the entirety of the empire to Philip, Duke of Anjou.
Louis~XIV decided to accept the new will, and Philip became King \textbf{Philip~V} of Spain.

This triggered some tensions in Europe, and resulted in the \textbf{War of the Spanish Succession} (1701--1714).
The French started by winning, but the work of John Churchill, Duke of Marlboro, and Eugene of Savoy repelled him.
The Austrians and the Palatinate took Bavaria after the Battle of Blenheim.
Maximilian~II, Elector of Bavaria, fled.
Other important battles included Ramillies, Turin, and Oudenarde.
France and Spain won at Villaviciosa and Brihuega,
and the Allies won a Pyrrhic victory at the Battle of Malplaquet.
France eventually won at Denain and regained their momentum.
The \textbf{Treaty of Utrecht} in 1713 brought peace to France, Spain, Britain, and the Dutch.
Afterwards, the Holy Roman Emperor made peace in the Treaties of Rastatt and Baden.
Spain would later stop its attempts at conquest after losing the War of the Quadruple Alliance,
long after Louis was dead.

Louis~XIV died of gangrene at Versailles, the palace he built, in 1715,
and he was succeeded by his great-grandson, the five year old Louis~XV\@.
