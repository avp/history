\chapter{Early Modern Asia}

\epigraph{%
  After victory, tighten the cords of your helmet.
}{Tokugawa Ieyasu}

We've been focused mostly on Europe for a while,
but let's take a moment to look at what was happening in the east.
This will be a short chapter, but it's important as a transitional period in history.

\section{Tsars of Russia}

Russia is in both Asia and Europe,
and because the Renaissance and Reformation chapter doesn't really deal with Russia,
it's in this chapter on Asia.

\subsection*{Background}

The Russian princes trace their heritage back to \textbf{Rurik}.
The princes ruled over Kievan Rus',
a predecessor to the modern Russian state that was centered around lots of smaller polities,
notably Kiev and Novgorod.

The Mongols that we saw in the Middle Ages were also taking up a fairly large amount of space in Russia.
But the Mongols (Tatars) of the Golden Horde, from their capital at Sarai,
generally left the princes of south and east Russia alone as long as they paid their taxes.

The princes generally cared more about the Northern Crusaders in the 13th century,
who were led by the Teutonic Knights and the Swedes.
\textbf{Alexander Nevsky}, prince of Novgorod,
became famous after victories such as the Battle of the Ice against the Teutonic Knights.
Nevsky was hailed as a hero as a result of his exploits even though he often cooperated with the Mongols.
Under the Mongols, the Russians developed an infrastructure and became much more organized.

\subsection*{Grand Duchy of Moscow}

The son of Alexander Nevsky, Daniil Aleksandrovich, founded Moscow,
a polity that would force the Tatars from Russia.
In 1380, the Mongols were defeated at \textbf{Kulikovo Field} by Grand Prince Dmitry Donskoy.
The Russian Orthodox Church also moved to Moscow, leading to it being known as a ``Third Rome''
(the first two were Rome and Constantinople, the previous centers of Orthodox Christianity).

By the 15th century, princes of Moscow were going around and taking as much land as they could.
Most successful among these was \textbf{Ivan~III, the Great} (1462--1505), Grand Prince of Moscow,
who began the foundations for the Russian state itself.
He annexed Novgorod and Tver', and defeated the Lithuanian armies at the Battle of Verdosha River,
tripling the size of the Grand Duchy of Moscow.
He also notably greatly renovated the Moscow Kremlin, and he was called ``sovereign of all Russia''.
His attacks would begin the eventual complete destruction of the once great Golden Horde.

\subsection*{Tsardom and Troubles}

The power that had been centralizing in Moscow became truly great during the time of
Tsar \textbf{Ivan~IV, the Terrible} (1547--1584).
He was the first to be called ``tsar'', and he made his position stronger than ever before.
He was advised by a ``Chosen Council''.
He set up a new code of laws (the Sudebnik), and he created the Zemsky Sobor, the first Russian representative body.
He waged the \textbf{Livonian War} against Stephen Bathory and Sigismund~II of Livonia to take the Baltic,
but he failed and asked Pope Gregory~XIII for help.
During the Livonian War, the Order of the Brothers of the Sword were defeated.
Ivan also supposedly blinded the architect of St.\ Basil's Cathedral.

Later in life, Ivan proceeded to divide his lands into two, one of which was the \textit{oprichnina}.
His secret police, the Oprichniki,
conducted the massacre of Novgorod because Ivan was suspicious of Archbishop Pimen.
Ivan ended up killing his heir Ivan Ivanovich,
and was therefore succeeded by the largely incompetent \textbf{Feodor~I}.

Feodor's reign was mostly actually run by \textbf{Boris Godunov}, the brother of Feodor's wife Irina;
Godunov usurped the Tsardom after Feodor died childless.
He had won back territories from Sweden after the Truce of Plussa.
Notably, he was the first tsar to use Siberia to send people into exile.
Godunov's later reign was full of civil wars, such as one against Grigori Otrepiev, a False Dmitri.
His successor was only in power for two months before being killed.
Russia descended into the \textbf{Time of Troubles}.

Russia would end up mostly surviving the Time of Troubles, despite the Russo-Polish war and the Ingrian war.
We'll look at the reemergence of the Tsars later --- Russia will be more a part of European history from now on.

\section{Shoguns of Japan}

Conspicuously missing from topics thus far, Japan actually did things happening before the 16th century.
A lot of ancient Japanese history doesn't come up too often,
so we'll start with classical Japan, for which we'll go back in time a bit.

\subsection*{Classical Japan}

Classical Japanese periods are named after their capital cities.

During the Asuka period (538--710), the Japanese polities began to become a truly centralized state,
and the three kingdoms of Korea were still in conflict.
Buddhism moved to Japan.

The Asuka period was succeeded by the Nara period of the 8th century.
Nara was generally considered a golden age of prosperity and progress.

The final period of classical Japan is the \textbf{Heian period} (794--1185).
It's noted for art and literature, and Lady Murasaki Shikibu wrote the \textit{Tale of Genji}.
Notable aristocratic families included the Fujiwara clan, who took control of the imperial family.
They faced the Hogen Rebellion, Heiji Rebellion, and Gempei War.
The Gempei War ended with the Battle of Dan-no-ura, when Minamoto clan won.
Then, the first Shogun was Yoritomo of the Minamoto clan, signaling an end to the power of the Fujiwara.

\subsection*{Feudal Japan}

Medieval Japan was controlled by \textit{daimyo}, powerful families, and \textit{shogun}, military warlords.
The emperor was mostly a figurehead.

Recall that the Mongols tried to invade Japan in 1274 and 1281.
They were deterred by the \textit{kamikaze}, the famous typhoon that saved Japan.

Fast forward a bit past a bunch of mildly important stuff,
and we get to the Azuchi-Momoyama period (1569--1603).

The first great unifier of Japan was \textbf{Oda Nobunaga} (1534--1582).
Oda introduced musketeers into the Japanese military,
and he allowed Jesuits into the country to combat the Ikko and Tendai monasteries.

He was killed in a revolt in 1582,
and true unification was completed by his general \textbf{Toyotomi Hideyoshi},
the second great unifier of Japan.
He invaded Korea, but he was stopped at the Battle of Noryang by the Korean turtle ships.
Hideyoshi tried to conquer the Ming dynasty in China, but he died.

\textbf{Tokugawa Ieyasu}, a student of Oda Nobunaga, was a regent for Hideyoshi's heir.
He previously had won battles at Komaki, Nagakute, Temmokuzan, Anegawa, and Nagashino.
At the \textbf{Battle of Sekigahara}, Tokugawa defeated Ishida Mitsunari,
general of Toyotomi Hideyori (Hideyoshi's son).

Tokugawa became the third great unifier of Japan and the first leader of Japan's final shogunate.
The Tokugawa Shogunate's crest was three hollyhock leaves.
Tokugawa created a ``four-class'' system of warriors, artists, farmers, and merchants,
and he also made the sankin-kotai system of government.
He destroyed the Toyotomi clan completely after the Siege of Osaka,
where he used the excavation of moats (that he had filled in himself) as an excuse to start the attack.
The Tokugawa would start a period of seclusion (sakoku) because of the Shimabara Rebellion.
The Tokugawa Shogunate stayed in power for quite a while, so let's go look at India for now.

\section{Mughals of India}

In 1526, \textbf{Babur}, a descendant of Tamerlane,
had tried multiple times to take Samarkand, but had failed and decided to go away.
He set up shop in Kabul and then went through the Khyber Pass into India.
After he defeated Ibrahim Lodi at the First Battle of Panipat, Babur took most of the north of India.
He defeated the Rajputs at the Battle of Khanwa to increase his strength.
Babur's son was \textbf{Humayun}, who didn't have a very stable empire,
and was pushed into Persia by rebellions led by people like Sher Shah.

In 1555, Humayun was able to come back to India, but he tripped on some stairs and died.
Humayun's son \textbf{Akbar} succeeded him as a 12-year-old.
The young Akbar was assisted by his regent Bairam Khan,
and in 1556, he defeated Hemu the Grocer and his Sur armies at the \textbf{Second Battle of Panipat}.
After the battle, Akbar reportedly didn't behead Hemu,
and instead pointed to a picture that he had drawn of Hemu bleeding on the ground,
saying that he had already killed him, denying him any final honor.
Akbar built a new capital at \textbf{Fatehpur Sikri} (his ``city of victory''),
and he ended Rajput wars by marrying Mariam uz-Zamani.
Akbar made a system of \textit{diwans} to make sure that finances were handled correctly.
He built the Jama Masjid around the tomb of the mystic Salim Chishti,
and even named his eldest son after him.

Akbar's son \textbf{Jahangir} ruled a spectacular era of the Mughal empire.
But he was an opium addict and wasn't a great ruler.

Jahangir's son was \textbf{Shah Jahan}, who brought the Mughal empire to a new peak.
Shah Jahan waged war on the Deccan Plateau,
taking Bijapur and Golconda and attacking the Uzbegs and the Persians.
He rebuilt and renovated the Red Fort and he had a Peacock Throne with lots of gems on it.
He put the Koh-i-noor diamond in the \textbf{Taj Mahal}, a mausoleum he built in Agra for his wife Mumtaz.
Shah Jahan's eldest son was Dara Shikoh, and his younger son was \textbf{Aurangzeb}.

In a battle for succession, Aurangzeb killed Dara after the \textbf{Battle of Samugarh} in 1659.
He also won at the Battle of Deorai.
Shah Jahan was declared incompetent and spent the end of his life locked up by Aurangzeb in the Red Fort.
Aurangzeb levied the \textit{jizya} tax, and he killed Tegh Bahadur for not converting to Islam.

Following Aurangzeb's death, \textbf{Bahadur Shah~I} took power, killing his brothers in the process.
His policies were often aimed at crushing Sikhs.

\section{Ottomans of Turkey}

The Ottomans would be an important force throughout modern times until their fall after World War I\@.
Let's quickly go over their origins until the late 19th century,
when we'll look at them again in the context of that war.

\subsection*{Rise of an Empire}

The \textbf{Ottoman Empire} was founded in the early 14th century by \textbf{Osman~I}, son of Ertugrul.
When the \textbf{Seljuk Turk} Sultanate of Rum fell, Osman expanded the Turkish borders towards the Byzantines.
Osman's son was \textbf{Orhan}, who captured Bursa in 1324 and made it the new capital.
The Ottomans proceeded to capture Thessaloniki and they were victorious at Kosovo and Nicopolis.

In 1383, Murad~I created the \textbf{Jannissaries}, an elite infantry group that formed the bodyguards for the Sultan.
They started out as Christian child slaves and evolved into a great force.
However, by 1620 or so, they were corrupt and failing.
Mahmud~II would abolish the Janissaries during the Auspicious Incident, which killed over 6,000 of them.

In 1402, Tamerlane won the Battle of Ankara and took Sultan Bayezid~I prisoner.
At the end of a civil war in 1413, \textbf{Mehmet~I} took the rule, ending the Interregnum (\textit{Fetret Devri}).
At the Battle of Varna (1444), Murad~II defeated Polish and Hungarian forces.

\subsection*{Heights of Power}

\textbf{Mehmed~II}, son of Murad~II, besieged and conquered Constantinople in 1453, overthrowing Constantine~XI\@.
By the 16th century, the empire was expanding fast.
Selim~I defeated Safavid Persia at the Battle of Chaldiran, and he established rule in Egypt.

\textbf{Suleiman the Magnificent} (1494--1566), son of Selim~I, ``Kanuni'', the Lawgiver,
was the longest-reigning Sultan.
During the Safavid wars, he signed the Peace of Amasya with the Shah of Iran.
In 1521, Suleiman captured Belgrade.
He fought parts of Hungary in the Ottoman-Hungarian wars.
Notably, he won the \textbf{Battle of Mohacs} (1526), and took Hungary.
He followed up by besieging Vienna in 1529, but failed.
At the Siege of Guns (1532), he was stopped from taking Vienna again.

\subsection*{Modernization to Decline}

During the Tanzimat period (1839--1876), the government created a more modern army and reformed many things.
The Sultan Abdulmecid~I issued the Rescript of G\"ulhane (Rosehouse or Rose Chamber), which stopped tax farming.
The Constitution, called the \textit{Kanun-u Esasi}, was the apex of this reform period.
The Crimean War was a part of the contest in which European powers wanted to take parts of the empire for themselves.
