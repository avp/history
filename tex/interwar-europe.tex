\chapter{Interwar Europe}

\section{From Tsar to Stalin}

\subsection*{Seeds of Revolution}

By 1917, the Russian people were getting tired of the mediocre leadership.
They didn't trust Rasputin, food was scarce, people were dying in the war, and the tsar was losing support.
In late February of 1917, strikes began occurring,
and soon much of the labor force of St.\ Petersburg were no longer working.
Nicholas ordered the Duma to disband, and he ordered his troops to shoot people in the street,
triggering the \textbf{February Revolution}.
The revolution led to Nicholas's abdication, and a new provisional government took over.

The provisional government was initially led by Prince Georgy Lvov,
who coordinated with the Petrograd (St.\ Petersburg) Soviet in the ``Dual Power''.

\subsection*{Rise of the Bolsheviks}

\textbf{Vladimir Lenin} had fled to western Europe after multiple encounters with the law in Russia.
He had been active in the 1905 revolution, after which he had joined \textit{New Life}, a radical newspaper.
In April 1917, he took a train from Zurich back to Finland Station in St.\ Petersburg.
On the way, he wrote his ``April Theses'' in an attempt to start undermining the provisional government.

During the \textbf{July Days}, Lenin tried to organize protests against the government, but he failed.
However, his failure still had the effect of getting Prince Lvov out of power
and putting the government in the hands of \textbf{Alexander Kerensky},
the Socialist Revolutionary minister,
while Lenin himself ran off to Finland.

Kerensky's government was a bit more progressive than Lvov's.
He instituted freedom of speech, but Lenin's \textbf{Bolsheviks} weren't really happy with that either.
During Kerensky's government, the Kornilov Affair attempted to overthrow Kerensky.
Lavr Kornilov failed, and Kerensky removed him as commander-in-chief.
It's worth noting that Kornilov had originally inherited the post from Aleksey Brusilov
(the general who led the Brusilov Offensive in World War~I).

The Bolsheviks under Trotsky and Lenin decided to start the \textbf{October Revolution} in late 1917.
Under the slogan ``Peace, Land, Bread'', the Bolsheviks aimed to supplant Kerensky.
The revolution began when a blank shot from the cruiser \textit{Aurora} signaled an assault on the Winter Palace.
The Women's Battalion of Death was ordered to guard the Winter Palace from the revolutionaries.
The tsar and his family were sent to Yekaterinburg and subsequently executed.
Notably, John Reed wrote \textit{Ten Days That Shook the World} about the revolution.

\subsection*{Lenin and Trotsky}

Now that Lenin was in charge, he wanted out of the quagmire that the war had placed the country in.
He sent \textbf{Leon Trotsky} to negotiate the \textbf{Treaty of Brest-Litovsk} with the Germans.
Richard von Kuhlmann was the German representative at the talks.
The Russians gave up a good chunk of western Russia to the Germans,
and the treaty also provided for a new independent Democratic Republic of Armenia.
The districts of Kars, Bactum, and Erdehan were ceded back to the Ottomans.
However, the treaty would eventually be nullified by the Treaty of Rapallo anyway.

Lenin instituted the \textbf{New Economic Policy} (NEP), which taxed farmers following the Kronstadt Rebellion.
His Red Army was led by his good friend Trotsky, having successfully dealt with the moderate Mensheviks.
During this period of war communism, all industry and land was nationalized,
and the rich peasants, \textbf{kulaks}, benefited quite a bit.

\subsection*{Stalin's Soviet Union}

Lenin was followed as Soviet Premier by \textbf{Joseph Stalin}.
At this point, Trotsky had been exiled to Mexico to escape his enemies, so Stalin ordered him killed;
Trotsky ended up with an ice-axe in his skull, but he didn't die for a day after the attack.

In 1929, Stalin proposed the first \textbf{Five Year Plan}.
He did away with Lenin's NEP and controlled all economic activity through the collectivization of agriculture.
Stalin forced a million kulaks off of their land, leading to some kulaks killing their own livestock in defiance;
the event is sometimes called the ``liquidation'' of the kulaks.
Russia, once the poorest power of Europe, quickly industrialized.

Stalin also ordered the NKVD, a law enforcement agency led by Lavrenti Beria,
to carry out the \textbf{Great Purges}.
During the \textbf{Moscow show trials} following the assassination of \textbf{Sergey Kirov},
Andrei Vyshinsky targeted Old Bolsheviks and other people Stalin didn't like, as Stalin consolidated his power.
After the Trial of the Sixteen, \textbf{Grigory Zinoviev} was notably executed;
he had allegedly sent a letter to the British Communist party, costing Ramsay Macdonald a 1924 British election.
At one point during WWII, Stalin didn't agree to exchange prisoner Friedrich Paulus for his own son.
He invented the ``Doctor's Plot'' but prosecution stopped when he eventually died.

\section{Nazism in Germany}

\subsection*{Weimar Republic}

Germany had been utterly destroyed by the fallout of World War I\@.
The military was decimated, the country was going broke paying war reparations,
and the Germans had lost quite a bit of territory, both within and outside of Europe.

Following the abdication of the Kaiser,
the once great German empire became the \textbf{Weimar Republic},
a collection of 19 states governed by a leadership that was falling apart as soon as it took power.
Early in the Republic's rule, the Spartacists (Spartacus Revolt) tried to overthrow it.
The first president of the Weimar Republic was \textbf{Friedrich Ebert} of the Social Democrats (SPD),
but he was forced out of Berlin during the \textbf{Kapp Putsch}.
When Ebert was killed in 1925, \textbf{Paul von Hindenburg} (the general during World War~I) took office.

Hindenburg would defeat Communist Ernst Thalmann in elections twice.
The ``camarilla'' of four advisors included his son Oscar.
He appointed Franz von Papen to the chancellorship,
and Hindenburg himself would hold office as President until 1935.

\subsection*{The Rise of Hitler}

\textbf{Adolf Hitler} was an Austrian-born artist.
Unfortunately, the Vienna Academy of Fine Arts rejected him multiple times because of his ``unfitness for painting''.
As a result, when World War~I broke out, Hitler volunteered to serve in the army.
He was wounded at the Somme and was given the Iron Cross.
After the war, Hitler joined the DAP (German Worker's Party),
which soon changed its name to the NSDAP (National Socialist German Worker's Party), or Nazi Party.
Hitler managed to gain a decent following on account of his vitriolic speeches against his enemies.

At the 1923 \textbf{Beer Hall Putsch}, Hitler, aided by Erich von Ludendorff,
attempted a coup against the Weimar government.
He wanted to emulate Mussolini's March on Rome the previous year and take power.
Hitler and his SA stormed a beer hall in Munich, interrupting Gustav Ritter von Kahr,
who had been placed in power along with Hans Ritter von Seisser and Otto von Lossow.
Hitler took the three men into a back room with a gun and claimed that he had four bullets,
one for each of them and one for himself, if they did not cooperate with him.
During the putsch, sixteen demonstrators and three policemen were killed.
Hitler and Rudolf Hess were sentenced to five years in Landsberg Prison,
while Ludendorff was acquitted.

While in prison, Hitler wrote \textit{Mein Kampf (My Struggle)}, laying out his plans for changing Germany.
When he got out of prison, the NSDAP was banned in Bavaria, but he got the ban lifted.
Then, the US stock market crashed.
Hitler took full advantage of the shock to the German economy and got the NSDAP over 100 seats in the Reichstag.
In 1932, he ran against Hindenburg in the presidential election, and came in second.
This rise to prominence made powerful people in Germany urge Hindenburg to appoint Hitler as the new chancellor,
which Hindenburg reluctantly agreed to.
Hitler quickly started trying to get Hindenburg to dissolve the Reichstag.

In 1933, the Reichstag was set on fire.
The Nazis blamed a communist plot, but historians disagree on who actually set the fire.
Some claim that the communist Dutchman Marinus var der Lubbe set the fire,
and some Bulgarians were also put on trial for it: Blagoy Popov, Vasil Tanev, and Georgi Dimitrov.
Hitler then convinced Hindenburg to issue the Reichstag Fire Decree,
repealing many human rights, and forcing many communists into concentration camps.

The Nazis didn't get a majority in the elections of 1933, but Hitler still got enough votes to pass the Enabling Act.
He got full legislating power, and by 1934 he decided to get rid of his enemies.
Franz von Papen's recent speech at Marburg University antagonized Hitler.
In the \textbf{Night of the Long Knives} (codename Operation Hummingbird),
Hitler targeted \textbf{Ernst Rohm}, Kurt von Schleicher, and Gustav von Kahr, among others.
Rohm had been the Chief of Staff of the SA, but Hitler had been disagreeing with Rohm lately.

\subsection*{Third Reich}

A few days later, Hindenburg died, and Hitler became the new head of state --- F\"uhrer and Chancellor of Germany.
Let's look at some of the high ranking people in Hitler's new Third Reich.

\textbf{Heinrich Himmler} was the new commander of the SS and Gestapo,
as well as supervisor of the new concentration camps.
He was in charge of purging the SA in 1934,
and he set up an assassination attempt on Albert Speer.

\textbf{Hermann Goering} was appointed Commander-in-Chief of the Luftwaffe (Air Force) in 1935,
having been an ace fighter pilot in World War~I.
In fact, he had led the fighter wing that had once been commanded by Manfred von Richthofen, the Red Baron,
and he had won the Blue Max award.

\textbf{Joseph Goebbels} was Hitler's Rich Minister of Propaganda.
He founded a newspaper called \textit{The Assault} and he was appointed General Plenipotentiary for Total War.
Goebbels also confiscated what he called ``degenerate art'' and helped deport Jews from Berlin.

Hitler's regime attacked Jews in Germany, treating them as subhuman and boycotting their businesses.
In 1933, the Nazis established their first concentration camp at \textbf{Dachau} in Bavaria,
and they created the Nuremberg Laws in 1935 to ban Jews from holding important positions.

In 1938, Hershel Greenspan killed the Third Secretary Ernst vom Rath.
Goebbels said that Greenspan was part of a huge Jewish conspiracy and
called for an attack on Jewish homes and synagogues.
The many broken windows during the November attack led the night to be called ``The Night of the Broken Glass'',
or \textbf{Kristallnacht}.

\section{Fascist Italy}

In the early 1920s, Italy wasn't having a great time.
The country had high unemployment, was politically unstable, and the economy was falling apart.
The period was called the \textit{Biennio Rosso}, or Red Biennium,
and anarchists tried to take power in a few places.

Into this context of rife instability emerged the National Fascist Party.
Led by \textbf{Benito Mussolini} and his \textbf{Blackshirts},
Mussolini and the Fascists aimed to bring down the current government.
In October 1922, Mussolini demanded power.
When the government didn't immediately give it to him,
he took his Blackshirts and led a 30,000 man \textbf{March on Rome} to oust Prime Minister Luigi Facta.
Mussolini took power and he passed the Acerbo Law, resulting in the Fascists taking control of Parliament.

By 1926, Mussolini had essentially taken full power over the country,
making himself only directly responsible to King Victor Emmanuel~III\@.
In 1928, all political parties were outlawed and the Fascists took complete control.

Mussolini managed to come to an agreement with the Catholic Church.
He signed the Lateran Accord of 1929 with Pope Pius~XI,
recognizing the pope as the sovereign of Vatican City, an independent state.

Mussolini's goal was to make Italy a great European power again, like in the days of Ancient Rome.
He ordered the bombing and capture of Corfu after General Enrico Tellini was assassinated.
In 1935, Mussolini took it upon himself to invade Ethiopia.
This Second Italo-Abyssinian war resulted in France and Britain no longer trusting him,
and he took Italy out of the League of Nations.

\section{Spanish Civil War}



\section{French Republic}
% De Gaulle
